\documentclass[../Year1.tex]{subfiles}
\usepackage{import}
\usepackage{../../style/header}

\begin{document}

\chapter{Calculus}
\renewcommand*\thesection{\arabic{section}}
\lhead{MATH40004A Calculus}
Lectured by Professor Demetrios Papageorgiou \\ Typed by Yu Coughlin \\
Autumn 2023

\section*{Introduction}

\lecture{1}{Thursday}{10/01/19}

The following are suggested textbooks:
\begin{itemize}
    \item G F Simmons, Calculus with Analytic Geometry, 1995
    \item J Stewart, Calculus, 2011
    \item S Lang, A First Course in Calculus, 1986
    \item S Lang, Undergradute Analysis, 1997
    \item J Marsden and A Weinstein, Calculus I and Calculus II, 1985
\end{itemize}
\begin{note*}
    The actual majority of MATH40004A Calculus was a less formal and more example / application based derivation of the entirety of MATH40002 Analysis. 
    As all of this content can be found in the corresponding document for Analysis, it isn't included in here.
\end{note*}

\tableofcontents\pagebreak

\section{Lengths, volumes and surfaces}

\subsection{Lengths}

\begin{theorem}[Arc length]
    The \textbf{arc length} of the curve $y=f(x)$ along $[a,b]$ is given by \[
        \intd{a}{b}{\sqrt{1+(f'(x))^2}}{x}
    \]
\end{theorem}

\begin{theorem}[Distance and velocity of parameterised curves]
    If a curve is parameterised by $(x(t),y(t),z(t))$, the \textbf{distance travelled} from time $t_0$ to $t$ is given by: \[
        L(t) = \intd{t_0}{t}{\sqrt{\br{\frac{\dif x}{\dif t}}^2+\br{\frac{\dif y}{\dif t}}^2+\br{\frac{\dif z}{\dif t}}^2}}{x}
    \] which naturally leads to the velocity at $t$: \[
        v(t) = \sqrt{\br{\frac{\dif x}{\dif t}}^2+\br{\frac{\dif y}{\dif t}}^2+\br{\frac{\dif z}{\dif t}}^2}
    \]
\end{theorem}

\subsection{Volumnes and volumes of revolution}

\begin{theorem}[Volume]
    If the cross sectional area of a shape when cut by a plane at $x=x_0$ is given by $A(x_0)$ for all $x_0\in[a,b]$, the volume of the shape is given by \[
        V = \intd{a}{b}{A(x)}{x}
    \]
\end{theorem}

\begin{theorem}[Disk method]
    The \textbf{volume of revolution} of $y=f(x)$ about the $x$-axis from $x=a$ to $x=b$ is given by, \[
        V_x = \intd{a}{b}{\pi\br{f(x)^2}}{x}
    \]
\end{theorem}

\begin{theorem}[Shell method]
    The \textbf{volume of revolution} of $y=f(x)$ about the $y$-axis from $y=a$ to $y=b$ is given by, \[
        V_y = \intd{a}{b}{\pi\br{f^{-1}(x)^2}}{y} = \intd{a}{b}{2\pi xf(x)}{x}
    \]
\end{theorem}

\subsection{Surfaces}

\begin{theorem}
    The \textbf{surface area of revolution} of $y=f(x)$ about the $x$-axis from $x=a$ to $x=b$ is given by, \[
        S_x = \intd{a}{b}{2\pi f(x)\sqrt{1+(f'(x))^2}}{x}
    \]
\end{theorem}

\begin{theorem}
    The \textbf{surface area of revolution} of $y=f(x)$ about the $y$-axis from $y=a$ to $y=b$ is given by, \[
        S_y = \intd{a}{b}{2\pi x\sqrt{1+(f'(x))^2}}{x}
    \]
\end{theorem}

\subsection{Centres of mass}

\begin{theorem}[1D discrete case]
    If we have a system of $n$ particles each with mass $m_k$ and position $x_k$ we can define the \textbf{centre of mass} at $\bar{x}$ by \[
        \bar{x} = \frac{\displaystyle\sum_{k=1}^nm_kx_k}{\displaystyle\sum_{k=1}^nm_k}
    \]
\end{theorem}

\begin{theorem}[2D continuous case]
    If we have a region limited by $f(x)$ and $g(x)$, give $g(x)\leq f(x)$ for all $x\in[a,b]$, with uniform mass, the coordinates of the \textbf{centre of mass}, $(\bar{x},\bar{y})$ is \[
        \bar{x} = \frac{\displaystyle\intd{a}{b}{x(f(x)-g(x))}{x}}{\displaystyle\intd{a}{b}{f(x)-g(x)}{x}} \qquad
        \bar{y} = \frac{\displaystyle\intd{a}{b}{f(x)^2-g(x)^2}{x}}{\displaystyle2\intd{a}{b}{f(x)-g(x)}{x}}
    \]
\end{theorem}

\begin{theorem}[Pappu's theorem]
    If $R$ is a reigon with area $A$ lying on one side of the line $l$, $V=Ad$ is the volume abtained by rotation $R$ about $l$, where $d$ is the distance travelled by the \textbf{com} when $R$ is rotated about $l$.
\end{theorem}

\subsection{Moments of inertia}

\begin{theorem}
    Given a curve $y=f(x)$ in the interval $[a,b]$, this is representing a wire in a given shape, and have the density per unit lenght of the wire at a given $x$ be $\rho(x)$, the \textbf{moment of inertia} of the curve about the $x$ and $y$ axis respectively is given by \[
        I_x = \intd{a}{b}{\rho(x)f(x)^2\sqrt{1+f'(x)^2}}{x} \qquad I_y = \intd{a}{b}{\rho(x)x^2\sqrt{1+f'(x)^2}}{x}
    \]
\end{theorem}

\subsection{Polar coordinates}

\begin{definition}[Polar coordinates]
    A parameterisation of $x,y$ is $r,\theta$ with $x=r\cos(\theta)$ and $y=r\sin(\theta)$.
\end{definition}

\begin{theorem}[Polar arc length]
    The arc length of a curve, $r=f(\theta)$ in polar coordinates between angles $\alpha,\beta$ is given by \[
        L = \intd{\alpha}{\beta}{\sqrt{\br{\frac{\dif r}{\dif \theta}}^2 + r^2}}{\theta}
    \]
\end{theorem}

\begin{theorem}[Polar area]
    The area of a polar curve, $r=f(\theta)$ between angles $\alpha,\beta$ is given by \[
        A = \frac{1}{2}\intd{\alpha}{\beta}{f(\theta)^2}{\theta}
    \]
\end{theorem}

\section{Fourier series}
\subsection{Orthogonal and orthonormal function spaces}

\begin{definition}[Inner product of functions]
    If $f,g:[a,b]\rightarrow\RR$ are integrable on $[a,b]$, the their \textbf{inner product} is defined as \[
        \abr{f,g} := \intd{a}{b}{f(x)g(x)}{x}
    \] 
\end{definition}

\begin{definition}[Orthogonal and orthonormal system]
    If $\SSS=\{\phi_0,\phi_1,\ldots\}$ is a collection of integrable real functions on $[a,b]$, iff $\abr{\phi_n,\phi_m}=0$ for all $n\neq m$ then $\SSS$ is an \textbf{orthogonal system} on $[a,b]$. Furthermore, $\SSS$ is a \textbf{orthonormal system} on $[a,b]$ iff $||\phi_n||:=\abr{\phi_n,\phi_n}=1$ for all $n$.
\end{definition}

\begin{theorem}
    The system \[
        \SSS = \left\{ \frac{1}{\sqrt{2\pi}},\frac{\cos(x)}{\sqrt{2\pi}},\frac{\sin(x)}{\sqrt{2\pi}},\frac{\cos(2x)}{\sqrt{2\pi}},\frac{\sin(2x)}{\sqrt{2\pi}},\ldots\right\}
    \] is orthonormal on all closed intervals of length $2\pi$.
\end{theorem}

\subsection{Periodic functions}

\begin{definition}[Periodic function]
    A function $f:\RR\rightarrow\RR$ is \textbf{periodic} with period $T$ iff $f(x+T)=f(x)$ for all $x\in\RR$.
\end{definition}

\begin{definition}[Discontinuity]
    When periodically extending a function, if $\lim\limits_{x\rightarrow\xi+}f(x) \neq \lim\limits_{x\rightarrow\xi-}f(x)$, we set $\displaystyle f(\xi):=\frac{1}{2}\left[\lim_{x\rightarrow\xi+}f(x) + \lim_{x\rightarrow\xi-}f(x)\right]$
\end{definition}

\begin{theorem}[Integral over period]
    If $f(x)$ is a $T$ periodic function, for all $a,b\in\RR$ we have \[
        \intd{a+T}{b+T}{f(x)}{x} = \intd{a}{b}{f(x)}{x}
    \]
\end{theorem}

\subsection{Trigonometric polynomials}

\begin{definition}[Trigonometric polynomial]
    A \textbf{trigonometric polynomial} is a function in the form \[
        S_n(x) = \frac{1}{2}a_0 + \sum_{k=1}^n\br{a_k\cos(kx)+b_k\sin(kx)}
    \]
\end{definition}

\begin{theorem}
    Using euler's identity we can rewrite a trigonometric polynomial \[
        S_n(x) = \frac{1}{2}a_0 + \sum_{k=1}^n\br{a_k\cos(kx)+b_k\sin(kx)} \ \text{\textcolor{black}{as }} \sum_{k=-n}^n\br{\gamma_ke^{ikx}} \ \text{\textcolor{black}{where }}
        \gamma_k = \begin{dcases}
            \hfil\frac{1}{2}a_0\hfil & \text{\textcolor{black}{if }} k=0 \\
            \frac{1}{2}(a_k-ib_k) & \text{\textcolor{black}{if }} k\in[1,n] \\
            \hfil\gamma_k^*\hfil & \text{\textcolor{black}{otherwise}} 
        \end{dcases}
    \]
\end{theorem}

\subsection{Fourier series}
\begin{definition}[Fourier series]
    If $f(x)$ is $2L$ periodic then its \textbf{Fourier series} is given by \[
        f(x) := \frac{1}{2}a_0 + \sum_{n=1}^\infty\left[a_k\cos\br{\frac{n\pi x}{L}}+b_k\sin\br{\frac{n\pi x}{L}}\right] \ \text{\textcolor{black}{where }}\]\[ a_n := \frac{1}{\pi}\intd{-L}{L}{f(x)\cos\br{\frac{n\pi x}{L}}}{x}, \qquad b_n := \frac{1}{\pi}\intd{-L}{L}{f(x)\sin\br{\frac{n\pi x}{L}}}{x}
    \]
    
\end{definition}

\begin{lemma}[Riemann-Lebesgue]
    If the function $f(x)$ is integrable on $[a,b]$ then \[
        I_\lambda := \intd{a}{b}{g(x)\sin(\lambda x)}{x}\rightarrow 0 \ \text{\textcolor{black}{as }}\lambda\rightarrow\infty
    \]
\end{lemma}

\begin{theorem}[Paerseval's]
    If $f(x)$ is periodic on $2\pi$ and is represented by its Fourier series, \[
        \frac{1}{\pi}\intd{-\pi}{\pi}{f^2(x)}{x} = \frac{1}{2}a_0^2 + \sum_{n=0}^\infty(a_n^2+b_n^2)
    \]
\end{theorem}

\section{Laplace transform}

\subsection{Definition}

\begin{definition}[Laplace transform]
    The \textbf{Laplace transform} is a linear operator that when applied to a function $f(x)$ gives \[
        F(p) := \LLL[f(x)] := \intd{0}{\infty}{e^{-px}f(x)}{x}
    \]
\end{definition}

\begin{theorem}[Common Laplace tranformations]
    These are some common functions with their Laplace transforms and the conditions for which they converge:
    \renewcommand{\arraystretch}{2}
    \begin{center}
        
        \begin{tabular}{|c|c|c|}
            \hline
            $f(x) = 1$ & $\displaystyle F(p) = \frac{1}{p}$ & Converges for $p>0$ \\
            $f(x) = x$ & $\displaystyle F(p) = \frac{1}{p^2}$ & Converges for $p>0$ \\
            $f(x) = x^n$ & $\displaystyle F(p) = \frac{n!}{p^{n+1}}$ & Converges for $p>0$ \\
            $f(x) = e^{ax}$ & $\displaystyle F(p) = \frac{1}{p-a}$ & Converges for $p>a$ \\
            $f(x) = \sin(ax)$ & $\displaystyle F(p) = \frac{a}{p^2+a^2}$ & Converges for $p>0$ \\
            $f(x) = \cos(ax)$ & $\displaystyle F(p) = \frac{p}{p^2+a^2}$ & Converges for $p>0$ \\
            $f(x) = \sinh(ax)$ & $\displaystyle F(p) = \frac{a}{p^2-a^2}$ & Converges for $p>a$ \\
            \rule[-12pt]{0pt}{0pt} $f(x) = \cosh(ax)$ & $\displaystyle F(p) = \frac{p}{p^2-a^2}$ & Converges for $p>a$ \\ \hline
        \end{tabular}
    \end{center}
    
\end{theorem}

\begin{theorem}[Existence of Laplace transform]
    The Laplace transform for a function $f(x)$ exists iff there exists constants $M,c\in\RR$ with $|f(x)|\leq Me^{cx}$ ($f(x)$ is of \textbf{exponential order}).
\end{theorem}

\begingroup\belowdisplayskip=-20pt
\subsection{Differentiating}

\begin{theorem}[Derivatives of Laplace transforms]
    By performing DUTIS $n\in\NN$ times we have \[
        F^{(n)}(p) = \LLL[(-1)^nx^nf(x)]
    \]
\end{theorem}

\subsection{Convolution theorem}

\begin{theorem}[Convolution theorem for Laplace transforms]
    For integrable functions $f,g:\RR\rightarrow\RR$:
    \[
        \LLL\left[\intd{0}{x}{f(x-t)g(t)}{t}\right]=F(p)G(p)
    \]
\end{theorem}
\endgroup

\end{document}