\documentclass[../Year1.tex]{subfiles}
\usepackage{import}
\usepackage{../../style/header}
\def\module{MATH40001 Introduction to University Mathematics}
\def\lecturer{Dr Marie-Amelie Lawn}
\def\term{Autumn 2023}
\def\cover{\vspace{1in}
}
\def\syllabus{This module provides a transition towards the way you will be thinking about, and doing, Mathematics during your degree. It will stress the importance of precise definitions and rigorous proofs, but also discuss their relationship to more informal styles of reasoning which are often encountered in applications of Mathematics. Topics to be covered will include an introduction to abstract sets, functions and relations, common proof strategies, the naturals, rationals and reals, and elementary vector operations and geometry.}
\def\thm{section}


\begin{document}

% Title page
\maketitle
\cover
\vfill
\begin{abstract}
\noindent\syllabus
\end{abstract}

\pagebreak

% Contents page
\tableofcontents

\pagebreak

% Document page
\setcounter{section}{-1}

\section{Introduction}

\lecture{1}{Thursday}{10/01/19}

The following are references.
\begin{itemize}
\item E Artin, Galois theory, 1994
\item A Grothendieck and M Raynaud, Rev\^etements \'etales et groupe fondamental, 2002
\item I N Herstein, Topics in algebra, 1975
\item M Reid, Galois theory, 2014
\end{itemize}

\begin{notation*}
If $ K $ is a field, or a ring, I denote
the \textbf{ring of polynomials} with coefficients in $ K $.
\end{notation*}

\end{document}