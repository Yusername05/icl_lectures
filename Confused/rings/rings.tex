\documentclass{report}
\usepackage{../../style/header}

\begin{document}

\renewcommand*\thesection{\arabic{section}}
\lhead{Yu's confused ramblings on algebra}

Roughly lecture notes from Kevin Buzzard's Algebra 3

\section{Rings}
\setcounter{subsection}{1}

\begin{definition}[Ring]
    A ring (with $1$) is a set $R$ with elements $0,1$ and binary operations $+,\times$ such that \begin{enumerate}
        \item $(R,+)$ is an abelian group with identity $0$,
        \item $(R,\times)$ is a semigroup with $1$ as the identity,
        \item both left and right multiplication are distributive over addition.
    \end{enumerate}
\end{definition}

\begin{examples}
    The following are common examples of rings:
    \begin{enumerate}
        \item  $\ZZ,\QQ,\RR,\CC$ are all rings with their normal operations,
        \item $n\ZZ$ and $\ZZ/n\ZZ$ with $n$ a positive integer, are both rings,
        \item $\RR[x]$, informally the set of polynomials with real coefficients, and $\QQ[x]$ are rings,
        \item given some set $X$ and a ring $R$ the set of functions $f:X\rightarrow R$ is a ring
        \item given a ring $R$, $M_{n\times n}(R)$ is a ring.
    \end{enumerate}
\end{examples}

\begin{definition}[Commutative ring]
    A ring, $R$, is \textbf{commutative} iff $a\times b=b\times a$ for all $a,b\in R$.
\end{definition}

\begin{definition}[Subring]
    A subset of a ring which is itself a ring under the same operators is a \textbf{subring}.
\end{definition}

\begin{lemma}\label{subring test}
    If $R$ is a ring and $0_R,1_R,s,t\in S\subseteq R$ with $s+t,st,s-t\in S$ then $S$ is a subring of $R$.
    \begin{proof}
        The only non-obvious axiom is that $S$ is closed under additive inverses which is given by $s-t\in S$.
    \end{proof}
\end{lemma}

\begin{example}
    $\ZZ[\sqrt{n}]$ with $n\in\ZZ$ is a ring; and note $a+b\sqrt{n}=c+d\sqrt{n}$ iff $a=c$ and $b=d$. \begin{proof}
        $\ZZ[\sqrt{d}]\subseteq\CC$ so by Lemma~\ref{subring test} take $r=a+b\sqrt{n}$ and $s=c+d\sqrt{n}$, and by simple manipulations have $r\pm s, rs\in \ZZ[\sqrt{n}]$. Arguing by contradiction that $\sqrt{d}$ is not rational gives the uniqueness.
    \end{proof}
\end{example}

\begin{corollary}
    The same argument extends to show $\QQ[\sqrt{n}]$ is a ring and in fact a field.\begin{proof}
        Commutativity is inherited from $\QQ$, if $r=a+b\sqrt{n}\neq 0$, $\displaystyle r^{-1}=\frac{a-b\sqrt{n}}{a^2-b^2d}$ with $a^2-b^2d\neq0$ easily coming from the irrationality of $\sqrt{d}$.
    \end{proof}
\end{corollary}

\begin{propositions}
    Have $R$ a ring with $r,s,r_i,s_j\in R$ for $i\in[1,n]$ and $j\in[1,m]$ respectively:\begin{enumerate}
        \item $r0=0r=0$,
        \item $(-r)s=r(-s)=-(rs)$ and $(-r)(-s)=rs$,
        \item $\displaystyle \br{\sum_{i=1}^nr_i}\sum_{j=1}^ms_j=\sum_{i=1}^n\sum_{j=1}^mr_is_j$,
        \item if $rs=s$ then $r=1$,
        \item if $0=1$ in $R$, $R=\{0\}$.
    \end{enumerate}
    \begin{proof}
        \begin{enumerate}
            \item $0+0=0\implies r(0+0)=r0\implies r0+r0=r0\implies r0=0$ and similarly for $0r$,
            \item $r-r=0\implies (r-r)s=0s=0 \implies (-r)s + rs = 0 \implies (-r)s=-(rs)$ with $r(-s)$ and $(-r)(-s) = rs$ immediately following,
            \item inducting on $m+n$ and distributivity,
            \item $s=1$ is sufficient,
            \item for any $r$, $r=r1=r0=0$, note $\{0\}$ is still a ring.\vspace{-12pt}
        \end{enumerate}
    \end{proof}
\end{propositions}

\begin{definition}[Invertible]
    An element $x$ of a ring $R$ is \text{invertible} if there exists $y,z\in R$ with $yx=xz=1$.
\end{definition}

\begin{definition}[Division ring]
    A ring $R$ is called a \textbf{division ring} if every element of $R$ is invertible.
\end{definition}

\begin{remark}
    A commutative division ring is a field.
\end{remark}

\begin{definition}[Zero divisor]
    An element $a$ of a ring $R$ is a \textbf{zero division} if there exists some $b\neq0\in R$ with $ab=0$.
\end{definition}

\begin{definition}[Integral domain]
    A ring with $1$, $R$, is an \textbf{integral domain} iff $R$ is commutative, has no zero divisors, and $0\neq 1$.
\end{definition}

\begin{examples}
    \begin{enumerate}
        \item $\ZZ$ is an integral domain,
        \item all fields are integral domains,
        \item $\{0\}$ is not an integral domain,
        \item a subring of an integral domain is also an integral domain,
        \item $\ZZ/n\ZZ$ is an integral domain iff $n$ is prime.
    \end{enumerate}
    \begin{proof}[Proof of 5] $(\implies)$ If $n=1$ we have $\ZZ/1\ZZ=\{0\}$, otherwise take $n=ab$ then $ab=0$ in $\ZZ/n\ZZ$ so neither are an integral domain. $(\impliedby)$ By lifting $a$ and $b$ to the integers we have $a',b'<n$ prime so $a,b\nmid n$ hence $ab\nmid n$ therefore $ab\neq 0$ in $\ZZ/n\ZZ$ which is therefore an integral domain.
    \end{proof}
\end{examples}

\begin{proposition}
    There is non-zero cancellation in integral domains. \begin{proof}
        Have $ar=as\implies a(r-s)=0 $ as $a$ is non-zero$ \implies r-s=0\implies r=s$
    \end{proof}
\end{proposition}

\begin{lemma}
    A finite integral domain is a field. \begin{proof}
        Consider $a\in R$ an integral domain with $\phi_a:R\rightarrow R$ by $\phi_a(r)=ar$. By cancellation $\phi_a$ is injective and as $R$ is finite $\phi_a$ is therefore also surjective and hence bijective. Take $a^{-1}=\phi_a^{-1}(1)$.
    \end{proof}
\end{lemma}

\begin{corollary}
    $\ZZ/n\ZZ$ is a field iff $n$ is prime.
\end{corollary}

\begin{theorem}[Wedderburn]
    A finite division ring is a field. \textit{Proof hard so left until later...}
\end{theorem}

\subsection{Ring homomorphisms}

\begin{definition}[Ring homomorphism]
    Let $R,S$ be rings, a function $f:R\rightarrow S$ is a \textbf{ring homomorphism} iff it satisfies \begin{enumerate}
        \item $f:(R,+)\rightarrow (S,+)$ is a group homomorphism,
        \item $f(xy)=f(x)f(y)$ for all $x,y\in R$,
        \item $f(1_R)=1_S$.
    \end{enumerate}
    A ring homomorphism $\phi:R\rightarrow S$ is an \textbf{isomorphism} if there exists some $\psi:R\rightarrow S$ with $\phi\circ\psi$ and $\psi\circ\phi$ both identity maps.
\end{definition}

\begin{examples}
    The following are some common examples of ring homomorphisms: \begin{enumerate}
        \item $\phi:\ZZ\rightarrow\ZZ/n\ZZ$ with $\phi(t)= t\mod{n}$, and $[0], [1]$ the identities,
        \item $f:\CC\rightarrow\CC$ with $f(z)=\overline{z}$ is in fact a self-inverse ring isomorphism,
        \item $\phi_\lambda:\RR[x]\rightarrow\RR$ which evaluates a polynomial at $\lambda\in\RR$,
        \item structure preserving inclusions are also ring homomorphisms.
    \end{enumerate}
\end{examples}

\begin{definition}[Ideal]
    For a ring $R$, a subset $I\subseteq R$ is a \textbf{left ideal}, denoted $I\unlhd R$ iff \begin{enumerate}
        \item $(I,+)$ is a subgroups of $(R,+)$,
        \item if $r\in R$ and $i\in I$, $ri\in R$.
    \end{enumerate} Similarly, for \textbf{right ideals}. A subset $I$ is a bi-ideal if it is both a left and right ideal.
\end{definition}

\begin{examples}
    \begin{enumerate}
        \item For any ring $R$, $\{0\}$ and $R$ are always ideals,
        \item $x\RR[x]$ is an ideal for $\RR[x]$,
        \item $m\ZZ$ is an ideal for $\ZZ$ and in fact all ideals are of this form.
    \end{enumerate}
\end{examples}

\begin{definition}[Quotient ring]
    Given a ring $R$ with $I\lhd R$, have \textbf{cosets} of $R$ by $I$ be the subsets of $R$ in the form $r+I:=\{r+i:i\in I\}$, the set of these cosets, $R/I$ is the \textbf{quotient ring} of $R$ by $I$ under the operations $(r+I)+(s+I)=(r+s+I)$ and $(r+I)(s+I)=(rs+I)$.\begin{proof}
        The structure of $R$ translates directly to $R/I$ so it is clearly a ring.
    \end{proof}
\end{definition}

\begin{lemma}\label{iso1}
    The kernel of a ring homomorphism $\phi:R\rightarrow S$ is an ideal. \begin{proof}
        The kernel of $\phi$ is a subgroup of $R$ and $\phi(ir)=\phi(i)\phi(r)=0\phi(r)=0$ so $ir\in\ker\phi$ and similarly for $ir$ hence $\ker\phi$ is a bi-ideal.
    \end{proof}
\end{lemma}

\begin{lemma}\label{iso2}
    The image of a ring homomorphism $\phi:R\rightarrow S$ is a subring. \begin{proof}
        By Lemma~\ref{subring test} and ring homomorphism axioms.
    \end{proof}
\end{lemma}

\begin{theorem}\label{iso}
    Have $\phi:R\rightarrow S$ a ring homomorphism, $\im\phi$ is naturally isomorphic to $R/\ker\phi$.
    \begin{proof}
        Have $\psi:R/\ker\phi\rightarrow\im\phi$ by $\psi(r+I)=\phi(r)$. \begin{enumerate}
            \item[](well defined) $r+\ker\phi=r'+\ker\phi\implies r-r'\in\ker\phi\implies\phi(r)=\phi(r')$,
            \item[](injective) the same argument but backwards,
            \item[](subjective) any $\phi(r)$ for $r\in R$ is $\psi(r+I)$.\vspace{-20pt}
        \end{enumerate}
    \end{proof}
\end{theorem}

\begin{proposition}
    A commutative ring $R$ is a field iff its only ideals are $\{0\}$ and $R$.
    \begin{proof}
        
    \end{proof}
\end{proposition}

\begin{proposition}
    Given $f:R\rightarrow S$ a ring homomorphism with $J$ a left (or right or bi) ideal of $S$, $f^{-1}(J)$ is a left (respectively) ideal of $R$.
    \begin{proof}
        
    \end{proof}
\end{proposition}

\begin{definition}[Prime ideal]
    Let $R$ be a commutative ring, a proper ideal $I\subset R$ is a \textbf{prime ideal} iff $ab\in I$ for $a,b\in R \implies a\in I$ or $b\in I$.
\end{definition}

\begin{proposition}
    If $I\subset R$ is a prime ideal, $R/I$ is an integral domain.
    \begin{proof}
        
    \end{proof}
\end{proposition}

\begin{definition}[Maximal ideal]
    A proper ideal $I$ in a commutative ring $R$ is \textbf{maximal} iff there are no other proper ideals $J$ with $I\subset J$.
\end{definition}

\begin{proposition}
    $I$ is a maximal ideal of $R$ iff $R/I$ is a field.
    \begin{proof}
        
    \end{proof}
\end{proposition}

\begin{corollary}
    Maximal ideals are prime in commutative rings. \begin{proof}
        
    \end{proof}
\end{corollary}

\begin{corollary}
    $\{0\}$ is a prime ideal of a commutative ring $R$ iff $R$ is an integral domain. $\{0\}$ is a maximal ideal of $R$ iff $R$ is a field. \begin{proof}
        
    \end{proof}
\end{corollary}

\begin{corollary}
    The maximal ideal of $\ZZ$ are $p\ZZ$ for prime $p$, the remaining non-maximal prime ideal is $\{0\}$. \begin{proof}
        
    \end{proof}
\end{corollary}

\subsection{Generators of ideals}

\begin{definition}[Generated ideal]
    Have $R$ a commutative ring (this is not necessary just much simpler) with finite $X=\{x_1,\ldots,x_n\}\subseteq R$, the \textbf{ideal generated by $X$} is the set $I:=\{r_1x_2+\cdots+r_nx_n:r_i\in R\}$. Write $I=(x_1,\ldots,x_n)$.
\end{definition}

\begin{lemma}
    Under the same definitions as above, $I$ is the smallest ideal of $R$ containing $X$.\begin{proof}
        
    \end{proof}
\end{lemma}

\begin{remark}
    If $X$ is infinite then $I$ is still the collection of finite \textit{linear combinations} of $X$ in $R$.
\end{remark}

\begin{definition}
    An ideal $I$ of the commutative ring $R$ is \textbf{finitely generated} if $I=(x_1,\ldots,x_n)$ for some $x_i\in R$. Furthermore, $I$ is \textbf{principal} if $I=(x)$ for some $x\in R$.
\end{definition}

\begin{definition}
    A commutative ring $R$ is \textbf{Noetherian} if all ideals of $R$ are finitely generated; and call $R$ a \textbf{principal ideal domain} if all ideals of $R$ are principal.
\end{definition}

\begin{remark}
    Consider $R$ being Noetherian roughly as $R$ being ``finite dimensional" and $R$ being a PID as being ``leq 1 dimensional".
\end{remark}

\subsection{Factorisation}
Throughout this section we will always have $R$ be an integral domain.

\begin{definition}[Unit]
    $r\in R$ is a \textbf{unit} if there exists some $y\in R$ with $x\times y=1_R$. We write  $R^\times$ for the group of units in $R$ under multiplication.
\end{definition}

\begin{examples}
    Some common examples of units in rings: \begin{enumerate}
        \item $\ZZ^\times = \{1,-1\}$,
        \item $\ZZ[i]^\times=\{1,-1,i,-i\}$,
        \item given $d<-1$, $\ZZ[\sqrt{d}]^\times=\{1,-1\}$,
        \item given a field $F$, $F[x]^\times=F^\times$.
    \end{enumerate}
\end{examples}

\begin{definition}[Associates]
    If $x,y\in R$, $x$ and $y$ are \textbf{associates} iff $x=uy$ for some unit $u$.
\end{definition}

\begin{proposition}
    Association is an equivalence relation on any integral domain, $x\sim y$ iff $(x)=(y)$.
    \begin{proof}
        
    \end{proof}
\end{proposition}

\begin{proposition}
    If $x\in R$ is a unite, $(x)=R$. \begin{proof}
        
    \end{proof}
\end{proposition}

\begin{definition}[Irreducible]
    $r\in R\setminus R^\times$ is \textbf{irreducible} if it cannot be written as the product of two elements of $R\setminus R^\times$.
\end{definition}

\begin{examples}
    Common examples of irreducible elements in rings: \begin{enumerate}
        \item the irreducible elements of $\ZZ$ is all $\pm p$ for a prime $p$,
        \item the irreducible elements of $\ZZ[i]$ are the set of associates of primes congruent to $3$ modulo $4$,
        \item in $\RR[x]$ the polynomial $x^2+1$ is irreducible, however in $\CC[x]$, $x^2+1=(x+i)(x-i)$; and in fact by the fundamental theorem of algebra, irreducible elements in $\CC[x]$ are all order $1$ polynomials.
    \end{enumerate}
    \begin{proof}[Proof of 2.]
        
    \end{proof}
\end{examples}

\begin{definition}
    A reminder of a very familiar definition, a number $p$ is \textbf{prime} iff $p|ab\implies p|a$ or $p|b$.
\end{definition}

\begin{lemma}
    All primes of $R$ are irreducible.\begin{proof}
        
    \end{proof}
\end{lemma}

\begin{proposition}
    If $0\neq r\in R$, then $r$ is prime iff $(r)$ is a prime ideal.\begin{proof}
        
    \end{proof}
\end{proposition}

\begin{definition}[Euclidean domain]
    $R$ is a Euclidean domain if there exists some  $\phi:R^*\rightarrow\ZZ_{\geq0}$ satisfying: \begin{enumerate}
        \item $\phi(ab)\geq\phi(a)$ for all $a,b\neq 0$,
        \item if $a,b\in R$ with $b\neq 0$ there exists $q,r\in R$ with $a=qb+r$ with either $r=0$ or $\phi(r)\leq\phi(b)$.
    \end{enumerate}
\end{definition}

\begin{examples}
    \begin{enumerate}
        \item $\ZZ$ is a euclidean domain with $\phi(n)=|n|$,
        \item given some field $k$, $k[x]$ is a euclidean domain with $\phi(p)=\deg(p)$
    \end{enumerate}
\end{examples}

\begin{theorem}
    $R$ is a Euclidean domain $\implies$ $R$ is a PID.\begin{proof}
        
    \end{proof}
\end{theorem}

\begin{corollary}
    If $k$ is a field, $k[x]$ is a PID. \textit{Proof.} Obvious \qed
\end{corollary}

\begin{lemma}
    All irreducibles in a PID are prime.
\end{lemma}

\begin{definition}[Unique factorisation domain]
    $R$ is a \textbf{unique factorisation domain} if: \begin{enumerate}
        \item[(U1)] if $r\neq0\in R$, $r=ur_1\ldots r_n$ for some unit $u$ and $r_i\in R$ with $n\geq0$,
        \item[(U2)] if $r=ur_1\ldots r_n=vs_1\ldots s_m$ for units $u,v$ and $r_i,s_i\in R$ with $m,n\geq 0$, $m=n$ and (after reordering) $r_i$ and $s_i$ are associates for all $i$.
    \end{enumerate}
\end{definition}

\begin{examples}
    $\ZZ$, $k[x]$, $\ZZ[i]$ and any ED are all UFDs.
\end{examples}

\begin{theorem}
    All PIDs are UFDs.\begin{proof}
        
    \end{proof}
\end{theorem}

\subsection{Localisation}

\begin{definition}[Multiplicative subset]
    Given $S\subseteq R$ a commutative ring, $S$ is a \textbf{multiplicative subset} of $R$ if $1\in S$ and $s,t\in S\implies st\in S$.
\end{definition}

\begin{definition}[Localisation]
    Have $R$ an integral domain with $S$ a multiplicative subset without $0$, the \textbf{localisation} of $R$ at $S$ is the set of equivalence classes of $(R\times S)/\sim$ with $(r_1,s_1)\sim(r_2,s_2)\iff r_1s_2=r_2s_1$. Denoted $R_S$ or $S^{-1}R$
\end{definition}

\begin{theorem}
    Have $R$ an integral domain, the localisation of $R$ by $R\setminus\{0\}$ called the \textbf{field of fractions} is a unique field up to isomorphism and is denoted $\Frac(R)$ \begin{proof}
        
    \end{proof}
\end{theorem}

\begin{corollary}
    Any integral domain is a subring of a field.
\end{corollary}

\begin{lemma}
    Have $R$ a commutative ring with $S$ a multiplicative subset of $R$ and $\phi:R\rightarrow A$ a ring homomorphism such that $\phi(s)$ is a unit in $A$ for all $s\in S$. There exists some $\tilde{\phi}:R_S\rightarrow A$ extending $\phi$. \begin{proof}
        
    \end{proof}
\end{lemma}

\begin{examples}
    \begin{enumerate}
        \item if $S=\{1\}$, $S^{-1}R=R$,
        \item if $p\in\ZZ$ is a prime and $S=\{1,p,p^2,\ldots\}$, $S^{-1}\ZZ=\ZZ[\frac{1}{p}]$ which is a subring of $\QQ$,
        \item have $P=(p)$ for a prime $p\in\ZZ$ and $S=\ZZ/P$ then $S^{-1}\ZZ=\ZZ_{(p)}$
    \end{enumerate}
\end{examples}

\begin{proposition}
    $\QQ$ has uncountably many subrings.
\end{proposition}

\subsection{Zorn's lemma}

\begin{definition}[Partial order]
    A \textbf{partial order} on a set is a binary relation, $\leq$ satisfying: \begin{enumerate}
        \item[(P1)] for all $s\in S$, $s\leq s$,
        \item[(P2)] if $s\leq t$ and $t\leq s$ then $s=t$,
        \item[(P3)] if $s\leq t$ and $t\leq u$ then $s\leq u$.
    \end{enumerate}
\end{definition}

\begin{definition}[Chain]
    A \textbf{chain} in a poset $S$ is a subset $T\subseteq S$ where $\leq$ is total.
\end{definition}

\begin{definition}[Upper bound, maximal element]
    An \textbf{upper bound} for $W\subseteq S$ a poset is some $s\in S$ such that for all $w\in W$, $w\leq s$. A maximal element of $S$ is an element $x\in S$ such that $x\leq y\implies x=y$ for all $y\in S$.
\end{definition}

\begin{example}
    Have $s_1\subseteq s_2\subseteq \cdots$ subsets of $X$, then $s=\bigcup_{n\geq1}s_n$ and $X$ are both upper bounds.
\end{example}

\begin{axiom}[Zorn's Lemma]
    If $S$ is a poset such that every chain in $S$ has an upper bound, $S$ has a maximal element (and possibly multiple).
\end{axiom}

\begin{remark}
    Zorn's lemma is equivalent to the axiom of choice.
\end{remark}

\begin{theorem}
    If $R$ is a ring with $I\lhd R$, there exists some maximal ideal $m$ with $I\subseteq m\subset R$. \begin{proof}
        
    \end{proof}
\end{theorem}

\begin{corollary}
    If $R\neq\{0\}$ then it has a maximal ideal.
\end{corollary}

\begin{proposition}
    An ideal $I$ of a commutative ring $R$ is the unique maximal ideal iff $R$ is the disjoint union of $I$ and $R^\times$.
\end{proposition}

\begin{definition}[Local]
    A commutative ring is \textbf{local} if it has a unique maximal ideal.
\end{definition}

\begin{proposition}
    Have $R$ an integral domain with $P\subset R$ a prime ideal and $S=R/P$, $S^{-1}R$ is a local ring with unique maximal ideal $S^{-1}P$. \begin{proof}
        
    \end{proof}
\end{proposition}

\subsection{Polynomial rings}

\begin{definition}[Polynomial ring]
    Let $R$ be a commutative ring, formally, the \textbf{polynomial ring} $R[x]$ is the set of infinite sequences with terms in $R$ and finitely many nonzero terms. Informally it is convenient to view this as the set of polynomials with coefficients in $R$. $R[x,y]:=(R[x])[y]$.
\end{definition}

\begin{proposition}
    If $R$ is an integral domain, so is $R[x]$.\begin{proof}
        
    \end{proof}
\end{proposition}

\begin{corollary}
    If $R$ is an integral domain, so is $R[x_1,\ldots,x_n]$.
\end{corollary}

\begin{theorem}[Hilbert basis theorem]
    If $R$ is Noetherian, so is $R[x]$. \begin{proof}
        
    \end{proof}
\end{theorem}

\begin{corollary}
    If $R$ is a field or PID, $R[x_1,\ldots,x_n]$ is Noetherian. \begin{proof}
        $R$ is clearly Noetherian so by induction on $n$.
    \end{proof}
\end{corollary}

\begin{lemma}
    Have $\phi:A\rightarrow B$ a surjective ring homomorphism, if $A$ is Noetherian then $B$ is Noetherian.\begin{proof}
        
    \end{proof}
\end{lemma}

\begin{corollary}
    If $R$ is a PID with $I\unlhd R[x_1,\ldots,x_n]$ then $R[x_1,\ldots,x_n]/I$ is Noetherian. \begin{proof}
        
    \end{proof}
\end{corollary}

\subsection{Factorisation in polynomial rings}

\begin{definition}[$\pi$-adic valuation]
    For some element $\pi\in X\subseteq R$ a UFD, the \textbf{$\pi$-adic valuation} on $F=\Frac{R}$ is the map $val_\pi:F^*\rightarrow\ZZ$ such that given some $f\neq0\in F$ written $\displaystyle f=u\pi^e\prod_{x_i\neq\pi\in X}x_i^{p_i}$, 

    \vspace{-10pt}
    have $val_\pi(f):=e$.
\end{definition}

\begin{proposition}
    $val_\pi$ is a group homomorphism $(F^*,\times)\rightarrow(\ZZ,+)$.
\end{proposition}

\begin{remark}
    It can be useful to consider $val_\pi(0):=+\infty$ to have $val_\pi:F\rightarrow\ZZ\cup\{+\infty\}$.
\end{remark}

\begin{definition}
    Given $0\neq q\in F[x]$ with $q=f_0+f_1x+\ldots+f_nx^n$ and $\pi\in X$, have $\displaystyle val_\pi(q):=\min_{i:f_i\neq0}\{val_\pi(f_i)\}$.
\end{definition}

\begin{lemma}
    \begin{enumerate}
        \item $val_\pi(xy)=val_\pi(x)val_\pi(y)$,
        \item if $val_\pi(x)$ and $val_\pi(y)>0$, $val_\pi(x+y)=0$.
    \end{enumerate}\begin{proof}
        
    \end{proof}
\end{lemma}

\begin{definition}[Content]
    The \textbf{content} of $f\neq0\in F[x]$ is the finite product \[
        cont(f):=\prod_{\pi\in X}\pi^{val_\pi(f)}\black{,}
    \] and say $f$ is \textbf{primitive} when $cont(f)=1$.
\end{definition}

\begin{lemma}
    $cont(f)\in R \iff f\in R[x]$, furthermore $f$ is primitive iff $f\in R[x]$ with the coefficients of $f$ having gcd $1$. \begin{proof}
        
    \end{proof}
\end{lemma}

\begin{lemma}
    If $f\neq 0\in F[x]$ and $\lambda\in F^\times$ then $cont(\lambda f)=u\lambda cont(f)$ for some unit $u$. \begin{proof}
        
    \end{proof}
\end{lemma}

\begin{lemma}
    Have $R$ a UFD with $F=\Frac(R)$, if $f,g\neq0\in F[x]$ then $cont(fg)=cont(f)cont(g)$.\begin{proof}
        
    \end{proof}
\end{lemma}

\begin{corollary}[Gauss' Lemma]
    Given a UFD $R$ with $f\neq0\in F[x]$ and $\deg(f)\geq 1$. If $f$ is irreducible in $R[x]$ then $f$ is irreducible in $F[x]$.\begin{proof}
        
    \end{proof}
\end{corollary}

\begin{theorem}
    If $R$ is a UFD, so is $R[x]$. Furthermore, the irreducible elements of $R[x]$ are the elements of $R$ and primitive polynomials which are also units in $F[x]$. \begin{proof}
        
    \end{proof}
\end{theorem}

\begin{corollary}
    If $R$ is a UFD, $R[x_1,\ldots,x_n]$ is a UFD. \textit{Proof.} Trivial. \qed
\end{corollary}

\section{Modules}
\setcounter{subsection}{1}

\subsection{Quotient modules}

\subsection{Snake lemma}

\subsection{Injective and projective modules}

\subsection{Hom}

\subsection{Tensor product}

\subsection{Semisimple modules}

\section{Fields}

\end{document}