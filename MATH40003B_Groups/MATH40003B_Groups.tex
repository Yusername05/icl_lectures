\documentclass[../Year1/Year1.tex]{subfiles}
\usepackage{import}
\usepackage{../style/header}

\begin{document}

\chapter{Groups}
\renewcommand*\thesection{\arabic{section}}
\lhead{MATH40003B Groups}
Lectured by Dr Michele Zordan \\ Typed by Yu Coughlin \\
Spring 2024

\section*{Introduction}

\lecture{1}{Thursday}{10/01/19}

The following are supplementary reading:
\begin{itemize}
    \item J B Fraleigh, A first course in abtract algebra, 2014
    \item R B J T Allenby, Rings, field and groups: an introduction to abstract algebra, 1991
    \item A W Knapp, Basic Algebra, 2006
\end{itemize}

\tableofcontents\pagebreak

\section{Binary operations and groups}

\begin{definition}[Binary operation]
    Given a set $G$ a \textbf{binary operation} on $G$ is a mapping $\cdot: G\times G \rightarrow G$ written $\cdot(g,h) = g\cdot h$ (and sometimes $gh$) for all $g,h\in G$.
\end{definition}

\begin{definition}[Group]
    A \textbf{group} is a pair $G=(G,\cdot)$, for some set $G$ and a binary operation $\cdot$, satisfying the following properties: \begin{enumerate}
        \item[(G1)] $(a\cdot b)\cdot c = a \cdot (b\cdot c)$ for all $a,b,c\in G$ (the binary operation is \textbf{associative}),
        \item[(G2)] $\exists e\in G$ such that $\forall g\in G g\cdot e = e\cdot g = g $ (there is an \textbf{identity} element),
        \item[(G3)] $\forall g\in G, \exists g^{-1} \in G$ such that $g\cdot g^{-1} = g^{-1}\cdot g = e$ (every element has an \textbf{inverse}).
    \end{enumerate}
    In some literature, the condition of \textbf{closure} is also required however this is given in the fact that $\cdot$ is a binary operation on $G$.
\end{definition}

\begin{theorem}[Uniqueness of identity]
    The identity element for some group $G$ is unique. The inverse, $g^{-1}$, of any element $g\in G$ is also unique.
\end{theorem}

\begin{proof}
    Given identities $e_1,e_2\in\G$, $e_1 = e_1\cdot e_2 = e_2$.
\end{proof}

\begingroup\belowdisplayskip=-10pt
\begin{lemma}[Inverse of product]
    Given a group $G$ and the elements $g_1,g_2,\ldots,g_n\in G$ we have, \[
        {(g_1g_2\dots g_n)}^{-1} = g_n^{-1}g_{n-1}^{-1}\dots g_1^{-1}
        \textcolor{black}{.}
    \]
\end{lemma}
\endgroup

\begin{proof}
    $(g_1g_2\ldots g_n)(g_n^{-1}\ldots g_2^{-1}g_1^{-1}) = e$ clearly, so ${(g_1g_2\dots g_n)}^{-1} = g_n^{-1}g_{n-1}^{-1}\dots g_1^{-1}$.
\end{proof}

\begin{lemma}[Uniquess of inverses]
    The inverse of an element $g\in G$ is unique.
\end{lemma}

\begin{proof}
    Suppose $a,b$ are inversers of $g\in G$, $ag=e=bg\implies a=b$.
\end{proof}

\begin{definition}[Abelian Group]
    If a group $G$ also satisfies the condition $g\cdot h = h\cdot g$ for all $g,h\in G$ (\textbf{commutativity}), then $G$ is an \textbf{abelian group}.
\end{definition}

\begin{definition}[Powers of elements]
    Given a group $G$ and some $g\in G$ the $n$th \textbf{power} of $g$ in $G$ is defined recursively as, \[
        g^n:= \begin{dcases}
            \omit\hfil e  \hfil & \text{\textcolor{black}{if }} n=0 \\
            g^{n-1}g & \text{\textcolor{black}{if }} n>0 \\
            {(g^{n})}^{-1} & \text{\textcolor{black}{if }} n<0 \\
        \end{dcases}
    \textcolor{black}{.}
    \]
\end{definition}
\vspace{-20pt}

\begin{definition}[Order of group]
    The \textbf{order} of a group $G$, written $|G|$, is the cardinality of the set of $G$.
\end{definition}

\begin{example}[Symmetric group]
    The \textbf{symmetric group of size $n$}, denoted $S_n$, is the set of bijections on the interval $[1,n]$, for $n\in\NN$, under function composition. In generarl, given a set $X$, $\Sym(X)$ is the group of permutations of $X$.
\end{example}

\section{Subgroups}

\subsection{Subgroups}

\begin{definition}[Subgroup]
    Given a group $(G,\cdot)$ and a subset $H\subseteq G$ we say $(H,\cdot)$ is a \textbf{subgroup} of $G$, written $H\leq G$, if $(H,\cdot)$ is a group. H is a \textbf{proper subgroup} iff $H\neq G$.
\end{definition}

\begin{theorem}[Subgroup test]
    Given a group $(G,\cdot)$, $(H,\cdot)$ is a subgroup iff: \begin{enumerate}
        \item[(S1)] $H$ is non-empty (\textbf{existence}),
        \item[(S2)] for all $h_1,h_2\in H$ we have $h_1\cdot h_2\in H$ (\textbf{closure under group operation}),
        \item[(S3)] for all $h\in H$ we have $h^{-1}\in H$ (\textbf{closure under inverses}).
    \end{enumerate}
\end{theorem}
 
\begin{proof}
    ($\Leftarrow$) is simple. For ($\Rightarrow$): group axioms $\Rightarrow$ (S1) and (S2), as $H$ is a group, $h$ must have an inverse $h'\in H$, inverses are unique $\implies$ (S3).
\end{proof}

\subsection{Cyclic groups and orders}

\begin{definition}[Cyclic group]
    We say a group $G$ is \textbf{cyclic} if there is an element $g\in G$ such that \[
    G = \abr{g} := \{g^n:n\in\NN\} 
    \textcolor{black}{.}
    \]
    We say that $G$ is \textbf{generated} by $g$ or $g$ is a \textbf{generator} of $G$.
\end{definition}

\begin{definition}[Order of elements]
    Given a group $G$ and some $g\in G$, the \textbf{order} of $g$ in $G$, written $\text{ord }g$, is the smallest positive integer $n$ such that $g^n = e$ or $\infty$ if no such $n$ exists.
\end{definition}

\begin{theorem}\label{cyclic order}
    Suppose $G$ is a group with $g\in G$ having finite order $n$, $\abr{g} = \{e,g,g^2,\ldots,g^{n-1}\}$.
\end{theorem}

\begin{lemma}
    For $a,b\in\ZZ$, $g^a=g^b\iff a\equiv b \mod{n}$
\end{lemma}

\begin{proof}
    ($\Leftarrow$) is simple. For ($\Rightarrow$), $g^a=g^b \implies g^{a-b}=e$, by division algorithm $\implies e=g^{qn+r}={(g^n)}^q\cdot g^r = g^r$ so $r=0$ and $n|a-b$.
\end{proof}

\begin{proof}[Proof of~\ref{cyclic order}]
    All $m\in\ZZ$ are congruent to one of $0,1,\ldots,n-1 \mod{n}$ so $\abr{g} =\{g^m:m\in\ZZ\}=\{e,g,\ldots,g^{n-1}\}$.
\end{proof}

\begin{theorem}
    Suppose  $G$ is a cyclic group with $G = \abr{g}$, the three statements: \begin{enumerate}
        \item $H\leq G \implies H$ is cyclic,
        \item suppose $|G| = n$ and $m\in\ZZ$ with $d=\text{gcd}(m,n)$, \[
            \abr{g^m} = \abr{g^d} \ \text{\textcolor{black}{ and }} \aabr{g^m} = \frac{n}{d}
            \textcolor{black}{.}
        \] In particular, $\abr{g^m} = G$  iff $\text{gcd} (m,n)=1$,
        \item if $|G|=n$ and $k\leq n$, then $G$ has a subgroup of order $k$ iff $k|n$, this subgroup is $\abr{g^{n/k}}$.
    \end{enumerate}
\end{theorem}

\begin{proof}
    \begin{enumerate}
        \item  Have $H\neq\{e\}$, consider $d:=\min\{n\in\NN:g^n\in H\}$, clearly $\abr{g^d}\leq H$. For all $h=g^m\in H$, $g^m=g^{pd+r}={(g^d)}^p\cdot g^r\implies g^r=h{(g^d)}^{-p}\in H$ therefore $r=0$ so $h\in\abr{g^d}$ and $H=\abr{g^d}$.
        \item ($\subseteq$) $g^d=g^{km}\in\abr{g^m}$. ($\supseteq$) Have $d=am+bn$ (Bézout's identity), $g^d = g^{am+bn}=g^{am}g^{bn}={(g^m)}^a\in\abr{g^d}$.
        \item ($\implies$) 1. ($\impliedby$) 2.
    \end{enumerate}
\end{proof}

\begin{definition}[Euler totient]
    The \textbf{Euler totient} function $\phi$ is defined as $\phi(n):=|\{k\in\NN: k\leq n$ and $\text{gcd}(k,n)=1\}|$.
\end{definition}

\vspace{-5pt}

\begin{corollary}
    For $n\in\NN$: \hfil $\displaystyle\sum_{d|n}\phi(d) = n$.
\end{corollary}

\vspace{-15pt}

\begin{proof}
    Consider the cyclic group of order $n$, $G$. If $d|n$, $\abr{g^{n/k}}$ is the subgroup with all elements of order $d$ with $\phi(d)$ elements of order $d$. By summing this for $d|n$ (orders of elements in $G$) we count all of the $n$ elements of $G$ by their order.
\end{proof}

\subsection{Cosets}

\begin{definition}[Coset]
    Given a group $G$ with $H\leq G$ and $g\in G$ then \[
    gH := \{gh: h\in H\}
    \textcolor{black}{,}
    \]
    is a \textbf{left coset} of $H$ in $G$ (similarly for a \textbf{right cosets}). We will now assume all \textbf{cosets} to be left cosets.
\end{definition}

\begin{lemma}
    Given a group $G$ with $H\leq G$, all cosets of $H$ in $G$ have the same size.
\end{lemma}

\begin{proof}
    Lemma~\ref{bijection from element} $\implies |H|=|gH|$ for all $g\in G$.
\end{proof}

\begin{lemma}
    If $G$ is a finite group with $H\leq G$, the  cosets of $H$ form a partition of $G$.
\end{lemma}

\begin{proof}
    \begin{enumerate}
        \item If $g_1\in g_2H$ (by $h$), for some $g_1h'\in g_1H$, $g_1h'=g_2(hh')\in g_2H$, $g_2=g_1h^{-1}\in g_1H$.
        \item If $x\in g_1H\cap g_2H$ ($g_1H\cap g_2H\neq\emptyset$), apply 1. twice to get $g_1 H= xH = g_2 H$.
    \end{enumerate}
\end{proof}

\subsection{Lagrange's theorem}

\begin{theorem}[Lagrange's theorem]
    If $G$ is a finite group and $H\leq G$, $|H|$ divides $|G|$.
\end{theorem}

\begin{proof}
    Partition $G$ into the $n\in\NN$ distinct cosets of $H$ all with size $|H|$, $|G|=n|H|$. Have $n:=[G:H]$.
\end{proof}

\begin{corollary}
    Given a group $G$ with $H\leq G$, the relation $\sim$ on $G$ given by: $g\sim k$ iff $g^{-1}k\in H$, is an equivalence relation with equivalence classes given by cosets of $H$.
\end{corollary}

\begin{proof}
    $g\sim k\implies k\in gH$ equivalence relation from partition (IUM part 1) given by cosets of $G$ by $H$.
\end{proof}

\begin{corollary}\label{lagrange for elements}
    Given a group $G$ of order $n$, for all $g\in G$, $\ord{g}|n$ and $g^n=e$.
\end{corollary}

\begin{proof}
    Apply Lagrange's theorem with $H=\abr{g}$, $g^n = {(g^{\ord{g}})^{n/\ord{g}}}=e^{n/\ord{g}}=e$ (due to first part).
\end{proof}

\begin{corollary}[Fermat's little theorem]
    Let $p$ be prime. If $x\in\ZZ$ and $p\nmid x$, then $x^{p-1}\equiv 1 (\text{mod }p)$.
\end{corollary}

\begin{proof}
    Let $G={(\ZZ/p\ZZ)}^*$, $|G|=p-1$ and (by Corollary~\ref{lagrange for elements}) $[x^{p-1}]={[x]}^{p-1}=[1]$ for all $[x]\in G$.
\end{proof}

\begin{corollary}
    If a group $G$ is of prime order, $G$ is cyclic and $\abr{g}=G$ for all $(g\neq e)\in G$.
\end{corollary}

\begin{proof}
    By Lagrange's Theorem $\aabr{g}$ divides $p$, as $g\neq e$, $\aabr{g}=p\implies \abr{g}=G$.
\end{proof}

\subsection{Generating groups}

\begin{definition}
    Given a group $G$ with $S\subseteq G$, $S^{-1}:=\{g^{-1}\in G:g\in S\}$.
\end{definition}

\begin{definition}[Subgroup generated by a set]
    Let $G$ be a group with non-empty $S\subseteq G$. The \textbf{subgroup generated by $S$} is defined as \[
        \abr{S} := \{g_1g_2\ldots g_k\in G: k\in\NN \ \text{\textcolor{black}{and }} g_i\in S\cup S^{-1} \ \text{\textcolor{black}{for all }} i\in[1,k]\}
        \textcolor{black}{.}
    \]
\end{definition}

\vspace{-30pt}

\begin{lemma}
    Given a group $G$ with non-empty $S\subseteq G$, $\abr{S}\leq G$ and, $H\leq G, \ S\subseteq H \implies \abr{S} \leq H$. This is equivalent to saying ``$\abr{S}$ is the smallest subgroup of $G$ containing $S$''.
\end{lemma}

\begin{proof}
    
\end{proof}

\section{Group homomorphisms}

\begin{definition}[Group homomorphism]
    If $(G,\cdot)$ and $(H,\ast)$ are goups, $\phi:G\rightarrow H$ is a \textbf{group homomorphism} iff $\phi(g_1)\ast\phi(g_2) = \phi(g_1\cdot g_2)$ for all $g_1,g_2\in G$. If $\phi$ is bijective then it is called a \textbf{group isomorphism} with $G$ and $H$ being \textbf{isomorphic}, written $G\cong H$.
\end{definition}

\begin{example}[determinant]
    The \textbf{determinant} is a group homomorphism, suppose $\FF$ is a field:\[
    \det: \GL(n,\FF)\rightarrow(\FF^*,\times)
    \textcolor{black}{.}
    \]
\end{example}

\vspace{-30pt}

\begin{lemma}
    If $G$,$H$ are groups with $\phi:G\rightarrow H$, \begin{enumerate}
        \item $\phi(e_G) = e_H$,
        \item $\phi(g^{-1}) {(\phi(g))}^{-1}$ for all $g\in G$.
    \end{enumerate}
\end{lemma}

\begin{lemma}[Isomorphism from group operation]\label{bijection from element}
    Given $g$ in the group $G$, $\phi_g:G\rightarrow G$ given by $\phi_g:x\mapsto gx$ is an isomorphism (same for right multiplication).
\end{lemma}

\begin{proof}
    injectivity: $\phi_g(x)=\phi_g(y)\implies gx=gy \implies x=y$, \qquad surjectivity: given $x\in G$, $\phi_g(g^{-1}x)=x$.
\end{proof}

\begin{definition}[Image and kernel of group homomorphism]
    If $G$,$H$ are groups with $\phi:G\rightarrow H$, the \textbf{image} of $\phi$ is: \[
        \im{\phi} := \{h\in H:\exists g\in G, h = \phi(g)\}
    \textcolor{black}{.}
    \] and the \textbf{kernel} of $\phi$ is \[
        \ker{\phi} := \{g\in G: \phi(g)=e_H\}
    \textcolor{black}{.}
    \]
    These are each subgroups of $H$ and $G$ respectively.
\end{definition}

\begin{lemma}
    A group homomorphism, $\phi:G\rightarrow H$, is injective iff $\ker\phi=\{e_H\}$.
\end{lemma}

\begin{theorem}
    The composition of two compatible group homomorphisms is also a group homomorphism.
\end{theorem}

\begin{theorem}
    All cyclic groups of the same order are isomorphic.
\end{theorem}

\section{Symmetric groups}

\subsection{Disjoint cycle decomposition}

\begin{definition}
    If $f,g\in S_n$ and $x\in[1,n]$ then $f$ \textbf{fixes} $x$ if $f(x)=x$ and $f$ \textbf{moves} $x$ otherwise. 
\end{definition}

\begin{definition}
    The \textbf{support} of $f\in S_n$ is the set of points $f$ moves, $\supp(f):=\{x\in[1,n]:f(x)\neq x\}$.
\end{definition}

\begin{definition}
    If $f,g\in S_n$ satisfy $\supp(f)\cap\supp(g)=\emptyset$, $f$ and $g$ are \textbf{disjoint}.
\end{definition}

\begin{lemma}
    If $f,g\in S_n$ are disjoint, $fg=gf$.
\end{lemma}

\begin{definition}[Cycles]
    If $f\in S_n$ with $i_1,i_2,\ldots,i_r\in[1,n]$ for some $r\leq n$ such that, \[
        f(i_s) = i_{s+1 \mod(r)} \ \text{\textcolor{black}{for all }} s\in[1,r]
    \textcolor{black}{,}
    \] with $f$ fixing all other elements of $[1,n]$, then $f$ is a \textbf{cycle of length $r$} or an \textbf{$r$-cycle} and we write $f=(i_1i_1\ldots i_r)$.
\end{definition}

\begin{theorem}[Disjoint cycle form]
    if $f\in S_n$ then there exists $f_1,f_2,\ldots,f_k\in S_n$ all with disjoint supports such that $f=f_1f_2\ldots f_n$. If we further have, for all $i\in[1,k]$, both $f_i$ is not a $1$-cycle when $f\neq \id$ and $\supp(f_i)\subseteq\supp(f)$. We say $f$ is in \textbf{disjoint cycle form} or \textbf{d.c.f}.
\end{theorem}

\begin{theorem}[Uniqueness of disjoint cycles]
    The disjoint cycle form of some $f\in S_n$ is unique up to rearrangement.
\end{theorem}

\begin{theorem}
    If $f\in S_n$ is written in d.c.f as $f=f_1f_2\ldots f_k$ where $f_i$ is an $r_i$-cycle for $i\in[1,k]$ then, \begin{enumerate}
        \item $f^m=\id$ iff $f_i^m=\id$ for all $i\in[1,k]$,
        \item $\ord(f) = \lcm(r_1,r_2,\ldots r_k)$.
    \end{enumerate}
\end{theorem}

\subsection{Alternating groups}

\begin{theorem}
    Every permutation in $S_n$ can be written as the product of $2$-cycles.
\end{theorem}

\begin{definition}[Sign of a permutation]
    We define the \textbf{sign} of a permutation with the group homomorphism, $\sgn:S_n\rightarrow\{-1,1\}$ with $\sgn(i \ j) := -1$ for all $i,j\in[1,n]$ with $i\neq j$. This is defined over all permutations by the decomposition into $2$-cycles, the sign of a permutation is unique. We say $f\in S_n$ is \textbf{even} if $f\in\ker(\sgn)$ and \textbf{odd} otherwise.
\end{definition}

\begin{definition}[Alternating group]
    The \textbf{alternating group} of size $n$ is $A_n := \ker(\sgn)$ with $A_n\leq S_n$.
\end{definition}

\subsection{Dihedral groups}

\begin{definition}[Dihedral group]
    The \textbf{dihedral group} of order $2n$, denoted $D_{2n}$, is the group of symmetries of a regular $n$-gon in $\RR^3$ centered at the origin, it is often written at \[
        D_{2n} = \{e,r,r^2,\ldots,r^{n-1},s,sr,sr^2\ldots,sr^{n-1}\}
    \textcolor{black}{,}
    \] where $r$ is a rotation by $\frac{2\pi}{n}$ and $s$ is the reflection along the centre of the polygon and the first vertex.
\end{definition}

\begin{theorem}
    The elements of $D_{2n}$ can be written as elements of $S_n$ giving $D_{2n}\leq S_n$. Specifically, $r = (1 \ 2 \ \ldots \ n)$ and $s = (1)(2 \ n)(3 \ n-1)\ldots$ or $(1 \ n)(2 \ n-1)\ldots$ if $n$ is odd or even respectively.
\end{theorem}

\section{Group-like objects*}

\begin{definition}[Group-like objects]
    There are multiple axioms in the defintion of a group, sometimes we are interested in objects which lack some / all of these axioms; the names of said objects are:
    \[
        \begin{tikzcd}
            \text{Magma} \arrow[rr, "\text{associativity}"] \arrow[rd, "\text{divisibility}"] \arrow[dd, "\text{identity}"] & & \text{Semigroup} \arrow[rd] \arrow[dd] \\
            & \text{Quasigroup} \arrow[rr, crossing over] & & \text{Inverse semigroup} \arrow[dd]\\
            \text{Unital magma} \arrow[rr] \arrow[rd, "\text{inverses}"] & & \text{Monoid} \arrow[rd] \arrow[dd]\\
            & \text{Loop} \arrow[from=uu, crossing over] \arrow[rr, crossing over] & & \text{Group} \arrow[dd, "\text{commutativity}"]\\
            & & \text{Commutative Monoid} \arrow[rd] \\
            & & & \text{Abelian Group}
        \end{tikzcd}
    \textcolor{black}{.}
    \]
\end{definition}

\end{document}