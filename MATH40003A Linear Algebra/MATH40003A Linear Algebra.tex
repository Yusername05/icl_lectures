\def\module{MATH40003A Linear Algebra}
\def\lecturer{Dr Charlotte Kestner}
\def\term{Autumn 2023}
\def\cover{\vspace{1in}
$$
\begin{tikzcd}[ampersand replacement=\&, column sep=tiny]
\& \qquad \& \& \& L \arrow[dash, dashed]{dddd} \& \& \& \& \& \& \\
K\br{\alpha} \arrow[dash]{urrrr} \arrow[dash, dashed]{dddd} \& \& K\br{\alpha'} \arrow[dash]{urr} \arrow[dash, dashed]{dddd} \& \& \& K\br{\beta, \gamma} \arrow[dash]{ul} \arrow[dash, dashed]{dddd} \& \& \& K\br{\delta} \arrow[dash]{ullll} \arrow[dash, dashed]{dddd} \& \& K\br{\delta'} \arrow[dash]{ullllll} \arrow[dash, dashed]{dddd} \\
\& \& \& K\br{\beta} \arrow[crossing over, dash]{ulll} \arrow[dash]{ul} \arrow[crossing over, dash]{urr} \arrow[dash, dashed]{dddd} \& \& \& K\br{\beta\gamma} \arrow[crossing over, dash]{ul} \arrow[dash, dashed]{dddd} \& \& \& K\br{\gamma} \arrow[crossing over, dash]{ullll} \arrow[crossing over, dash]{ul} \arrow[crossing over, dash]{ur} \arrow[dash, dashed]{dddd} \& \\
\& \& \& \& \& \& \& K \arrow[crossing over, dash]{ullll} \arrow[crossing over, dash]{ul} \arrow[crossing over, dash]{urr} \arrow[dash, dashed]{dddd} \& \& \& \\
\& \qquad \& \& \& \abr{e} \& \& \& \& \& \& \\
\abr{\tau} \arrow[dash]{urrrr} \& \& \abr{\sigma^2\tau} \arrow[dash]{urr} \& \& \& \abr{\sigma^2} \arrow[dash]{ul} \& \& \& \abr{\sigma\tau} \arrow[dash]{ullll} \& \& \abr{\sigma^3\tau} \arrow[dash]{ullllll} \\
\& \& \& \abr{\sigma^2, \tau} \arrow[dash]{ulll} \arrow[dash]{ul} \arrow[dash]{urr} \& \& \& \abr{\sigma} \arrow[dash]{ul} \& \& \& \abr{\sigma^2, \sigma\tau} \arrow[dash]{ullll} \arrow[dash]{ul} \arrow[dash]{ur} \& \\
\& \& \& \& \& \& \& G \arrow[dash]{ullll} \arrow[dash]{ul} \arrow[dash]{urr} \& \& \&
\end{tikzcd}
$$
$$ G = \Gal\br{L / K} \cong \DDD_8 $$
}
\def\syllabus{Systems of linear equations, Matrices, Augmented matrices, Elementary matrices, EROs, REF \& rREF, Linear maps, Fields, Vector Spaces, Subspaces, Spanning, Linear independence, Bases, Rank Nullity, Representations}
\def\thm{subsection}

\input{../style/header}

\begin{document}

\input{../style/cover}


\section{Introduction}

\lecture{1}{Thursday}{10/01/19}

The following are references.
\begin{itemize}
\item E Artin, Galois theory, 1994
\item A Grothendieck and M Raynaud, Rev\^etements \'etales et groupe fondamental, 2002
\item I N Herstein, Topics in algebra, 1975
\item M Reid, Galois theory, 2014
\end{itemize}

\begin{notation*}
If $ K $ is a field, or a ring, I denote
$$ K\sbr{X} = \cbr{a_0 + \dots + a_nX^n \st a_i \in K}, $$
the \textbf{ring of polynomials} with coefficients in $ K $.
\end{notation*}

\section{Linear Systems and matrices}
\subsection{Linear systems}
\begin{definition}[Linear system]
    A \textbf{linear system} is a set of linear equations in the same variables.
\end{definition}

\begin{notation}
    The follow are all equivalent notation for the same linear system:
    \[
        \begin{matrix}
        a_{11}x_1 & + & a_{12}x_2 & + & \dots & + & a_{1n}x_n & = & b_1\\
        a_{21}x_1 & + & a_{22}x_2 & + & \dots & + & a_{2n}x_n & = & b_2\\
        \vdots & & \vdots & & \ddots & & \vdots & & \vdots\\
        a_{m1}x_1 & + & a_{m2}x_2 & + & \dots & + & a_{mn}x_n & = & b_m\\
        \end{matrix}
        \iff
        \begin{pmatrix}
        a_{11} & a_{12} & \dots & a_{1n}\\
        a_{21} & a_{22} & \dots & a_{2n}\\
        \vdots & \vdots & \ddots & \vdots\\
        a_{m1} & a_{m2} & \dots & a_{mn}\\
        \end{pmatrix}
        \begin{pmatrix}
            x_1 \\
            x_2 \\
            \vdots \\
            x_n \\
        \end{pmatrix} = 
        \begin{pmatrix}
            b_1 \\
            b_2 \\
            \vdots \\
            b_m \\
        \end{pmatrix}
    \]
    \[
        \iff
        \begin{amatrix}{4}
        a_{11} & a_{12} & \dots & a_{1n} & b_{1}\\
        a_{21} & a_{22} & \dots & a_{2n} & b_{2}\\
        \vdots & \vdots & \ddots & \vdots & \vdots\\
        a_{m1} & a_{m2} & \dots & a_{mn} & b_{m}\\
        \end{amatrix}
        \textcolor{black}{.}
    \]
\end{notation}

\subsection{Matrix algebra}
\begin{definition}[Matrix by elements]
    An $m\times n$ matrix written as $\textbf{A} = [a_{ij}]_{m\times n}$ has the element $a_{ij}$ in the $i$th row and $j$th column.
\end{definition}

\begin{definition}[Matrix addition]
    If $\textbf{A} = [a_{ij}]_{m\times n}$ and $\textbf{B} = [b_{ij}]_{m\times n}$ then $\textbf{A + B} := [a_{ij} + b_{ij}]_{m\times n}$.
\end{definition}

\begin{definition}[Scalar multiplication]
    If $\textbf{A} = [a_{ij}]_{m\times n}$ then $\lambda\textbf{A} := [\lambda a_{ij}]_{m\times n}$.
\end{definition}

\begin{definition}[Matrix multiplication]
    If $\textbf{A} = [a_{ij}]_{p\times q}$ and $\textbf{B} = [b_{ij}]_{q\times r}$ then $\textbf{AB} := \textbf{C} = [c_{ij}]_{p\times r}$ where $c_{ij} = \sum\limits_{k=1}^qa_{ik}b_{kj}$.
\end{definition}

\begin{remark}
    Matrix multiplication is not commutative.
\end{remark}

\subsection{EROs}
\begin{definition}[Elementary row operations]
    The three \textbf{elementary row operations (EROs)} that can be performed on augmented matrixes are as follows:
    \begin{enumerate}
        \item Multiply a row by a non-zero scalar.
        \item Swap two rows.
        \item Add a scalar multiple of a row to another row.
    \end{enumerate}
\end{definition}

\begin{remark}
    EROs preserve the set of solutions of a linear system. Each ERO has an inverse.
\end{remark}

\begin{definition}[Equivalence of linear systems]
    Two systems of linear equations are equivalent iff either:
    \begin{enumerate}
        \item They are both inconsistent.
        \item (wlog) The augmented matrix of the first system can be transformed to the augmented matrix of the second system with just EROs.
    \end{enumerate}
\end{definition}

\begin{definition}[Row echelon form / Echelon form/ REF]
    
\end{definition}

\begin{definition}[Reduced row echelon form / Row reduced echelon form / rREF]
    
\end{definition}

\subsection{Matirces of note}

\begin{definition}[Square matirx]
    A matrix is \textbf{square} iff it has the same number of rows and columns.
\end{definition}

\begin{definition}
    A square matrix ($\textbf{A} = [a_{ij}]_{n\times n}$) is: \begin{enumerate*}
        \item \textbf{Upper triangular} iff $i>j \implies a_{ij}=0$.
        \item \textbf{Lower triangular} iff $i<j \implies a_{ij}=0$.
        \item \textbf{Diagonal} iff $i\neq j \implies a_{ij}=0$ .
    \end{enumerate*}
\end{definition}

\begin{definition}[Identity matrix]
    The \textbf{identity matrix} of size $n$ written $\textbf{I}_{n}$, is the square diagonal matrix of size $n$ with all diagonal entries equal $1$.
\end{definition}

\begin{definition}[Elementary matrix]
    An \textbf{elementary matrix} is a matrix that can be achieved by appling a single ERO to the identity matrix.
\end{definition}

\begin{definition}[Inverse]
    For a square matrix $\textbf{B}$ if there exists a matrix $\textbf{B}^{-1}$ such that $\textbf{BB}^{-1} = I = \textbf{B}^{-1}\textbf{B}$ then $\textbf{B}^{-1}$ is the \textbf{inverse} of $\textbf{B}$ and vice versa.
\end{definition}

\begin{definition}[Singular]
    A matrix without an inverse is \textbf{singular}.
\end{definition}

\begin{theorem}
    The inverse of a matrix is unique
\end{theorem}

\end{document}