\documentclass[../Year2.tex]{subfiles}
\usepackage{import}
\usepackage{../style/header}

\begin{document}

\chapter{Groups and Rings}
\renewcommand*\thesection{\arabic{section}}
\lhead{MATH50005 Groups and Rings}
Lectured by Someone \\ Typed by Yu Coughlin \\
Autumn 2024

\section*{Introduction}

The following are complementary reading for the course.
\begin{itemize}
    \item G. Grimmett and D. J. A. Welsh, Probability: An Introduction, 1986
    \item J. K. Blitzstein and J. Hwang, Introduction to Probability, 2019
    \item D. F. Anderson et al, Introduction to Probability, 2018
    \item S. M. Ross, Introduction to Pro ability Models, 2014
    \item G. Grimmett and D. Stirzaker, Probability and Random Processes, 2001
    \item G. Grimmett and D. Stirzaker, One Thousand Exercises in Probability, 2009
\end{itemize}

\tableofcontents\pagebreak

\section{Quotient groups}

\subsection{Group homomorphisms}

\begin{definition}[Group isomorphism]
    Given groups $G,H$, a function $f:G\rightarrow H$ is a \textbf{group isomorphism} if it is a bijective group homomorphism. If there exists an isomorphism between groups, $G$ is \textbf{isomorphic} to $H$ written $G\cong H$.
\end{definition}

\begin{definition}[Group automorphism]
    Given $G$ a group, an isomorphism $f:G\xrightarrow{\sim}\textcolor{red}{G}$ is a \textbf{group automorphism}.
\end{definition}

\begin{theorem}
    $\Aut G$ (the set of automorphisms of a group $G$) is a group under function composition.
    \begin{proof}
    \end{proof}
\end{theorem}

\begin{theorem}
    Given groups $G,H$, if $f:G\xrightarrow{\sim}\textcolor{red}{H}$ then $f^{-1}:H\xrightarrow{\sim}\textcolor{red}{G}$.
\end{theorem}

\begin{proof}
    
\end{proof}

\subsection{Normal subgroups}

\begin{definition}[Normal subgroup]
    A sugroup $N$ of $G$ is \textbf{normal}, written $N\unlhd G$, if it satisfies any of these equal properties: \begin{enumerate}
        \item[(N1)] $N$ is the kernel of some homomorphism,
        \item[(N2)] $N$ is stable under conjugations ($\forall n\in N$ and $g\in G$, $gng^{-1}\in N$),
        \item[(N3)] for all $g\in G$ $gN=Ng$.
    \end{enumerate}
    \begin{proof}[Proof of equivalence]
        
    \end{proof}
\end{definition}

\subsection{Quotient groups}

\begin{definition}[Quotient groups]
    Let $N\unlhd G$, the \textbf{quotient group} of $G$ modulo $N$, written $G/N$, is the group with elements as left cosets of $N$ in $G$ with $(g_1N)\cdot(g_2N) = (g_1g_2N)$.

    \begin{proof}
        One can easily check this satisfies all of the group axioms. 
    \end{proof}
\end{definition}

\begin{remark}
    By Lagrange's theorem $|G/N| = |G|/|N|$.
\end{remark}

\begin{definition}[Simple group]
    A group $G$ is \textbf{simple} if it has no normal subgroups except $\{e_G\}$ and $G$.
\end{definition}

\subsection{Isomorphism theorems}

\begin{theorem}[First isomorphism theorem]\label{iso1}
    If $f:G\rightarrow H$ is a group homomorphism, $G/\ker f\cong \im f$.

    \begin{proof}
        Have $\phi:G/\ker f\rightarrow \im f$ with $\phi:g\ker f\mapsto f(g)$.
    \end{proof}
\end{theorem}

\begin{theorem}[Universal property of quotients]
    Let $N\unlhd G$ and $f:G\rightarrow H$ be a group homomorphism such that $N\subseteq \ker f$. There exists a \textit{unique} homomorphism $\tilde{f}:G/N\rightarrow H$ such that the diagram \[
        \begin{tikzcd}
            G \arrow[d, swap, "\pi"] \arrow[rd, "\phi"]\\
            G/N \arrow[r, dashed, swap, "\tilde{\phi}"] & H
        \end{tikzcd}
    \] commutes, (here $\pi:G\rightarrow G/N$ is the projection map with $\pi:g\rightarrow gN$)

    \begin{proof}
        The proof follows Theorem~\ref{iso1} with $H=\im f$.
    \end{proof}
\end{theorem}

\subsection{Centres}

\begin{definition}[Inner automorphisms]
    Given the group $G$ the conjugations by elements of $G$ form the group $\Inn G\unlhd\Aut G$.
    \begin{proof}
        
    \end{proof}
\end{definition}

\begin{definition}[Centre of group]
    Given the group $G$ the elements of $G$ that commute with all other elements form the \textbf{centre} of $G$, $Z(G)\unlhd G$.
    \begin{proof}
        Have $\phi:G\rightarrow\Aut G$ with $\phi:g\mapsto$ conjugation by $g$, $\ker \phi = Z(G)$.
    \end{proof}
\end{definition}

\begin{theorem}
    If $G/Z(G)$ is cyclic, $G$ is Abelian.
    \begin{proof}
         
    \end{proof}
\end{theorem}

\subsection{Commutators}

\subsection{$p$-primary subgroups}

\subsection{Generators}

\section{Group actions}

\subsection{Actions}

\subsection{Orbit-stabiliser theorem}

\subsection{$p$-groups}

\subsection{Jordan's theorem}

\section{Finitely generated Abelian groups}

\subsection{Smith normal form}

\subsection{Classification of finitely generated Abelian groups}

\section{Rings}

\subsection{Rings}

\subsection{Ring homomorphisms}

\subsection{Ideals}

\section{Integral domains}

\subsection{Integral domains}

\subsection{Charateristic}

\subsection{Vector spaces}

\section{PIDs and UFDs}

\subsection{Polynomial rings}

\subsection{Euclidian domains}

\subsection{Principal ideal domains}

\subsection{Unique factorisation domains}

\section{Fields}

\subsection{Field extensions}

\subsection{Constructing fields}

\subsection{Existence of finite fields}

\end{document}