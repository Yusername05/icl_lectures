\documentclass[../Year2.tex]{subfiles}
\usepackage{import}
\usepackage{../style/header}

\begin{document}

\chapter{Groups and Rings}
\renewcommand*\thesection{\arabic{section}}
\lhead{MATH50005 Groups and Rings}
Lectured by Someone \\ Typed by Yu Coughlin \\
Autumn 2024

\section*{Introduction}

The following are complementary reading for the course.
\begin{itemize}
    \item G. Grimmett and D. J. A. Welsh, Probability: An Introduction, 1986
    \item J. K. Blitzstein and J. Hwang, Introduction to Probability, 2019
    \item D. F. Anderson et al, Introduction to Probability, 2018
    \item S. M. Ross, Introduction to Pro ability Models, 2014
    \item G. Grimmett and D. Stirzaker, Probability and Random Processes, 2001
    \item G. Grimmett and D. Stirzaker, One Thousand Exercises in Probability, 2009
\end{itemize}

\tableofcontents\pagebreak

\section{Quotient groups}

\subsection{Group homomorphisms}

\begin{definition}[Group isomorphism]
    Given groups $G,H$, a function $f:G\rightarrow H$ is a \textbf{group isomorphism} if it is a bijective group homomorphism. If there exists an isomorphism between groups, $G$ is \textbf{isomorphic} to $H$ written $G\cong H$.
\end{definition}

\begin{definition}[Group automorphism]
    Given $G$ a group, an isomorphism $f:G\xrightarrow{\sim}\textcolor{red}{G}$ is a \textbf{group automorphism}.
\end{definition}

\begin{theorem}
    $\Aut G$ (the set of automorphisms of a group $G$) is a group under function composition.
    \begin{proof}
    \end{proof}
\end{theorem}

\begin{theorem}
    Given groups $G,H$, if $f:G\xrightarrow{\sim}\textcolor{red}{H}$ then $f^{-1}:H\xrightarrow{\sim}\textcolor{red}{G}$.
\end{theorem}

\begin{proof}
    
\end{proof}

\subsection{Normal subgroups}

\begin{definition}[Normal subgroup]
    A sugroup $N$ of $G$ is \textbf{normal}, written $N\unlhd G$, if it satisfies any of these equal properties: \begin{enumerate}
        \item[(N1)] $N$ is the kernel of some homomorphism,
        \item[(N2)] $N$ is stable under conjugations ($\forall n\in N$ and $g\in G$, $gng^{-1}\in N$),
        \item[(N3)] for all $g\in G$ $gN=Ng$.
    \end{enumerate}
    \begin{proof}[Proof of equivalence]
        
    \end{proof}
\end{definition}

\subsection{Quotient groups}

\begin{definition}[Quotient groups]
    Let $N\unlhd G$, the \textbf{quotient group} of $G$ modulo $N$, written $G/N$, is the group with elements as left cosets of $N$ in $G$ with $(g_1N)\cdot(g_2N) = (g_1g_2N)$.

    \begin{proof}
        One can easily check this satisfies all of the group axioms. 
    \end{proof}
\end{definition}

\begin{remark}
    By Lagrange's theorem $|G/N| = |G|/|N|$.
\end{remark}

\begin{definition}[Simple group]
    A group $G$ is \textbf{simple} if it has no normal subgroups except $\{e_G\}$ and $G$.
\end{definition}

\subsection{Isomorphism theorems}

\begin{theorem}[First isomorphism theorem]\label{iso1}
    If $f:G\rightarrow H$ is a group homomorphism, $G/\ker f\cong \im f$.

    \begin{proof}
        Have $\phi:G/\ker f\rightarrow \im f$ with $\phi:g\ker f\mapsto f(g)$.
    \end{proof}
\end{theorem}

\begin{theorem}[Universal property of quotients]
    Let $N\unlhd G$ and $f:G\rightarrow H$ be a group homomorphism such that $N\subseteq \ker f$. There exists a \textit{unique} homomorphism $\tilde{f}:G/N\rightarrow H$ such that the diagram \[
        \begin{tikzcd}[cramped]
            G \arrow[d, swap, "\pi"] \arrow[rd, "f"]\\
            G/N \arrow[r, dashed, swap, "\tilde{f}"] & H
        \end{tikzcd}
    \] commutes, (here $\pi:G\rightarrow G/N$ is the projection map with $\pi:g\rightarrow gN$).

    \begin{proof}
        The proof follows Theorem~\ref{iso1} with $H=\im f$.
    \end{proof}
\end{theorem}

\begin{definition}[Frobenius product]
    Given $A,B\subseteq G$ a group, the \textbf{(Frobenius) product} of $A$ and $B$ is \[
        AB := \{ab\in G: a\in A, b\in B\}
    \black{.}
    \]
    \vspace{-20pt}
\end{definition}

\begin{lemma}
    Given $H,N\leq G$ a group, $N$ is normal $\implies$ $HN\leq G$ and $N,H$ normal $\implies HN\unlhd G$.
    \begin{proof}
        
    \end{proof}
\end{lemma}

\begin{theorem}[Second isomorphism theorem]\label{iso2}
    If $H\leq G$ and $N\unlhd G$, $H/(H\cap N)\cong (HN)/N$. This is ometimes called the \textit{diamond theorem} due to the shape of the subgroup lattice it produces: \[
        \begin{tikzcd}[cramped, column sep=small]
            & G \arrow[d]& \\
            & HN \arrow[ld] \arrow [rd]& \\
            H \arrow[rd] & & N \arrow[ld]\\
            & H\cap N \arrow[d]& \\
            & \{e_G\}& 
        \end{tikzcd}
    \] where arrows point to subgroups.
\end{theorem}

\begin{note}
    There are third and fourth isomorphism theorems that will not appear in this module.
\end{note}

\subsection{Centres}

\begin{definition}[Inner automorphisms]
    Given the group $G$ the conjugations by elements of $G$ form the group $\Inn G\unlhd\Aut G$.
    \begin{proof}
        
    \end{proof}
\end{definition}

\begin{definition}[Centre of group]
    Given the group $G$ the elements of $G$ that commute with all other elements form the \textbf{centre} of $G$, $Z(G)\unlhd G$.
    \begin{proof}[Proof of normality]
        Have $\phi:G\rightarrow\Aut G$ with $\phi:g\mapsto$ conjugation by $g$, $\ker \phi = Z(G)$.
    \end{proof}
\end{definition}

\begin{theorem}
    If $G/Z(G)$ is cyclic, $G$ is Abelian.
    \begin{proof}
         
    \end{proof}
\end{theorem}

\begin{definition}[$p$-group]
    A finite group $G$ is a \textbf{$p$-group} is the order of $G$ is a power of prime $p$.
\end{definition}

\begin{theorem}
    Let $G$ be a $p$-group, $Z(G)\neq \{e_G\}$.
\end{theorem}

\subsection{Commutators}

\begin{definition}[Commutator]
    For $a,b\in G$ a group, we have $[a,b]:=aba^{-1}b^{-1}$ the \textbf{commutator} of $a$ and $b$. $[G,G]$ is the smallest subgroup of $G$ containing all commutators of elements of $G$, called the \textbf{commutator} of $G$.
\end{definition}

\begin{remark}
    A group $G$ is Abelian iff $[G,G]=e_G$.
\end{remark}

\begin{theorem}
    Given $G$ a group, $[G,G]\unlhd G$ with its quotient in $G$ Abelian.
\end{theorem}

\begin{theorem}
    Let $N\unlhd G$, $G/N$ is Abelian iff $[G,G]\subseteq N$.
\end{theorem}

\begin{theorem}
    Given a group $G$ with $A,B\unlhd G$, $A\cap B=\{e_G\}$ and $AB=G$; $A\times B\cong G$.
\end{theorem}

\subsection{Torsion and $p$-primary subgroups}

\begin{definition}[Torsion subgroup]
    Given an abelian group $G$, the set of elemnts of $G$ with finite order form the \textbf{torsion subgroup} of $G$, denoted $G_{\tors}$. When $G=G_{\tors}$, we call $G$ a \textbf{torsion Abelian group}.
\end{definition}

\begin{definition}[$p$-primary subgroups]
    Given an abelian group $G$, the set of elements of $g$ with order $p$ (a prime) is the \textbf{$p$-primary subgroup} of $G$, written $G\{p\}$. When $G=G_G\{p\}$, we call $G$ a \textbf{$p$-primary torsion Abelian group}.
\end{definition}

\begin{theorem}
    Let the prime factorisation of $n\in\NN$ be $p_1^{a_1}p_2^{a_2}\ldots p_m^{a_m}$ with $C_n$ the cyclic group of order $n$. \[
        C_n\cong C_{p_1^{a_1}}\times C_{p_2^{a_2}} \times \cdots \times C_{p_m^{a_m}}
    \black{.}
    \]
\vspace{-20pt}
    \begin{proof}
        
    \end{proof}
\end{theorem}

\subsection{Generators}

\begin{lemma}
    Given an indexing set $\III$, and a sequence of subgroups $(H_i)_{i\in\III}\leq H$, $\displaystyle \bigcap_{i\in\III}H_i\leq G$.
\end{lemma}

\begin{definition}[Subgroup generated by a set]
    Given $S\subseteq G$ a group,  \[
        \abr{S}:=\br{\bigcap_{S\subseteq H\leq G}H} \leq G
    \] is the \textbf{subgroup of $G$ generated by $S$}. If $\abr{S}=G$ then we say $S$ \textbf{generates} $G$ and $G$ is \textbf{finitely generated} is $S$ is finite.
\end{definition}

\subsection{Classification of finitely generated Abelian groups}
\begin{definition}[Free Abelian group of rank $n$]
    The \textbf{Free Abelian group of rank $n$} is the group $\ZZ^n$ under addition. The free abelian group of rank 0 is the trivial group.
\end{definition}

\begin{lemma}
    If $\ZZ^m\cong\ZZ^n$ then $n=m$, so the rank of a free abelian group is well defined.
\end{lemma}

\begin{lemma}
    Any subgroup of $\ZZ^n$ is isomorphic to some $\ZZ^m$ for some $m\leq m$.
\end{lemma}

\begin{theorem}
    Every finitely generated Abelian group is isomorphic to a product of finitely many cyclic groups.
\end{theorem}

\begin{theorem}
    Every finitely generated Abelian group is isomorphic to a product of finitely many infinite cyclic groups and finitely many cyclic groups of prime order. The number of ininfite cyclic factors and the number of cclic factors of order $p^r$, where $p$ is primse and $r\in\NN$ is determined solely by the group.
\end{theorem}

\begin{theorem}
    A finitely generated Abelian group, $G$, is not cyclic iff there exists a prime $p$ such that $G\cong C_p\times C_p$.
\end{theorem}

\section{Group actions}

\subsection{Actions}

\begin{definition}[Actions]
    Given a group $G$ and a set $X$, a \textbf{group action} is: a binary operation \[
        \function[\cdot]{G\times X}{X}{(g,x)}{g\cdot x}
    \] with $e_G\cdot x=x$ for all $x\in X$ and $(g_1g_2)\cdot x = g_1 \cdot (g_2 x)$ for all $g_1,g_2\in G$ and $x\in X$; or, equivalently, a homomorphism $\rho:G\rightarrow \Sym(X)$.
\end{definition}

\begin{definition}[Faithful set]
    An action of a group $G$ on a set $X$ is \textbf{faithful} if the map $\rho:G\rightarrow\Sym(X)$ is injective.
\end{definition}

\subsection{Orbit-stabiliser theorem}

\begin{definition}[Orbit]
    Given a group $G$ acting on a set $X$, the \textbf{$G$-orbit} of $x\in X$ is \[
        G(x):=\{g\cdot x:g\in G\} \subseteq X
    \black{.}
    \] Orbits partition $X$ into $X/G$.
\end{definition}

\begin{definition}[Stabiliser]
    Given a group $G$ acting on a set $X$, the \textbf{stabiliser} of $x\in X$ is \[
        \Stab_G(x):= \{g\in G: g\cdot x = x\} \subseteq G
        \black{.}
    \] Stabilisers also partition $G$.
\end{definition}

\begin{lemma}
    Given a group $G$ acting on a set $X$, $\Stab_G(g\cdot x)=g\Stab_G(x)g^{-1}$
\end{lemma}

\begin{theorem}[Orbit-stabiliser theorem]
    Given a group $G$ acting on a set $X$. For all $x\in X$, we have $\phi_x:G/\Stab(x)\xrightarrow{\sim}\textcolor{red}{G(x)}$ by $\phi_x:g\Stab(x)\mapsto g\cdot x$, giving $|G(x)| = [G:\Stab(x)] = |G|/|\Stab(x)|$.
    \begin{proof}
        asdfsd
    \end{proof}
\end{theorem}

\vspace{-15pt}

\begin{corollary}
    $\displaystyle|X| = \sum_{i=1}^n|G(x_i)| = \sum_{i=1}^n[G:\Stab(x_i)]$. 
\end{corollary}

\begin{corollary}[Cayley's theorem]
    Let $G$ be a finite group of order $n$. Then $S_n$ contains a finite subgroup isomorphic to $G$.
\end{corollary}

\begin{corollary}[Cauchy's theorem]
    Let $G$ be a finite group of order $n$ and let $p$ be a prime factor of $n$. Then $G$ contains an element of order $p$.
\end{corollary}

\subsection{Jordan's theorem}

\begin{definition}[Transitive action]
    Given a group $G$ acting on a set $X$, if $X$ is a $G$-orbit then we say $G$ acts \textbf{transitively} on $X$.
\end{definition}

\begin{definition}[Fixed points]
    Given a group $G$ acting on a set $X$, an element $x\in X$ is a \text{fixed point} of $g\in G$ iff $g\cdot x=x$. We have $\Fix(g)\subseteq X$ the set of fixed points of $g\in G$ satisying: \[
        \begin{tikzcd}
            \Stab(x) & \arrow[l, "\pi_G"] \{(x,g)\in X\times G; \  g\cdot x = x\} \arrow[r, swap, "\pi_X"] & \Fix(g)
        \end{tikzcd}
        \black{.}
    \]
\end{definition}

\vspace{-20pt}

\begin{theorem}[Jordan's theorem]
    Let $G$ act transitively on a finite set $X$, we have \[
        \sum_{g\in G}|\Fix(g)| = |G|
    \black{,}
    \] with there being some element $g\in G$ such that $\Fix(g)=\emptyset$.
\end{theorem}

\begin{corollary}[Burnside's lemma]
    Given a group $G$ acting on a finite set $X$: \[
        |X/G| = \frac{1}{|G|}\sum_{g\in G}|\Fix(g)|
        \black{.}
    \]
\end{corollary}

\vspace{-30pt}

\section{Rings}

\subsection{Rings}

\begin{definition}[Ring]
    A ring (with $1$) is a set $R$ with elements $0,1$ and binary operations $+,\times$ such that \begin{enumerate}
        \item $(R,+)$ is an abelian group with identity $0$,
        \item $(R,\times)$ is a semigroup with $1$ as the identity,
        \item both left and right multiplication are distributive over addition.
    \end{enumerate}
\end{definition}

\begin{examples}
    $\ZZ,\QQ,\RR,\CC$ are all rings with their normal operations. $\RR[x]$ is the set of real-valued polynomials and is also a ring.
\end{examples}

\begin{definition}[Subring]
    A subset of a ring wich is itself a ring under the same operators with the same $1$ is a \textbf{subring}.
\end{definition}

\begin{definition}[Commutative ring]
    A ring, $R$, is \textbf{commutative} iff $a+b=b+a$ for all $a,b\in\RR$.
\end{definition}

\begin{definition}[Invertible]
    An element $x$ of a ring $R$ is \text{invertible} if there exists $y,z\in R$ with $yx=zx=1$.
\end{definition}

\begin{definition}[Division ring]
    A ring $R$ is called a \textbf{division ring} if $R\setminus\{0\}$ is a group under multiplication with identity $1$.
\end{definition}

\begin{remark}
    A commutative division ring is a field.
\end{remark}

\subsection{Ring homomorphisms}

\subsection{Ideals}

\section{Integral domains}

\subsection{Integral domains}

\subsection{Charateristic}

\subsection{Vector spaces}

\section{PIDs and UFDs}

\subsection{Polynomial rings}

\subsection{Euclidian domains}

\subsection{Principal ideal domains}

\subsection{Unique factorisation domains}

\section{Fields}

\subsection{Field extensions}

\subsection{Constructing fields}

\subsection{Existence of finite fields}

\end{document}