\documentclass[../Year2.tex]{subfiles}
\usepackage{import}
\usepackage{../style/header}

\begin{document}

\chapter{Groups and Rings}
\renewcommand*\thesection{\arabic{section}}
\lhead{MATH50005 Groups and Rings}
Lectured by Someone \\ Typed by Yu Coughlin \\
Autumn 2024

\section*{Introduction}

The following are complementary reading for the course.
\begin{itemize}
    \item G. Grimmett and D. J. A. Welsh, Probability: An Introduction, 1986
    \item J. K. Blitzstein and J. Hwang, Introduction to Probability, 2019
    \item D. F. Anderson et al, Introduction to Probability, 2018
    \item S. M. Ross, Introduction to Pro ability Models, 2014
    \item G. Grimmett and D. Stirzaker, Probability and Random Processes, 2001
    \item G. Grimmett and D. Stirzaker, One Thousand Exercises in Probability, 2009
\end{itemize}

\tableofcontents\pagebreak

\section{Quotient groups}

\subsection{Group homomorphisms}

\begin{definition}[Group isomorphism]
    Given groups $G,H$, a function $f:G\rightarrow H$ is a \textbf{group isomorphism} if it is a bijective group homomorphism. If there exists an isomorphism between groups, $G$ is \textbf{isomorphic} to $H$ written $G\cong H$.
\end{definition}

\begin{definition}[Group automorphism]
    Given $G$ a group, an isomorphism $f:G\xrightarrow{\sim}\textcolor{red}{G}$ is a \textbf{group automorphism}.
\end{definition}

\begin{theorem}
    $\Aut G$ (the set of automorphisms of a group $G$) is a group under function composition.
    \begin{proof}
        By examining the defintion of $\Aut G$, taking $e=\id$ and showing association elementwise.
    \end{proof}
\end{theorem}

\begin{theorem}
    Given groups $G,H$, if $f:G\xrightarrow{\sim}\textcolor{red}{H}$ then $f^{-1}:H\xrightarrow{\sim}\textcolor{red}{G}$.
\end{theorem}

\begin{proof}
    
\end{proof}

\subsection{Normal subgroups}

\begin{definition}[Normal subgroup]
    A sugroup $N$ of $G$ is \textbf{normal}, written $N\unlhd G$, if it satisfies any of these equal properties: \begin{enumerate}
        \item[(N1)] $N$ is the kernel of some homomorphism,
        \item[(N2)] $N$ is stable under conjugations ($\forall n\in N$ and $g\in G$, $gng^{-1}\in N$),
        \item[(N3)] for all $g\in G$ $gN=Ng$.
    \end{enumerate}
    \begin{proof}[Proof of equivalence]
        
    \end{proof}
\end{definition}

\subsection{Quotient groups}

\begin{definition}[Quotient groups]
    Let $N\unlhd G$, the \textbf{quotient group} of $G$ modulo $N$, written $G/N$, is the group with elements as left cosets of $N$ in $G$ with $(g_1N)\cdot(g_2N) = (g_1g_2N)$.

    \begin{proof}
        One can easily check this satisfies all of the group axioms. 
    \end{proof}
\end{definition}

\begin{remark}
    By Lagrange's theorem $|G/N| = |G|/|N|$.
\end{remark}

\begin{definition}[Simple group]
    A group $G$ is \textbf{simple} if it has no normal subgroups except $\{e_G\}$ and $G$.
\end{definition}

\subsection{Isomorphism theorems}

\begin{theorem}[First isomorphism theorem]\label{iso1}
    If $f:G\rightarrow H$ is a group homomorphism, $G/\ker f\cong \im f$.

    \begin{proof}
        Have $\phi:G/\ker f\rightarrow \im f$ with $\phi:g\ker f\mapsto f(g)$.
    \end{proof}
\end{theorem}

\begin{theorem}[Universal property of quotients]
    Let $N\unlhd G$ and $f:G\rightarrow H$ be a group homomorphism such that $N\subseteq \ker f$. There exists a \textit{unique} homomorphism $\tilde{f}:G/N\rightarrow H$ such that the diagram \[
        \begin{tikzcd}[cramped]
            G \arrow[d, swap, "\pi"] \arrow[rd, "f"]\\
            G/N \arrow[r, dashed, swap, "\tilde{f}"] & H
        \end{tikzcd}
    \] commutes, (here $\pi:G\rightarrow G/N$ is the projection map with $\pi:g\rightarrow gN$).

    \begin{proof}
        The proof follows Theorem~\ref{iso1} with $H=\im f$.
    \end{proof}
\end{theorem}

\begin{definition}[Frobenius product]
    Given $A,B\subseteq G$ a group, the \textbf{(Frobenius) product} of $A$ and $B$ is \[
        AB := \{ab\in G: a\in A, b\in B\}
    \black{.}
    \]
    \vspace{-20pt}
\end{definition}

\begin{lemma}
    Given $H,N\leq G$ a group, $N$ is normal $\implies$ $HN\leq G$ and $N,H$ normal $\implies HN\unlhd G$.
    \begin{proof}
        
    \end{proof}
\end{lemma}

\begin{theorem}[Second isomorphism theorem]\label{iso2}
    If $H\leq G$ and $N\unlhd G$, $H/(H\cap N)\cong (HN)/N$. This is ometimes called the \textit{diamond theorem} due to the shape of the subgroup lattice it produces: \[
        \begin{tikzcd}[cramped, column sep=small]
            & G \arrow[d]& \\
            & HN \arrow[ld] \arrow [rd]& \\
            H \arrow[rd] & & N \arrow[ld]\\
            & H\cap N \arrow[d]& \\
            & \{e_G\}& 
        \end{tikzcd}
    \] where arrows point to subgroups.
\end{theorem}

\begin{note}
    There are third and fourth isomorphism theorems that will not appear in this module.
\end{note}

\subsection{Centres}

\begin{definition}[Inner automorphisms]
    Given the group $G$ the conjugations by elements of $G$ form the group $\Inn G\unlhd\Aut G$.
    \begin{proof}
        
    \end{proof}
\end{definition}

\begin{definition}[Centre of group]
    Given the group $G$ the elements of $G$ that commute with all other elements form the \textbf{centre} of $G$, $Z(G)\unlhd G$.
    \begin{proof}[Proof of normality]
        Have $\phi:G\rightarrow\Aut G$ with $\phi:g\mapsto$ conjugation by $g$, $\ker \phi = Z(G)$.
    \end{proof}
\end{definition}

\begin{theorem}
    If $G/Z(G)$ is cyclic, $G$ is Abelian.
    \begin{proof}
         
    \end{proof}
\end{theorem}

\begin{definition}[$p$-group]
    A finite group $G$ is a \textbf{$p$-group} is the order of $G$ is a power of prime $p$.
\end{definition}

\begin{theorem}
    Let $G$ be a $p$-group, $Z(G)\neq \{e_G\}$.
\end{theorem}

\subsection{Commutators}

\begin{definition}[Commutator]
    For $a,b\in G$ a group, we have $[a,b]:=aba^{-1}b^{-1}$ the \textbf{commutator} of $a$ and $b$. $[G,G]$ is the smallest subgroup of $G$ containing all commutators of elements of $G$, called the \textbf{commutator} of $G$.
\end{definition}

\begin{remark}
    A group $G$ is Abelian iff $[G,G]=e_G$.
\end{remark}

\begin{theorem}
    Given $G$ a group, $[G,G]\unlhd G$ with its quotient in $G$ Abelian.
\end{theorem}

\begin{theorem}
    Let $N\unlhd G$, $G/N$ is Abelian iff $[G,G]\subseteq N$.
\end{theorem}

\begin{theorem}
    Given a group $G$ with $A,B\unlhd G$, $A\cap B=\{e_G\}$ and $AB=G$; $A\times B\cong G$.
\end{theorem}

\subsection{Torsion and $p$-primary subgroups}

\begin{definition}[Torsion subgroup]
    Given an abelian group $G$, the set of elemnts of $G$ with finite order form the \textbf{torsion subgroup} of $G$, denoted $G_{\tors}$. When $G=G_{\tors}$, we call $G$ a \textbf{torsion Abelian group}.
\end{definition}

\begin{definition}[$p$-primary subgroups]
    Given an abelian group $G$, the set of elements of $g$ with order $p$ (a prime) is the \textbf{$p$-primary subgroup} of $G$, written $G\{p\}$. When $G=G_G\{p\}$, we call $G$ a \textbf{$p$-primary torsion Abelian group}.
\end{definition}

\begin{theorem}
    Let the prime factorisation of $n\in\NN$ be $p_1^{a_1}p_2^{a_2}\ldots p_m^{a_m}$ with $C_n$ the cyclic group of order $n$. \[
        C_n\cong C_{p_1^{a_1}}\times C_{p_2^{a_2}} \times \cdots \times C_{p_m^{a_m}}
    \black{.}
    \]
\vspace{-20pt}
    \begin{proof}
        
    \end{proof}
\end{theorem}

\subsection{Generators}

\begin{lemma}
    Given an indexing set $\III$, and a sequence of subgroups $(H_i)_{i\in\III}\leq H$, $\displaystyle \bigcap_{i\in\III}H_i\leq G$.
\end{lemma}

\begin{definition}[Subgroup generated by a set]
    Given $S\subseteq G$ a group,  \[
        \abr{S}:=\br{\bigcap_{S\subseteq H\leq G}H} \leq G
    \] is the \textbf{subgroup of $G$ generated by $S$}. If $\abr{S}=G$ then we say $S$ \textbf{generates} $G$ and $G$ is \textbf{finitely generated} is $S$ is finite.
\end{definition}

\subsection{Classification of finitely generated Abelian groups}
\begin{definition}[Free Abelian group of rank $n$]
    The \textbf{Free Abelian group of rank $n$} is the group $\ZZ^n$ under addition. The free abelian group of rank 0 is the trivial group.
\end{definition}

\begin{lemma}
    If $\ZZ^m\cong\ZZ^n$ then $n=m$, so the rank of a free abelian group is well defined.
\end{lemma}

\begin{lemma}
    Any subgroup of $\ZZ^n$ is isomorphic to some $\ZZ^m$ for some $m\leq m$.
\end{lemma}

\begin{theorem}
    Every finitely generated Abelian group is isomorphic to a product of finitely many cyclic groups.
\end{theorem}

\begin{theorem}
    Every finitely generated Abelian group is isomorphic to a product of finitely many infinite cyclic groups and finitely many cyclic groups of prime order. The number of ininfite cyclic factors and the number of cclic factors of order $p^r$, where $p$ is primse and $r\in\NN$ is determined solely by the group.
\end{theorem}

\begin{theorem}
    A finitely generated Abelian group, $G$, is not cyclic iff there exists a prime $p$ such that $G\cong C_p\times C_p$.
\end{theorem}

\section{Group actions}

\subsection{Actions}

\begin{definition}[Actions]
    Given a group $G$ and a set $X$, a \textbf{group action} is: a binary operation \[
        \function[\cdot]{G\times X}{X}{(g,x)}{g\cdot x}
    \] with $e_G\cdot x=x$ for all $x\in X$ and $(g_1g_2)\cdot x = g_1 \cdot (g_2 x)$ for all $g_1,g_2\in G$ and $x\in X$; or, equivalently, a homomorphism $\rho:G\rightarrow \Sym(X)$.
\end{definition}

\begin{definition}[Faithful set]
    An action of a group $G$ on a set $X$ is \textbf{faithful} if the map $\rho:G\rightarrow\Sym(X)$ is injective.
\end{definition}

\subsection{Orbit-stabiliser theorem}

\begin{definition}[Orbit]
    Given a group $G$ acting on a set $X$, the \textbf{$G$-orbit} of $x\in X$ is \[
        G(x):=\{g\cdot x:g\in G\} \subseteq X
    \black{.}
    \] Orbits partition $X$ into $X/G$.
\end{definition}

\begin{definition}[Stabiliser]
    Given a group $G$ acting on a set $X$, the \textbf{stabiliser} of $x\in X$ is \[
        \Stab_G(x):= \{g\in G: g\cdot x = x\} \subseteq G
        \black{.}
    \] Stabilisers also partition $G$.
\end{definition}

\begin{lemma}
    Given a group $G$ acting on a set $X$, $\Stab_G(g\cdot x)=g\Stab_G(x)g^{-1}$
\end{lemma}

\begin{theorem}[Orbit-stabiliser theorem]
    Given a group $G$ acting on a set $X$. For all $x\in X$, we have $\phi_x:G/\Stab(x)\xrightarrow{\sim}\textcolor{red}{G(x)}$ by $\phi_x:g\Stab(x)\mapsto g\cdot x$, giving $|G(x)| = [G:\Stab(x)] = |G|/|\Stab(x)|$.
    \begin{proof}
        asdfsd
    \end{proof}
\end{theorem}

\vspace{-15pt}

\begin{corollary}
    $\displaystyle|X| = \sum_{i=1}^n|G(x_i)| = \sum_{i=1}^n[G:\Stab(x_i)]$. 
\end{corollary}

\begin{corollary}[Cayley's theorem]
    Let $G$ be a finite group of order $n$. Then $S_n$ contains a finite subgroup isomorphic to $G$.
\end{corollary}

\begin{corollary}[Cauchy's theorem]
    Let $G$ be a finite group of order $n$ and let $p$ be a prime factor of $n$. Then $G$ contains an element of order $p$.
\end{corollary}

\subsection{Jordan's theorem}

\begin{definition}[Transitive action]
    Given a group $G$ acting on a set $X$, if $X$ is a $G$-orbit then we say $G$ acts \textbf{transitively} on $X$.
\end{definition}

\begin{definition}[Fixed points]
    Given a group $G$ acting on a set $X$, an element $x\in X$ is a \text{fixed point} of $g\in G$ iff $g\cdot x=x$. We have $\Fix(g)\subseteq X$ the set of fixed points of $g\in G$ satisying: \[
        \begin{tikzcd}
            \Stab(x) & \arrow[l, "\pi_G"] \{(x,g)\in X\times G; \  g\cdot x = x\} \arrow[r, swap, "\pi_X"] & \Fix(g)
        \end{tikzcd}
        \black{.}
    \]
\end{definition}

\vspace{-20pt}

\begin{theorem}[Jordan's theorem]
    Let $G$ act transitively on a finite set $X$, we have \[
        \sum_{g\in G}|\Fix(g)| = |G|
    \black{,}
    \] with there being some element $g\in G$ such that $\Fix(g)=\emptyset$.
\end{theorem}

\begin{corollary}[Burnside's lemma]
    Given a group $G$ acting on a finite set $X$: \[
        |X/G| = \frac{1}{|G|}\sum_{g\in G}|\Fix(g)|
        \black{.}
    \]
\end{corollary}

\vspace{-30pt}

\section{Rings}

\subsection{Rings}

\begin{definition}[Ring]
    A ring (with $1$) is a set $R$ with elements $0,1$ and binary operations $+,\times$ such that \begin{enumerate}
        \item $(R,+)$ is an abelian group with identity $0$,
        \item $(R,\times)$ is a semigroup with $1$ as the identity,
        \item both left and right multiplication are distributive over addition.
    \end{enumerate}
\end{definition}

\begin{examples}
    $\ZZ,\QQ,\RR,\CC$ are all rings with their normal operations. $\RR[x]$ is the set of real-valued polynomials and is also a ring.
\end{examples}

\begin{definition}[Subring]
    A subset of a ring wich is itself a ring under the same operators with the same $1$ is a \textbf{subring}.
\end{definition}

\begin{definition}[Commutative ring]
    A ring, $R$, is \textbf{commutative} iff $a+b=b+a$ for all $a,b\in\RR$.
\end{definition}

\begin{definition}[Invertible]
    An element $x$ of a ring $R$ is \text{invertible} if there exists $y,z\in R$ with $yx=zx=1$.
\end{definition}

\begin{definition}[Division ring]
    A ring $R$ is called a \textbf{division ring} if $R\setminus\{0\}$ is a group under multiplication with identity $1$.
\end{definition}

\begin{remark}
    A commutative division ring is a field.
\end{remark}

\begin{definition}[Integral domain]
    A commutative ring $R$ is an integral domain iff $0\neq1$ and for all $a,b\in R$ $ab=0\implies a=0$ or $b=0$.
\end{definition}

\subsection{Ring homomorphisms}

\begin{definition}[Ring homomorphism]
    Let $R,S$ be rings, a function $f:R\rightarrow S$ is a \textbf{ring homomorphism} iff it satisfies \begin{enumerate}
        \item $f:(R,+)\rightarrow (S,+)$ is a group homomorphism,
        \item $f(xy)=f(x)f(y)$ for all $x,y\in R$,
        \item $f(1_R)=1_S$.
    \end{enumerate}
\end{definition}

\begin{lemma}
    Given the ring homomorphism $f:R\rightarrow S$ the kernel of $f$ is a subgroup of $(R,+)$ which satisfies $xr,rx\in\ker f$ for all $x\in \ker f$ and $r\in R$.
\end{lemma}

\subsection{Ideals}

\begin{definition}[Ideal]
    For a ring $R$, a subset $I\subseteq R$ is a \textbf{left ideal}, denoted $I\unlhd R$ iff \begin{enumerate}
        \item $(I,+)$ is a subgroups of $(R,+)$,
        \item if $r\in R$ and $i\in I$, $ri\in R$.
    \end{enumerate} Similarly, for \textbf{right ideals}. A subset $I$ is a bi-ideal if it is both a left and right ideal.
\end{definition}

\begin{definition}[Quotient ring]
    Given ring $R$ with proper ideal $I\subset R$, The quotient abelian group $R/I$, with natural multiplication, forms the \textbf{quotient ring} of $R$ by $I$.
\end{definition}

\begin{definition}[Principal ideal]
    Given a commutative ring $R$ and some $a\in R$, $aR:= \{ax:x\in R\}$ is an ideal called a \textbf{principal ideal} with \textbf{generator} a.
\end{definition}

\begin{definition}
    A bijective ring homomorphism is a \textbf{ring isomorphism}, a ring homomorphism $f:R\rightarrow R$ is a \textbf{ring endomorphism}, an isomorphic ring endomorphism is \textbf{ring automorphism}.
\end{definition}

\begin{proposition}
    Given the ring homomorphism $f:R\rightarrow S$, $f(R)=\im R$ is a subring of $S$ which is isomorphic to $R/\ker f$.
\end{proposition}

\begin{proposition}
    A commutative ring is a field iff its only proper ideal is the trivial / zero ideal.
\end{proposition}

\begin{proposition}
    Given $f:R\rightarrow S$ a ring homomorphism with $J$ a left (or right or bi) ideal of $S$, $f^{-1}(J)$ is a left (respectively ) ideal of $R$.
\end{proposition}

\begin{definition}[Prime ideal]
    Let $R$ be a commutative ring, a proper ideal $I\subset R$ is a \textbf{prime ideal} iff $ab\in I \ \black{for } a,b\in R \implies a\in I \ \black{or } b\in I$.
\end{definition}

\begin{theorem}
    If $I\subset R$ is a prime ideal, $R/I$ is an integral domain
\end{theorem}

\begin{definition}[Maximal ideal]
    A proper ideal $I$ in a commutative rign $R$ is \textbf{maximal} iff there are no other proper ideals $J$ with $I\subset J$.
\end{definition}

\begin{theorem}
    $I$ is a maximal ideal of $R$ iff $R/I$ is a field.
\end{theorem}

\section{Integral domains}
Throughout this section we will always have $R$ be an integral domain.

\subsection{Integral domains}

\begin{theorem}
    $ab=ac\implies b=c$ for all $a,b,c\in R$. (the cancellation law holds for all integral domains)
\end{theorem}

\begin{proposition}
    For $a,b\in R$, $aR=bR$ iff $a=br$ for some $r\neq 0 \in R$.
    \begin{proof}
        
    \end{proof}
\end{proposition}

\begin{theorem}
    All fields are integral domains and all finite integral domains are fields.
\end{theorem}

\begin{remark}
    The ring $\ZZ/n\ZZ$ is an integral domain iff it is a field $\iff$ n is prime.
\end{remark}

\begin{definition}[Unit]
    $r\in R$ is a \textbf{unit} if there exists some $y\in R$ with $x\times y=1_R$. We write  $R^\times$ for the group of units in $R$ under multiplication.
\end{definition}

\begin{definition}[Irreducible]
    $r\in R\setminus R^\times$ is \textbf{irreducible} if it cannot be written as the product of two elements of $R\setminus R^\times$.
\end{definition}

\subsection{Charateristic}

\begin{lemma}
    For any ring $S$ there is a uniqure ring homomorphism $f:\ZZ\rightarrow S$.
    \begin{proof}
        Have $f(0_R)=0$, $f(1)\rightarrow 1_S$ and inductively have $f(n)$ be the sum of $1_S$ $n$ times.
    \end{proof}
\end{lemma}

\begin{lemma}
    The kernel of the unique homomorphism $\ZZ\rightarrow R$ is either $\{0\}$ or $p\ZZ$ for some prime $p$.
\end{lemma}

\begin{definition}[Charateristic]
    The \textbf{characteristic} of $R$ is the unique non-negative generator of the kernel of $\ZZ\rightarrow R$, denoted $\text{char}\ R$.
\end{definition}

\subsection{Polynomial rings}

\begin{definition}[Polynomial ring]
    $R[t]$ is, formally, the set of infinite sequences of elements of $R$ with finitely many non-zero terms, but more helpfully: the set of polynomials in $t$ with coefficients in $R$.
\end{definition}

\begin{definition}[Polynomial degree]
    The \textbf{degree} of a polynomial, $r_0 + r_1t + r_2t^2 + \ldots + r_i t^i + \ldots \in R[t]$, is the unique maximum $i\in\NN$ with $r_i\neq 0$ and $0$ otherwise.
\end{definition}

\begin{lemma}
    Given $p(t),q(t)\in R$, $\deg(p(t)q(t))=\deg(p(t))+\deg(q(t))$, $R[t]$ is an integral domain and $R[t]^* = R^*$.
\end{lemma}

\begin{theorem}
    If $k$ is a field with $a(t),b(t)\in k[t]$ with $b(t)\neq 0$, there exists $q(t),r(t)\in k[t]$ such that $a(t)=q(t)b(t)=r(t)$ with $\deg(r(t))<\deg(b(t))$ and $q(t),r(t)$ unique.
\end{theorem}

\section{PIDs and UFDs}

\subsection{Euclidian domains}

\begin{definition}[Euclidian domain]
    An integral domain $R$ is a Euclidian domain if there exists some  $\phi:R^*\rightarrow\NN_0$ satsifying: \begin{enumerate}
        \item $\phi(ab)\leq\phi(a)$ for all $a,b\neq 0$,
        \item for all $a,b\in R$ there exists $q,r\in R$ with $a=qb+r$ with $r=0$ or $\phi(r)\leq\phi(b)$.
    \end{enumerate}
\end{definition}

\subsection{Principal ideal domains}

\begin{definition}[Principal integral domain]
    An integral domain $R$ is a \textbf{principal integral domain} iff every ideal of $R$ is principal.
\end{definition}

\begin{theorem}
    R is a Euclidian domain $\implies$ $R$ is a principal integral domain.
    \begin{proof}
        
    \end{proof}
\end{theorem}

\begin{corollary}
    $F$ is a field $\implies F[t]$ is a PID.
\end{corollary}

\subsection{Unique factorisation domains}

\begin{definition}[Unique factorisation domain]
    An integral domain $R$ is a \textbf{unique factorisation domain} iff every element of $R\setminus R^\times$ can be written as the product of a single unit and finitely many irreducibles in $R$ which is unique up to rearrangement.
\end{definition}

\begin{definition}[Division]
    Given $a,b$ in the integral domain $R$, we say $a$ \textbf{divides} $b$, written $a|b$ iff $b=ra$ for some $r\in R$ and \textbf{properly divides} if $r\not\in R^\times$.
\end{definition}

\begin{lemma}\label{ufd1}
    Given $p,a,b\in R$ a UFD, if $p$ is irreducible then $p|ab\implies p|a$ or $p|b$.
\end{lemma}

\begin{lemma}\label{ufd2}
    There is no infinite sequence of non-zero $r_1,r_2,\ldots\in R$ a UFD such that $r_{n+1}$ properly divides $r$ for all $n\geq1$.
\end{lemma}

\begin{theorem}
    The integral domain $R$ is a UFD iff the properties in Lemma~\ref{ufd1} and Lemma~\ref{ufd2} hold.
\end{theorem}

\begin{theorem}
    Every principal ideal domain is a unique factorisation domain.
\end{theorem}

\section{Fields}

\subsection{Vector spaces}
Throughout this section let $k$ be a field.

\begin{definition}[Vector space]
    A $k$-vector space $V$ is an abelian group with an action of $k$ on the elements of $V$ satisfying \begin{enumerate}
        \item $1_k v=v$ for all $v\in V$,
        \item $(x+y)V = xv + yv$ for all $x,y\in k$ and $v\in V$,
        \item $x(v+w) = xv + xw$ for all $x\in k$ and $v,w\in V$.
    \end{enumerate}
\end{definition}

\begin{proposition}
    If $\ch k=0$ then $k$ contains a unique subfield isomorphic to $\QQ$. Otherwise, if $\ch k = p$ then $k$ contains a unique subfield isomorphic to $\FF_p$.
\end{proposition}

\begin{theorem}
    Every finite field has $p^n$ elements for some prime $p$ and $n\in\NN$.
\end{theorem}

\subsection{Field extensions}

\begin{definition}[Field extension]
    A \textbf{field extension} $F$ of $k$ is a $k$-vector space.
\end{definition}

\begin{proposition}
    All homomorphisms between fields and rings are injective.
    \begin{proof}
        The only possible maps between fields are field extensions, the only proper ideal of a field is the zero ideal.
    \end{proof}
\end{proposition}

\begin{definition}[Finite field extension]
    An extension of the fields $k\subset K$ is \textbf{finite} iff $K$ is a finite dimensional vector space over $k$ with $\dim K$ the \textbf{degree} of the extension
\end{definition}

\begin{theorem}
    If $k\subset F\subset K$ are field extensions, $K$ is a finite extension of $k$ iff $K$ is a finite extension of $F$ and $F$ is a finite extension of $k$. We then have $[K:k]=[K:F][F:k]$.
\end{theorem}

\begin{remark}
    Degree $2$ and $3$ field extensions are called quadratics and cubics respectively.
\end{remark}

\subsection{Constructing fields}

\begin{lemma}
    Given $R$ a PID with $a\neq0\in R$, $aR$ is maximal iff $a$ is irreducible.
    \begin{proof}
        
    \end{proof}
\end{lemma}

\begin{corollary}
    Given $R$ a PID with reducible $a\in R$, $R/aR$ is a field.
\end{corollary}

\begin{theorem}
    A polynomial $f(t)\in k[t]$ of degree $2$ or $3$ is irreducible iff it has no root in $k$.
\end{theorem}

\begin{definition}[Non-Square]
    $a\in k$ is non-square if there is no element $b\in k$ with $b^2=a$.
\end{definition}

\begin{lemma}
    Let $p$ be an odd prime. The field $\FF_p$ contins $(p-1)/2$ non-squares. For all non-square $a\in\FF_p$, $t^2-a$ is irreducible in $\FF_p[t]$.
\end{lemma}

\begin{theorem}
    For all $p(t)\in k[t]$, there exists a finite field extension $k\subset K$ such that: \[
        p(t)=c\prod_{i=1}^n(t-a_i)
        \black{,}
    \] for some $c\in k^\times$ and $a_i\in K$ for all $i\in[1,n]$.
\end{theorem}

\subsection{Existence of finite fields}

\begin{theorem}
    Let $k$ have characteristic $p\neq 0$, for all $x,y\in k$ and $m\in\ZZ^{\geq 0}$, \[
        (x+y)^{p^m}=x^{p^m}+y^{p^m}
        \black{.}
    \]
\end{theorem}

\vspace{-30pt}

\begin{definition}[Derivative]
    Let $p(t)= a_0 + a_1t + \ldots + a_nt^n\in k[t]$, the \textbf{derivative} of $p(t)$ is \[
        p'(t):= a_1 + 2a_2t + \ldots + na_nt^{n-1}
        \black{.}
    \]
\end{definition}

\begin{lemma}
    Let $p(t)=(x-a_1)(x-a_2)\ldots(x-a_n)\in k[t]$, $a_i\neq a_j$ for all $i\neq j$ iff $p(t)$ and $p'(t)$ have no common roots.
\end{lemma}

\begin{theorem}
    For all prime $p$ and natural $n$, there exists a field with $p^n$ elements.
\end{theorem}

\end{document}