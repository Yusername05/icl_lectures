\documentclass[../Year2.tex]{subfiles}
\usepackage{import}
\usepackage{../style/header}

\begin{document}

\chapter*{Real Analysis and Topology}
\renewcommand*\thesection{\arabic{section}}
\lhead{MATH50001A Real Analysis and Topology}
Lectured by Someone \\ Typed by Yu Coughlin \\
Season Year

\section*{Introduction}

The following are complementary reading for the course.
\begin{itemize}
    \item G. Grimmett and D. J. A. Welsh, Probability: An Introduction, 1986
    \item J. K. Blitzstein and J. Hwang, Introduction to Probability, 2019
    \item D. F. Anderson et al, Introduction to Probability, 2018
    \item S. M. Ross, Introduction to Pro ability Models, 2014
    \item G. Grimmett and D. Stirzaker, Probability and Random Processes, 2001
    \item G. Grimmett and D. Stirzaker, One Thousand Exercises in Probability, 2009
\end{itemize}

\begin{notation*}
    Unbracketed superscripts are used to label the components of vectors, with unbracketed subscripts labellin different vectors.
\end{notation*}

\tableofcontents\pagebreak

\lecture{1}{Monday}{30/10/2023}


\section{Euclidean spaces}

\begin{definition}[$\RR^n$]
    The set $\RR^n=\{(x^1,x^2,\ldots,x^n):x^i\in\RR, \ \forall i\in[1,n]\}$ will be considered with the operations to make it a real vector space.
\end{definition}

\subsection{Euclidean norm}

\begin{definition}[Inner product]
    We will have the \textbf{inner product} on $\RR^n$ by $\abr{\cdot,\cdot}:\RR\times\RR\rightarrow\RR$ satisfying: 
    \[
        \abr{x,y} := \sum_{i=1}^n x^i y^i \black{,}
    \] with the \textbf{Euclidean norm} given by, \[
        ||\cdot||:\RR^n\rightarrow[0,\infty) \ \black{with } ||x||=\sqrt{\abr{x,x}}\black{.}
    \]
\end{definition}

\vspace{-20pt}

\begin{proposition}[Properties of the Euclidean norm]
    The Euclidean norm satisfies the following properties: \begin{enumerate}
        \item[(N1)] for all $x\in\RR^n$, $||x||\geq 0$ achieving equality iff $x=0$,
        \item[(N2)] for all $x\in\RR^n$ and $\lambda\in\RR$, $||\lambda x|| = |\lambda|\cdot||x||$,
        \item[(N3)] for all $x,y\in\RR^n$: $||x+y||\leq||x||+||y||$,
    \end{enumerate}
\end{proposition}

\begin{theorem}[Cauchy-Swartz innequality]
    For all $x,y\in\RR^n$, $\aabr{x,y}\leq||x||\cdot||y||$.
\end{theorem}

\begin{theorem}[Reverse triangle innequality]
    For all $x,y\in\RR^n$, $\big|\ ||x||-||y||\ \big|\leq||x-y||$.
\end{theorem}

\begin{proposition}
    For $x=(x^1,x^2,\ldots,x^n)\in\RR^n$,\[
        \max_{k\in[1,n]}\left|x^k\right|\leq||x||\leq\sqrt{n}\max_{k\in[1,n]}\left|x^k\right|\black{.}
    \]
    \vspace{-20pt}
    \begin{proof}
        Exercise
    \end{proof}
\end{proposition}

\subsection{Convergence in \texorpdfstring{$\RR^n$}{Rn}}

\begin{definition}[Open ball]
    In $\RR^n$ we define the \textbf{open ball} around $x\in\RR^n$ of size $r\in\RR$ as \[
        B_r(x):=\{y\in\RR^n:||x-y||<r\}\black{.}
    \] This will be analoguous the the notion of open intervals used throughout analysis 1.
\end{definition}

\begin{definition}[Sequence in \texorpdfstring{$\RR^n$}{Rn}]
    A \textbf{sequence} in $\RR^n$ is an ordered list $x_0,x_1,\ldots,x_i\ldots$ with $x_i\in\RR^n$ for all $i\in\NN$, written ${(x_i)}_{i=0}^\infty$
\end{definition}

\begin{definition}[Convergence in \texorpdfstring{$\RR^n$}{Rn}]
    We say a sequence in $\RR^n$,  ${(x_i)}_{i=0}^\infty$ \textbf{converges to} $x\in\RR^n$ iff \[
        \forall\epsilon>0 \ \exists N\in\NN \ \black{such that, } \forall n\geq N, \ ||x_i-x||<\epsilon
    \] and we write $x_i\rightarrow x$ as $i\rightarrow\infty$ or $\lim\limits_{i\rightarrow\infty}x_i=x$.
\end{definition}

\begin{lemma}
    The sequence of vectors in $\RR^n$, ${(x_i)}_{i=0}^\infty$, converges to some $x=(x^1,x^2,\ldots,x^n)\in\RR^n$ iff each componenet of $x_i$ converges to the corresponding component in $x$: \[
        \forall k\in[1,n] \ \lim_{i\rightarrow\infty}x_i^k=x^k\black{.}
    \]
    \vspace{-20pt}
    \begin{proof}
        ($\implies$) Given $\displaystyle \lim\limits_{i\rightarrow\infty}x_i^k=x^k$ for all $k\in[1,n]$ we have that for all $\epsilon>0$, $\displaystyle\left|x_i^k-x^k\right|<\frac{\epsilon}{\sqrt{n}}$ for all $i\geq N_k$ for each $k\in[1,n]$ respectively. We take $N=\max\limits_{k\in[1,n]}N_k$ and now have: \vspace{-10pt}\[
            \max_{k\in[1,n]}\left|x_i^k-x^k\right|<\frac{\epsilon}{\sqrt{n}} \implies ||x_i-x||\leq\sqrt{n}\max_{k\in[1,n]}\left|x_i^k-x^k\right|<\epsilon\black{.}
        \]($\impliedby$)
    Similarly, given $\lim\limits_{i\rightarrow\infty}x_i=x \implies ||x_i-x||<\epsilon$ for all $\epsilon>0$: \[
        \left|x_i^k-x^k\right|\leq\max_{k\in[1,n]}\left|x_i^k-x^k\right|\leq||x_i-x||<\epsilon\black{,}
    \] therefore $\displaystyle \lim\limits_{i\rightarrow\infty}x_i^k=x^k$ for all $k\in[1,n]$.
    \end{proof}
\end{lemma}

\section{Continuity and limits of functions}

\subsection{Open sets}

\begin{definition}[Open set in \texorpdfstring{$\RR^n$}{Rn}]
    A subset $U\subseteq\RR^n$ is \textbf{open} in $\RR^n$ iff: \[
        \forall x\in U, \ \exists r>0 \ \black{such that } B_r(x)\subseteq U\black{.}
    \]
\end{definition}

\vspace{-40pt}

\subsection{Continuity}

\begin{definition}[Continuity]
    Let $A\subseteq \RR^n$ the we have $f:A\rightarrow \RR^m$ \textbf{continuous at} some $p\in A$ iff \[
        \forall\epsilon>0, \ \exists\delta>0 \ \black{such that } \forall x\in A \ \black{with } ||x-p||<\delta, \ ||f(x)-f(p)||<\epsilon\black{.} 
    \] If $f$ is continuous at all $p\in A$ we say $f$ is \textbf{continuous on} $A$.
\end{definition}

\begin{theorem}
    Let $A\subseteq\RR^n$ and $B\subseteq\RR^m$ with $f:A\rightarrow B$ continuous at $p\in A$. Supporse $g:B\rightarrow\RR^l$ is continuous as $f(p)$, then $g\circ f:A\rightarrow\RR^l$ is cotninuous at $p$.
    \begin{proof}
        Given any $\epsilon>0$ have $||x-p||<\delta_f\circ\delta_g(\epsilon)\implies||f(x)-f(p)||<\delta_g(\epsilon)\implies||g\circ f(x)-g\circ f(p)||<\epsilon$.
    \end{proof}
\end{theorem}

\section{Derivative of maps of Euclidean spaces}

\subsection{Total derivatives}

\begin{definition}[Total derivate]
    Given open $\Omega\subset\RR^n$, the function $f:\Omega\rightarrow \RR^m$ is \textbf{differentiable as} $p\in\Omega$ iff there is a linear linear map $\Lambda:\RR^n\rightarrow\RR^m$ satisfying: \[
        \lim_{x\rightarrow p} \frac{||f(x)-f(p)-\Lambda(x-p)||}{||x-p||}=0\black{.}
    \] Have $Df(p):=\Lambda$ be the \textbf{total derivative} of $f$ at $p$.
\end{definition}

\begin{remark}
    Given $f:(a,b)\rightarrow\RR$ differentiable at $p\in(a,b)$, we have 
    \[
            \lim_{x\rightarrow p} \frac{||f(x)-f(p)-\Lambda(x-p)||}{||x-p||}
            = \lim_{x\rightarrow p} \frac{|f(x)-f(p)-\lambda\cdot(x-p)|}{|x-p|}
            = \lim_{x\rightarrow p} \abs{\frac{f(x)-f(p)}{x-p}-\lambda}=0
    \] \vspace{-10pt}
    \[
        \implies \lim_{x\rightarrow p} \abs{\frac{f(x)-f(p)}{x-p}}=\lambda 
        \black{, which satisfies the normal definition for a derivative.}
    \]
\end{remark}

\begin{theorem}[Uniqueness of total derivatve]
    If the total derivative of a function $f:\Omega\subseteq\RR^n\rightarrow\RR^m$ exists, then it is unique.
    \begin{proof}
        
    \end{proof}
\end{theorem}

\begin{theorem}[Chain rule]
    Let $\Omega\subset\RR^n$, $\Omega'\subset\RR^m$ be open and have $g:\Omega\rightarrow\Omega'$, $f:\Omega'\rightarrow\RR^l$ differentiable at $p,g(p)$ respectively and let $h:=f\circ g$, $Dh(p) = Df(g(p))\circ Dg(p)$.
    \begin{proof}
        
    \end{proof}
\end{theorem}

\subsection{Directional and partial derivatives}

\begin{definition}[Direction derivative]
    Suppose $\Omega\subseteq\RR^n$ is open with $f:\Omega\rightarrow\RR^m$ differentiable at $p\in\Omega$. For all $v\in\RR^n$ the \textbf{directional derivative} of $f$ at $p$ in the direction of $v$ is: \[
        \frac{\partial f}{\partial v}(p) := \lim_{t\rightarrow 0} \frac{f(p+tv)-f(p)}{t} = Df(p)[v]\black{.}
    \] With the partial derivatives of $f$ given by: \[
        D_i f(p):=\frac{\partial f}{\partial e_i}(p) \black{, for all } i\in[1,n]\black{.}
    \]
\end{definition}

\begin{remark}
    If the total derivative of a function exists, then so do all of its directional derivatives.
\end{remark}

\begin{theorem}
    If $\Omega\subset\RR^n$ is open with $f:\Omega\rightarrow\RR$ with all partial derivatives existing for all $x\in\Omega$. If the map $x\mapsto D_i f(x)$ is continuous at $p\in\Omega$ for all partial derivatives, then $f$ is differentiable at $p$.
    \begin{proof}
        
    \end{proof}
\end{theorem}

\subsection{Higher order derivatives}

\begin{definition}[Second order partial derivatives]
    Let $\Omega\subset\RR^n$ be open with differentiable $f:\Omega\rightarrow\RR$ written as $(f^1,f^2,\ldots,f^n)^\T$, the \textbf{$ik$th second partial derivative} at $p$ is \[
        D_k D_i f^j(p) := \lim_{t\rightarrow 0}\frac{D_i f^j(p+te_k)-D_i f^j(p)}{t} \black{.}
    \] This can naturally be extended to $n$th order partial derivatives.
\end{definition}

\begin{theorem}
    Given open $\Omega\subseteq\RR^n$ and $f:\Omega\rightarrow\RR^m$ differentiable on $\Omega$, consider the map: \[
        \function[Df]{\Omega}{\LLL(\RR^n,\RR^m)\cong M_{n\times m}(\RR)\cong \RR^{m\times n}}{p}{Df(p)}\black{,}
    \] which we can now show to be continuous or differentiable at $p\in\Omega$, when differentiable we can take $DDf(p)\in\LLL(\RR^n,\RR^m)$. The componenents of the corresponding matrix are give by: \[
        [DDf(p)[h]]_{ij} = \sum_{k=1}^n D_k D_i f^j(p)h^k\black{.}
    \]
    \vspace{-10pt}
    \begin{proof}
        
    \end{proof}
\end{theorem}

\begin{remark}
    The condition of a function being $k$ times differentiable at a point $p$ can is often difficult to establish, instead the continuous existence of all $k-th$ partial derivatives in a neighbourhood of $p$ is a prefereable question which implies the former statement.
\end{remark}

\begin{theorem}[Schwartz's theorem]
    Suppose $\Omega\subseteq\RR^n$ is open and $f:\Omega\rightarrow\RR^m$ is differentiable on $\Omega$ with $D_i D_j f(p),D_j D_i f(p)$ both exist continuous only $\Omega$; then we have \[
        D_i D_j f(p) = D_j D_i f(p) \ \black{for all } p\in\Omega \black{.}
    \]
    \begin{proof}
        
    \end{proof}
\end{theorem}

\begin{notation}
    We need the following necessary notation around an $n$-vector of non-negative integers, $\alpha=(\alpha_1,\alpha_2,\ldots,\alpha_n)\in{(\ZZ_{>0})}^n$ for some $n\in\ZZ_{>0}$, to easily express Taylor's theorem in multiple dimensions: \begin{enumerate}
        \item $|\alpha| = \alpha_1+\alpha_2+\ldots+\alpha_n$,
        \item $D^\alpha f={(D_1)}^{\alpha_1}{(D_2)}^{\alpha_2}\cdots{(D_n)}^{\alpha_n}$,
        \item for some vector $h=(h^1,h^2,\ldots,h^n)\in\RR^n$, $h^\alpha = ({(h^1)}^{\alpha_1},{(h^2)}^{\alpha_2},\ldots,{(h^n)}^{\alpha_n})$,
        \item $\alpha! = \alpha_1!\alpha_2!\cdots\alpha_n!$.
    \end{enumerate}
\end{notation}

\begin{theorem}[Taylor's theorem]
    Given $p\in\RR^n$ with $f:B_r(p)\rightarrow\RR$, for some $r>0$, $k$-times continuous differentiable on $B_r(p)$ and some $||h||<r$; we have: \[
        f(p+h)=\sum_{|\alpha|\leq k-1}\frac{h^\alpha}{\alpha!}D^\alpha f(p) + R_k(p,h)\black{.}
    \] Where the remainder term, $R_k(p,h)$ is given by: \[
        R_k(p,h) = \sum_{|\alpha|=k}\frac{h^\alpha}{\alpha!}D^\alpha f(x)\black{.}
    \]
    \vspace{-10pt}
    \begin{proof}
        
    \end{proof}
\end{theorem}

\section{Inverse and implicit function theorems}

\subsection{Inverse function theorem}

\begin{theorem}[Inverse function theorem]\label{inv}
    Have $f:\RR^n\rightarrow \RR^n$ continuous differentiable on $\Omega\subseteq\RR^n$ and $Df(p)$ be invertible for a $p\in\Omega$. There exists open sets $U\in\Omega$ and $V\in\RR^n$ such that $f:U\rightarrow V$ is a bijection. Furthermore, $f^{-1}:V\rightarrow U$ is continuous differentiable on $V$ with: \[
        Df^{-1}(y)={\left[Df(f^{-1}(y))\right]}^{-1}\black{.}
    \]
    \vspace{-20pt}
\end{theorem}

\begin{lemma}
    Have $B_r(p)\subset\RR^n$ with $f:B_r(p)\rightarrow\RR^n$ contoniously differntiable. If there exists some $M\in\RR_{>0}$ with $|D_j f^i(x)|< M$ for all $x\in B_r(p)$ then \[
        ||f(x)-f(y)||\leq nM||x-y||\black{, for all } x,y\in B_r(p)\black{.}
    \]
    \begin{proof}
        
    \end{proof}
\end{lemma}

\begin{lemma}
    Given $f:\RR^n\rightarrow\RR^n$ continuous differentiable on some $B_r(p)$ with $Df(p)$ invertible, there exists some $\delta>0$ such that $f:B_\delta(p)\rightarrow\RR^n$ is injective.
    \begin{proof}
        
    \end{proof}
\end{lemma}

\begin{lemma}\label{inv1}
    $f:\RR^n\rightarrow\RR^n$ continuous differentiable on some $B_r(p)$ with $Df(p)$ invertible and $f:B_\delta(p)\rightarrow\RR^n$ injective, there exists some $\kappa>0$ with all $y\in B_\kappa(f(p))$ having a uniqe $x=B_\delta(p)$ such that $f(x)=y$.
    \begin{proof}
        
    \end{proof}
\end{lemma}

\begin{lemma}
    \begin{proof}
        
    \end{proof}
\end{lemma}

\begin{proof}[Proof of Theorem~\ref{inv}(Inverse function theorem)]
    By Lemma~\ref{inv1} 
\end{proof}

\subsection{Implicit function theorem}

\begin{theorem}[Implicit function theorem]
    Given $\Omega\subseteq\RR^n$ and $\Omega'\subseteq\RR^m$ both open with $f:\Omega\times\Omega'\rightarrow\RR^m$ continuous differentiable on $\Omega\times\Omega'$. If there is some $p\in\Omega\times\Omega'$ with $f(p)=0$ and $D_{n+j}f^i(p)$ invertible for $1\leq i, j\leq m$. Then, there are open sets $A\in\Omega$ and $B\in\Omega'$ containg $a$ and $b$ respectively such that for all $x\in A$ ther is a unique and differentiable $g(x)\in B$ with $f(x,g(x))=0$.
    \begin{proof}
        
    \end{proof}
\end{theorem}

\section{Metric spaces}

\subsection{Introduction}

\begin{definition}[Metric]
    A \textbf{metric} on some arbitrary set $X$ is a function: \[
        d:X\times X\rightarrow\RR
    \] that satisfies the following properties for all $x,y,z\in X$: \begin{enumerate}
        \item[(M1)] $d(x,y)\geq 0$ with $d(x,y)=0$ iff $x=y$ (positibity),
        \item[(M2)] $d(x,y)=d(y,x)$ (symmetry),
        \item[(M3)] $d(x,y) \leq d(x,z) + d(z,y)$ (triangle innequality).
    \end{enumerate}
\end{definition}

\begin{definition}[Metric space]
    A \textbf{metric space} is a pair consisting of a set and a metric on said set, often denoted $M=(X,d)$. The elements of $X$ are called \textbf{points} and for any two points of $M$, $x,y$, their \textbf{distance (with respect to $d$)} is $d(x,y)$.
\end{definition}

\begin{examples}
    The following are common examples of metric spaces: \begin{enumerate}
        \item have $X=\RR$ and $d_1:\RR\times\RR\rightarrow\RR$ by $d_1(x,y):=|x-y|$,
        \item have $X=\RR^n$ and have $\displaystyle d(x,y):=\sqrt{\sum_{i=1}^n{(x^i-y^i)}^2}$,
        \item for an arbitary non-empty set $X$ we have $d_\text{disc}:X\times X\rightarrow\RR$ by $d_\text{disc}(x,y):=0$ iff $x=y$ and $1$ otherwise (discrete metric),
        \item have $X$ be the set of bounded real sequences, then we can have $d_\infty:\RR\times\RR\rightarrow\RR$ given by $d_\infty(x,y) := \sup\limits_{k\geq1}|x^k-y^k|$,
        \vspace{-10pt}
        \item let $X$ be the set of continuous real functions on $[a,b]$ with $\displaystyle d(f,g):=\intd{t=a}{b}{|f(t)-g(t)|}{t}$.
    \end{enumerate}
\end{examples}

\begin{definition}[Induced metric]
    Given the metric space $(X,d)$ and some $Y\subset X$, we have $d_Y:Y\times Y\rightarrow\RR$ with $d_Y(x,y)=d(x,y)$ for all $x,y\in Y$ as the \textbf{induced metric} on $Y$. $(Y,d_Y)$ is a \textbf{metric subspace} of $(X,d)$.
\end{definition}

\subsection{Normed vector spaces}

\begin{definition}[Normed vector spaces]
    Given a real-vector space $V$, a function $||\cdots||:V\rightarrow\RR$ is a \textbf{norm} on $V$ iff the following hold for all $u,v\in V$:\begin{enumerate}
        \item[(N1)] $||v||\geq 0$ with $||v||=0$ iff $v=0_V$,
        \item[(N2)] for all $\lambda\in\RR$, $||\lambda v|| = |\lambda|\cdot||v||$,
        \item[(N3)] $||u+v||\leq||u||+||v||$.
    \end{enumerate} A vector space together with a norm is a \textbf{normed vector space}.
\end{definition}

\begin{lemma}
    If $(V,||\cdot||)$ is a normed vector space, $d_{||\cdot||}:V\times V\rightarrow\RR$ with $d_{||\cdot||}(u,v) = ||u-v||$ is a metric on $V$.
    \begin{proof}
        
    \end{proof}
\end{lemma}

\subsection{Open and closed sets}

\begin{definition}[$\epsilon$-ball]
    Given a point $x$ in the metric space $(X,d)$ and a real $\epsilon>0$, the \textbf{ball} of radius $\epsilon$ centred at $x$ is the set, \[
        B_\epsilon(x) := \{y\in X: d(x,y)<\epsilon\}\black{,}
    \] which is sometimes referred to as a neighbourhood of $x$.
\end{definition}

\begin{definition}[Open sets]
    Given metric space $(X,d)$ a set $U\subseteq X$ is \textbf{open} in $(X,d)$ iff, for all $\u\in U$ there exists some $\delta>0$ such that $B_\delta(u)\subseteq U$.
\end{definition}

\begin{proposition}
    Have $\XXX=(X,d)$ a metric space, the follow hold true: \begin{enumerate}
        \item $\emptyset$ and $\XXX$ are open in $\XXX$,
        \item for all $x\in\XXX$ and $\epsilon>0$, $B_\epsilon(x)$ is open in $\XXX$,
        \item the union of (up to uncountably many) open sets in $\XXX$ are open in $\XXX$,
        \item the intersection of finitely many open sets in $\XXX$ is open in $\XXX$.
    \end{enumerate}
    \begin{proof}
        
    \end{proof}
\end{proposition}

\begin{definition}[Topological equivalence]
    Two metrics $d,d'$ on $X$ are \textbf{topologically equivalent} iff $U\subseteq X$ is open in $(X,d)$ iff it is also open in $(X,d')$.
\end{definition}

\begin{definition}[Closed sets]
    Given the metric space $(X,d)$ with $U\subseteq X$, $U$ is \textbf{closed} iff $X\setminus U$ is open.
\end{definition}

\begin{proposition}
    A set $U\subseteq X$ with $(X,d)$ a metric spacpe is closed iff, every convergenct sequence in $V$ has a limit in $V$.
    \begin{proof}
        
    \end{proof}
\end{proposition}

\begin{proposition}
    The intersection of (up to ocuntable many) closed sets in a metric space is closed; the union of finitely many sets in a metric space is closed.
    \begin{proof}
        
    \end{proof}
\end{proposition}

\subsection{Separable space}

\begin{definition}[Interior, isolated, limits and boundary points]
    We will have $(X,d)$ be a metric space with $V\subseteq X$ and $x\in X$: \begin{itemize}
        \item $x$ is an \textbf{interor point} of $V$ if there is some $\delta>0$ with $B_\delta(x)\subseteq V$,
        \item $x$ is an \textbf{isolated point} of $V$ if there is some $\delta>0$ such that $V\cap B_\delta(x)=\{x\}$,
        \item $x$ is a \textbf{limit point} of $V$ if for all $\delta>0$, we have $(B_\delta(x)\cap V)\setminus\{x\}\neq\emptyset$, 
        \item $x$ is a \textbf{boundary point} of $V$ if it is  a limit point, under the previous definition, and $B_\delta(x)\setminus V\neq \emptyset$.
    \end{itemize}
\end{definition}

\begin{remark}
    Interior and isolated points are necessarily in $V$, but limit points and boundary points need not be elements of $V$.
\end{remark}

\begin{definition}[Interior, closure and boundary]
    Once again, we will have $(X,d)$ a metric space with $V\subseteq X$: \begin{itemize}
        \item the \textbf{interior} of $V$ is the set of all $v\in V$ with $v$ an interior point of $V$, denoted $V^\circ$,
        \item the \textbf{closure} of $V$ is the union of $V$ with the set of limit points of $V$, denoted $\overline{V}$,
        \item the \textbf{boundary} of $V$ is the set of boundary points of $V$, denoted $\partial V$.
    \end{itemize}
\end{definition}

\begin{proposition}
    $\partial V=\overline{V}\setminus V^\circ$.
    \begin{proof}
        
    \end{proof}
\end{proposition}

\begin{definition}[Dense set]
    Have $(X,d)$ a metric space, $V\subseteq X$ is \textbf{dense} in $(X,d)$ iff $\overline{V}=X$.
\end{definition}

\begin{definition}[Separable space]
    We say the metric space $(X,d)$ is \textbf{separable} if there is a countable, dense set in $X$.
\end{definition}

\section{Continuous maps in metric spaces}

\subsection{Convergence}

\begin{definition}[Convergence in metric spaces]
    Let ${(x_n)}_{n\geq1}$ be a sequence in the metric space $(X,d)$. We say ${(x_n)}_{n\geq 1}$ \textbf{converges} in $(X,d)$ iff: \[
        \exists x\in X \ \black{such that, } \forall\epsilon>0, \ \exists N\in\ZZ_{>0} \ \black{with } d(x_n,x)<\epsilon \ \black{for all } n\geq N\black{.}
    \] And we say ${(x_n)}_{x\geq 1}$ converges to $x$ in $(X,d)$, or any other equivalent phrasing from analysis.
\end{definition}

\begin{definition}[Cauchy sequences]
    A sequence ${(x_n)}_{n\geq 1}$ is \textbf{Cauchy} in $(X,d)$ iff \[
        \forall\epsilon>0, \ \exists N\in\ZZ_{>0} \ \black{such that } \forall n,m\geq N,\  d(x_n,x_m)<\epsilon\black{.}
    \]
    \vspace{-20pt}
\end{definition}

\begin{lemma}[Uniqueness of limits]
    If the sequence ${(x_n)}_{n\geq 1}$ converges to some $x$ in the metric space $(X,d)$ then this limit is unique.
    \begin{proof}
        
    \end{proof}
\end{lemma}

\begin{theorem}
    Given two topologically equivalent metrics $d,d'$ on $X$, the sequence ${(x_n)}_{n\geq 1}$ converges in $(X,d)$ iff it also converges in $(X,d')$.
    \begin{proof}
        
    \end{proof}
\end{theorem}

\subsection{Continuity of maps}

\begin{definition}[Continuous map]
    Given the metric spaces $(X,d_X), (Y,d_Y)$ and $f:X\rightarrow Y$: \begin{enumerate}
        \item $f$ is \textbf{continuous at} $x\in X$ iff for all $x'\in X$: \[
            \forall\epsilon>0, \exists\delta>0 \ \black{such that } d_X(x,x')<\delta \implies d_X(f(x),f(x'))<\epsilon \black{,}
        \] \vspace{-20pt}
        \item $f$ is \textbf{continuous on} $U\subseteq X$ if $f$ is continuous at every $u\in U$,
        \item $f$ is \textbf{uniformly continuous} on $U\subseteq X$ is $f$ is continuous on $U$ and $\delta=\delta(\epsilon)$ does not depend on $x$.
    \end{enumerate}
\end{definition}

\begin{theorem}
    Let $(X,d_X), (Y,d_Y)$ be metric spaces, a function $f:X\rightarrow Y$ is continuous iff the pre-image of any open $U\subseteq Y$ is open in $X$.
    \begin{proof}
        
    \end{proof}
\end{theorem}

\begin{proposition}
    If, similarly, $(X,d_X), (Y,d_Y)$ are metric spaces with $f:X\rightarrow Y$, the following are equivalent: \begin{enumerate}
        \item $f$ is continuous at $x\in X$,
        \item if a sequence ${(x_n)}_{n\geq 1}$ converges to $x\in X$ then ${(f(x_n))}_{n\geq 1}$ converges to $f(x)\in Y$.
    \end{enumerate}
    \begin{proof}
        
    \end{proof}
\end{proposition}

\subsection{Metric homeomorphisms}

\begin{definition}[Homeomorphism]
    Have $(X,d_X),(Y,d_Y)$ be metric spaces, a mapping $f:X\rightarrow Y$ is a \textbf{homeomorphism} if it is a bijection with $f,f^{-1}$ both continuous. Metric spaces with homeomorphisms between then are \textbf{homeomorphic}.
\end{definition}

\begin{definition}[Lipschitz]
    Given metric spaces $(X,d_X),(Y,d_Y)$ and $f:X\rightarrow Y$ we say:\begin{enumerate}
        \item  $f$ is \textbf{Lipschitz} if there is some $M>0$ with: 
        \[ d_Y(f(x_1),f(x_2))\leq M\cdot d_X(x_1,x_2) \ \black{for all } x_1,x_2\in X\black{,}\]
        \vspace{-20pt}
        \item  $f$ is \textbf{bi-Lipschitz} if there is some $M_1,M_2>0$ with: 
        \[ M_1\cdot d_X(x_1,x_2) \leq d_Y(f(x_1),f(x_2))\leq M_2\cdot d_X(x_1,x_2) \ \black{for all } x_1,x_2\in X\black{,}\]
        \vspace{-20pt}
        \item  $f$ is \textbf{isometric} if, 
        \[ d_Y(f(x_1),f(x_2)) =  d_X(x_1,x_2) \ \black{for all } x_1,x_2\in X\black{.}\]
        \vspace{-20pt}
    \end{enumerate}
\end{definition}

\begin{remark}
    An isometry between metric spaces is a bi-Lipschitz map with two unit constants.
\end{remark}

\section{Topological spaces}

\subsection{Topologies and their spaces}

\begin{definition}[Topology]
    Given a non-emtpy set $X$, we say $\tau$, a collection of subsets of $X$, is a \textbf{topology} on $X$ if it satisfies the following conditions: \begin{enumerate}
        \item[(T1)] $\emptyset,X\subseteq\tau$,
        \item[(T2)] if $X_i\in\tau$ for all $i$ in a indexing set $\III$, $\displaystyle \bigcup_{i\in\III}X_i\in\tau$,
        \vspace{-10pt}
        \item[(T3)] if $X_1,X_2,\ldots,X_n\in\tau$, $\displaystyle \bigcap_{i=1}^m X_i\in\tau$.
    \end{enumerate}
    The pair $(X,\tau)$ is called a \textbf{topological space} with elements of $X$ called \textbf{points} and elements of $\tau$ called open sets. If $x\in X$ and $x\in U\in\tau$, $U$ is a neighbourhood of $x$.
\end{definition}

\begin{examples}
    These are some common examples of topological spaces: \begin{enumerate}
        \item for any set $X$ have $\tau=\{\emptyset, X\}$, the trivial topology on $X$,
        \item instead have $\tau$ be the collection of subsets of $X$, the discrete topology on $X$,
        \item if $(X,d)$ is a metric space, $\tau:=\{U\subseteq X: U \ \black{is open in } (X,d)\}$ the metric topology on $X$,
        \item for a non-empty set $X$, $\tau=\{\emptyset,V,X\}$ for some non-empty $V\subset X$,
        \item if $X=\{a,b\}$ and $\tau=\{\emptyset,\{a,b\},\{b\}\}$ is the smallest toplogical space that is neither trivial nor discrete (called the Sierpinski topology).
    \end{enumerate}
\end{examples}

\begin{definition}[Metrisability]
    A topological space $(X,\tau)$ is \textbf{metrisable} iff it is the topology induced by some metric.
\end{definition}

\begin{definition}[Coarser and finer topologies]
    Given two topologies $\tau_1,\tau_2$ both on $X$, we say $\tau_1$ is \textbf{coarser} than $\tau_2$, and equivalently $\tau_2$ is \textbf{finer} than $\tau_1$, iff $\tau_2\subseteq\tau_1$.
\end{definition}

\subsection{Bases}

\begin{definition}[Basis]
    Given a topological space $(X,\tau)$ we call a subfamily $B\subseteq\tau$ a \textbf{basis} for $\tau$ iff every open set in $\tau$ is the union of open sets in $B$.
\end{definition}

\subsection{Closed sets}

\begin{definition}[Closed sets]
    Given a topological space $(X,\tau)$, we say $V\subseteq X$ is \textbf{closed} iff $X\setminus V$ is open.
\end{definition}

\begin{proposition}
    Closed sets in any given topological space $(X,\tau)$ satisfy the followoing: \begin{enumerate}
        \item[(C1)] $X,\emptyset$ are closed,
        \item[(C2)] if $C_1,C_2$ are closed, $C_1\cup C_2$ is closed,
        \item[(C3)] the (up to uncountable) intersection of closed sets is closed.
    \end{enumerate}
    \begin{proof}
        
    \end{proof}
\end{proposition}

\begin{definition}[Closure]
    Given an open set $U$ in the topological space $(X,\tau)$ the \textbf{closure} of $U$ in $(X_\tau)$ is given by: \[
        \overline{U}:=\mathop{\bigcap_{V\subseteq X}}_{V  \black{closed}, A\subseteq V}V
    \black{.}
        \]
\end{definition}

\begin{definition}[Point of closure]
    Given the topological space $\XXX$ with $A\subseteq\XXX$, $x\in\XXX$ is a \textbf{point of closure} of $A$ iff every open set $U$ with $x\in U$ has $U\cap A=\emptyset$.
\end{definition}

\begin{proposition}
    $\overline{A}=\{x\in X: x \ \black{is a point of closure for} A\}$.
    \begin{proof}
        
    \end{proof}
\end{proposition}

\subsection{Convergence and Hausdorff property}

\begin{definition}[Convergence]
    For a sequences ${(x_n)}_{n\geq 1}$ in a topological space $(X,\tau)$ we say ${(x_n)}_{n\geq 1}$ \textbf{converges} (in $(X,\tau)$) to $x\in X$ iff \[
        \forall T\in\tau \ \black{with } x\in T, \ \exists N\in\ZZ_{>0} \ \black{such that } \forall n\geq N, \ x_n\in T\black{.}
    \]\vspace{-20pt}
\end{definition}

\begin{definition}[Hausdorff]
    A topological space $(X,\tau)$ is \textbf{Hausdorff} iff for all $x,y\in X$ with $x\neq y$ there are open sets $U,V$ containing $x,y$ respectively with $U\cap V=\emptyset$. With $U$ and $V$ \textbf{separating} $x$ and $y$.
\end{definition}

\begin{theorem}
    Limits of convergent sequences in Hausdorff spaces are unique.
    \begin{proof}
        
    \end{proof}
\end{theorem}

\begin{definition}[Regular spaces]
    A topological space $(X,\tau)$ is \textbf{regular} iff for every closed subset $C\subseteq X$ with point $p\not\in C$ there are open sets $U,V\in\tau$ such that $p\in U$, $C\subseteq V$ and $U\cap V=\emptyset$.
\end{definition}

\subsection{Continuous maps}

\begin{definition}[Continuous map]
    Given two topological spaces $(X,\tau_X),(Y,\tau_Y)$ the map $f:X\rightarrow Y$ is \textbf{continuous} iff $f^{-1}(U)\in\tau_X$ for all $U\in\tau_Y$.
\end{definition}

\begin{definition}[Continuity at points]
    The map $f:X\rightarrow Y$, with $(X,\tau_X),(Y,\tau_Y)$ topological spaces, is \textbf{continuous at} $x\in X$ iff $f^{-1}(U)\in\tau_X$ for all $U\in\tau_Y$ with $f(x)\in U$.
\end{definition}

\begin{definition}[Homeomorphism]
    A \textbf{homeomorphism} between topolgical spaces is a bijection map, $f$, where both $f$ and $f^{-1}$ are continuous. Spaces with homeomorphisms between them are \textbf{topologically equivalent}.
\end{definition}

\subsection{Subspaces}

\begin{definition}[Subspace]
    If $(X,\tau)$ is a topological space and $A\subseteq X$, the \textbf{subspace topology} on $A$ is $\tau_A=\{A\cap U: U\in\tau\}$, $(A,\tau_A)$ is a topological space called the \textbf{supspace} of $(X,\tau)$.
    \begin{proof}[Proof of topological space]
        
    \end{proof}
\end{definition}

\begin{proposition}[Universal property]
    Given topological spaces $(X,\tau_X), (Y,\tau_Y)$ with $A\subseteq X$ with its supspace topology and $g:Y\rightarrow A$, $g$ is continuous iff $i\circ g$ is continuous, where $i$ is the inclusion map, \[
        \begin{tikzcd}
            & X \\
            Y \arrow[ur, "i\circ g"] \arrow[r, swap, "g"] & A \arrow[u, swap, hook, "i"]
        \end{tikzcd}
        \black{.}
    \]
    \begin{proof}
        
    \end{proof}
\end{proposition}

\begin{theorem}
    Given the topological space $(X,\tau)$ and $A\subseteq X$, the subspace topology is the only topology such that for all $(Y,\tau_Y)$, $g:Y\rightarrow A$ is continuous iff $(i\circ g)$ is continuous.
    \begin{proof}
        
    \end{proof}
\end{theorem}

\begin{lemma}
    If $B$ is a basis for the topological space $(X,\tau)$ and $A\subseteq X$, $B_A:=\{U\cap A:U\in B\}$ is a basis for $\tau_A$.
    \begin{proof}
        
    \end{proof}
\end{lemma}

\begin{proposition}
    For a metric space $(X,d)$ with $A\subseteq X$, the two canonical topologies on $A$, $\tau_{d_A}$ and $\tau_A$ are equal. \begin{proof}
        
    \end{proof}
\end{proposition}

\section{Connectedness}

\subsection{Definition}

\subsection{Continuous maps}

\subsection{Path connected sets}

\section{Compactness}

\subsection{Covers}

\subsection{Sequential compactness}

\subsection{Continuous maps}

\subsection{Arzelá-Ascoli theorem}

\section{Completeness}

\subsection{Banach spaces}

\subsection{Fixed point theorem}

\end{document}