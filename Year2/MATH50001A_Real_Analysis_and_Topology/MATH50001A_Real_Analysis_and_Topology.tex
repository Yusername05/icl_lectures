\documentclass[../Year2.tex]{subfiles}
\usepackage{import}
\usepackage{../style/header}

\begin{document}

\chapter{Real Analysis and Topology}
\renewcommand*\thesection{\arabic{section}}
\lhead{MATH50001A Real Analysis and Topology}
Lectured by Someone \\ Typed by Yu Coughlin \\
Season Year

\section*{Introduction}

The following are complementary reading for the course.
\begin{itemize}
    \item G. Grimmett and D. J. A. Welsh, Probability: An Introduction, 1986
    \item J. K. Blitzstein and J. Hwang, Introduction to Probability, 2019
    \item D. F. Anderson et al, Introduction to Probability, 2018
    \item S. M. Ross, Introduction to Pro ability Models, 2014
    \item G. Grimmett and D. Stirzaker, Probability and Random Processes, 2001
    \item G. Grimmett and D. Stirzaker, One Thousand Exercises in Probability, 2009
\end{itemize}

\begin{notation*}
    Unbracketed superscripts are used to label the components of vectors, with unbracketed subscripts labellin different vectors.
\end{notation*}

\tableofcontents\pagebreak

\lecture{1}{Monday}{30/10/2023}

\section{Euclidean spaces}

\begin{definition}[$\RR^n$]
    The set $\RR^n=\{(x^1,x^2,\ldots,x^n):x^i\in\RR, \ \forall i\in[1,n]\}$ will be considered with the operations to make it a real vector space.
\end{definition}

\subsection{Euclidean norm}

\begin{definition}[Inner product]
    We will have the \textbf{inner product} on $\RR^n$ by $\abr{\cdot,\cdot}:\RR\times\RR\rightarrow\RR$ satisfying: 
    \[
        \abr{x,y} := \sum_{i=1}^n x^i y^i \black{,}
    \] with the \textbf{Euclidean norm} given by, \[
        ||\cdot||:\RR^n\rightarrow[0,\infty) \ \black{with } ||x||=\sqrt{\abr{x,x}}\black{.}
    \]
\end{definition}
\vspace{-30pt}

\begin{proposition}[Properties of the Euclidean norm]
    The Euclidean norm satisfies the following properties: \begin{enumerate}
        \item[(N1)] for all $x\in\RR^n$, $||x||\geq 0$ achieving equality iff $x=0$,
        \item[(N2)] for all $x\in\RR^n$ and $\lambda\in\RR$, $||\lambda x|| = |\lambda|\cdot||x||$,
        \item[(N3)] for all $x,y\in\RR^n$: $||x+y||\leq||x||+||y||$,
    \end{enumerate}
\end{proposition}

\begin{theorem}[Cauchy-Swartz innequality]
    For all $x,y\in\RR^n$, $\aabr{x,y}\leq||x||\cdot||y||$.
\end{theorem}

\begin{theorem}[Reverse triangle innequality]
    For all $x,y\in\RR^n$, $\big|\ ||x||-||y||\ \big|\leq||x-y||$.
\end{theorem}

\begin{proposition}
    For $x=(x^1,x^2,\ldots,x^n)\in\RR^n$,\[
        \max_{k\in[1,n]}\left|x^k\right|\leq||x||\leq\sqrt{n}\max_{k\in[1,n]}\left|x^k\right|\black{.}
    \]
    \vspace{-20pt}
    \begin{proof}
        Exercise
    \end{proof}
\end{proposition}

\subsection{Convergence in \texorpdfstring{$\RR^n$}{Rn}}

\begin{definition}[Open ball]
    In $\RR^n$ we define the \textbf{open ball} around $x\in\RR^n$ of size $r\in\RR$ as \[
        B_r(x):=\{y\in\RR^n:||x-y||<r\}\black{.}
    \] This will be analoguous the the notion of open intervals used throughout analysis 1.
\end{definition}

\begin{definition}[Sequence in \texorpdfstring{$\RR^n$}{Rn}]
    A \textbf{sequence} in $\RR^n$ is an ordered list $x_0,x_1,\ldots,x_i\ldots$ with $x_i\in\RR^n$ for all $i\in\NN$, written ${(x_i)}_{i=0}^\infty$
\end{definition}

\begin{definition}[Convergence in \texorpdfstring{$\RR^n$}{Rn}]
    We say a sequence in $\RR^n$,  ${(x_i)}_{i=0}^\infty$ \textbf{converges to} $x\in\RR^n$ iff \[
        \forall\epsilon>0 \ \exists N\in\NN \ \black{such that, } \forall n\geq N, \ ||x_i-x||<\epsilon
    \] and we write $x_i\rightarrow x$ as $i\rightarrow\infty$ or $\lim\limits_{i\rightarrow\infty}x_i=x$.
\end{definition}

\begin{lemma}
    The sequence of vectors in $\RR^n$, ${(x_i)}_{i=0}^\infty$, converges to some $x=(x^1,x^2,\ldots,x^n)\in\RR^n$ iff each componenet of $x_i$ converges to the corresponding component in $x$: \[
        \forall k\in[1,n] \ \lim_{i\rightarrow\infty}x_i^k=x^k\black{.}
    \]
    \vspace{-20pt}
    \begin{proof}
        ($\implies$) Given $\displaystyle \lim\limits_{i\rightarrow\infty}x_i^k=x^k$ for all $k\in[1,n]$ we have that for all $\epsilon>0$, $\displaystyle\left|x_i^k-x^k\right|<\frac{\epsilon}{\sqrt{n}}$ for all $i\geq N_k$ for each $k\in[1,n]$ respectively. We take $N=\max\limits_{k\in[1,n]}N_k$ and now have: \vspace{-10pt}\[
            \max_{k\in[1,n]}\left|x_i^k-x^k\right|<\frac{\epsilon}{\sqrt{n}} \implies ||x_i-x||\leq\sqrt{n}\max_{k\in[1,n]}\left|x_i^k-x^k\right|<\epsilon\black{.}
        \]($\impliedby$)
    Similarly, given $\lim\limits_{i\rightarrow\infty}x_i=x \implies ||x_i-x||<\epsilon$ for all $\epsilon>0$: \[
        \left|x_i^k-x^k\right|\leq\max_{k\in[1,n]}\left|x_i^k-x^k\right|\leq||x_i-x||<\epsilon\black{,}
    \] therefore $\displaystyle \lim\limits_{i\rightarrow\infty}x_i^k=x^k$ for all $k\in[1,n]$.
    \end{proof}
\end{lemma}

\section{Continuity and limits of functions}

\subsection{Open sets}

\begin{definition}[Open set in \texorpdfstring{$\RR^n$}{Rn}]
    A subset $U\subseteq\RR^n$ is \textbf{open} in $\RR^n$ iff: \[
        \forall x\in U, \ \exists r>0 \ \black{such that } B_r(x)\subseteq U\black{.}
    \]
\end{definition}

\vspace{-40pt}

\subsection{Continuity}

\begin{definition}[Continuity]
    Let $A\subseteq \RR^n$ the we have $f:A\rightarrow \RR^m$ \textbf{continuous at} some $p\in A$ iff \[
        \forall\epsilon>0, \ \exists\delta>0 \ \black{such that } \forall x\in A \ \black{with } ||x-p||<\delta, \ ||f(x)-f(p)||<\epsilon\black{.} 
    \] If $f$ is continuous at all $p\in A$ we say $f$ is \textbf{continuous on} $A$.
\end{definition}

\begin{theorem}
    Let $A\subseteq\RR^n$ and $B\subseteq\RR^m$ with $f:A\rightarrow B$ continuous at $p\in A$. Supporse $g:B\rightarrow\RR^l$ is continuous as $f(p)$, then $g\circ f:A\rightarrow\RR^l$ is cotninuous at $p$.
    \begin{proof}
        
    \end{proof}
\end{theorem}

\section{Derivative of maps of Euclidean spaces}

\subsection{Total derivatives}

\begin{definition}[Total derivate]
    Given open $\Omega\subset\RR^n$, the function $f:\Omega\rightarrow \RR^m$ is \textbf{differentiable as} $p\in\Omega$ iff there is a linear linear map $\Lambda:\RR^n\rightarrow\RR^m$ satisfying: \[
        \lim_{x\rightarrow p} \frac{||f(x)-f(p)-\Lambda(x-p)||}{||x-p||}=0\black{.}
    \] Have $Df(p):=\Lambda$ be the \textbf{total derivative} of $f$ at $p$.
\end{definition}

\begin{remark}
    Given $f:(a,b)\rightarrow\RR$ differentiable at $p\in(a,b)$, we have 
    \[
            \lim_{x\rightarrow p} \frac{||f(x)-f(p)-\Lambda(x-p)||}{||x-p||}
            = \lim_{x\rightarrow p} \frac{|f(x)-f(p)-\lambda\cdot(x-p)|}{|x-p|}
            = \lim_{x\rightarrow p} \abs{\frac{f(x)-f(p)}{x-p}-\lambda}=0
    \] \vspace{-10pt}
    \[
        \implies \lim_{x\rightarrow p} \abs{\frac{f(x)-f(p)}{x-p}}=\lambda 
        \black{, which satisfies the normal definition for a derivative.}
    \]
\end{remark}

\begin{theorem}[Uniqueness of total derivatve]
    If the total derivative of a function $f:\Omega\subseteq\RR^n\rightarrow\RR^m$ exists, then it is unique.
    \begin{proof}
        
    \end{proof}
\end{theorem}

\begin{theorem}[Chain rule]
    Let $\Omega\subset\RR^n$, $\Omega'\subset\RR^m$ be open and have $g:\Omega\rightarrow\Omega'$, $f:\Omega'\rightarrow\RR^l$ differentiable at $p,g(p)$ respectively and let $h:=f\circ g$, $Dh(p) = Df(g(p))\circ Dg(p)$.
    \begin{proof}
        
    \end{proof}
\end{theorem}

\subsection{Directional and partial derivatives}

\begin{definition}[Direction derivative]
    Suppose $\Omega\subseteq\RR^n$ is open with $f:\Omega\rightarrow\RR^m$ differentiable at $p\in\Omega$. For all $v\in\RR^n$ the \textbf{directional derivative} of $f$ at $p$ in the direction of $v$ is: \[
        \frac{\partial f}{\partial v}(p) := \lim_{t\rightarrow 0} \frac{f(p+tv)-f(p)}{t} = Df(p)[v]\black{.}
    \] With the partial derivatives of $f$ given by: \[
        D_i f(p):=\frac{\partial f}{\partial e_i}(p) \black{, for all } i\in[1,n]\black{.}
    \]
\end{definition}

\begin{remark}
    If the total derivative of a function exists, then so do all of its directional derivatives.
\end{remark}

\begin{theorem}
    If $\Omega\subset\RR^n$ is open with $f:\Omega\rightarrow\RR$ with all partial derivatives existing for all $x\in\Omega$. If the map $x\mapsto D_i f(x)$ is continuous at $p\in\Omega$ for all partial derivatives, then $f$ is differentiable at $p$.
\end{theorem}

\subsection{Higher order derivatives}

\section{Inverse and implicit function theorems}

\subsection{Inverse function theorem}

\subsection{Implicit function theorem}

\section{Metric spaces}

\subsection{Introduction}

\subsection{Normed vector spaces}

\subsection{Sets in metric spaces}

\subsection{Continuous maps of metric spaces}

\section{Topological spaces}

\subsection{Topologies and their spaces}

\subsection{Convergence and Hausdorff property}

\subsection{Closed sets}

\subsection{Continuous maps}

\section{Connectedness}

\subsection{Definition}

\subsection{Continuous maps}

\subsection{Path connected sets}

\section{Compactness}

\subsection{Covers}

\subsection{Sequential compactness}

\subsection{Continuous maps}

\subsection{Arzelá-Ascoli theorem}

\section{Completeness}

\subsection{Banach spaces}

\subsection{Fixed point theorem}

\end{document}