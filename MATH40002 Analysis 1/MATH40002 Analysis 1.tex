\def\module{MATH40002 Analysis 1}
\def\lecturer{Dr Ajay Chandra}
\def\term{Autumn 2023}
\def\cover{\vspace{1in}
$$
\begin{tikzcd}[ampersand replacement=\&, column sep=tiny]
\& \qquad \& \& \& L \arrow[dash, dashed]{dddd} \& \& \& \& \& \& \\
K\br{\alpha} \arrow[dash]{urrrr} \arrow[dash, dashed]{dddd} \& \& K\br{\alpha'} \arrow[dash]{urr} \arrow[dash, dashed]{dddd} \& \& \& K\br{\beta, \gamma} \arrow[dash]{ul} \arrow[dash, dashed]{dddd} \& \& \& K\br{\delta} \arrow[dash]{ullll} \arrow[dash, dashed]{dddd} \& \& K\br{\delta'} \arrow[dash]{ullllll} \arrow[dash, dashed]{dddd} \\
\& \& \& K\br{\beta} \arrow[crossing over, dash]{ulll} \arrow[dash]{ul} \arrow[crossing over, dash]{urr} \arrow[dash, dashed]{dddd} \& \& \& K\br{\beta\gamma} \arrow[crossing over, dash]{ul} \arrow[dash, dashed]{dddd} \& \& \& K\br{\gamma} \arrow[crossing over, dash]{ullll} \arrow[crossing over, dash]{ul} \arrow[crossing over, dash]{ur} \arrow[dash, dashed]{dddd} \& \\
\& \& \& \& \& \& \& K \arrow[crossing over, dash]{ullll} \arrow[crossing over, dash]{ul} \arrow[crossing over, dash]{urr} \arrow[dash, dashed]{dddd} \& \& \& \\
\& \qquad \& \& \& \abr{e} \& \& \& \& \& \& \\
\abr{\tau} \arrow[dash]{urrrr} \& \& \abr{\sigma^2\tau} \arrow[dash]{urr} \& \& \& \abr{\sigma^2} \arrow[dash]{ul} \& \& \& \abr{\sigma\tau} \arrow[dash]{ullll} \& \& \abr{\sigma^3\tau} \arrow[dash]{ullllll} \\
\& \& \& \abr{\sigma^2, \tau} \arrow[dash]{ulll} \arrow[dash]{ul} \arrow[dash]{urr} \& \& \& \abr{\sigma} \arrow[dash]{ul} \& \& \& \abr{\sigma^2, \sigma\tau} \arrow[dash]{ullll} \arrow[dash]{ul} \arrow[dash]{ur} \& \\
\& \& \& \& \& \& \& G \arrow[dash]{ullll} \arrow[dash]{ul} \arrow[dash]{urr} \& \& \&
\end{tikzcd}
$$
$$ G = \Gal\br{L / K} \cong \DDD_8 $$
}
\def\syllabus{Number systems, decimal expansions, sup and inf, sequences, series, convergence tests, power series, continuity, closure, compactness, uniform continuity, differentiation, integration}
\def\thm{subsection}

\documentclass{article}

% Packages

\usepackage{amssymb}
\usepackage{amsthm}
\usepackage[UKenglish]{babel}
\usepackage{commath}
\usepackage[inline]{enumitem}
\usepackage{etoolbox}
\usepackage{fancyhdr}
\usepackage[margin=1in]{geometry}
 \geometry{
 a4paper,
 total={170mm,257mm},
 left=20mm,
 top=20mm,
 }
\usepackage{graphicx}
\usepackage[hidelinks]{hyperref}
\usepackage[utf8]{inputenc}
\usepackage{listings}
\usepackage{mathdots}
\usepackage{mathtools}
\usepackage{stmaryrd}
\usepackage{tikz-cd}
\usepackage{csquotes}
\usepackage{nicematrix,tikz}
\usetikzlibrary{fit}
\usepackage{color}
\usepackage{xcolor}
\usepackage{xparse}
\usepackage{changepage}

% Formatting

\addto\captionsUKenglish{\renewcommand{\abstractname}{Syllabus}}
\delimitershortfall5pt
\ifx\thm\undefined\newtheorem{n}{}\else\newtheorem{n}{}[\thm]\fi
\newcommand\from{\leftarrow}
\newcommand\newoperator[1]{\ifcsdef{#1}{\cslet{#1}{\relax}}{}\csdef{#1}{\operatorname{#1}}}
\setlength{\parindent}{0cm}

\makeatletter  \def\m@th{\mathsurround\z@\color{red}} \makeatother
\everymath{\color{red}}
\everydisplay{\color{red}}
\def\thm{subsection}

% Environments

\newcommand{\setmytheoremdisplayskip}[1]{\setlength{\belowdisplayskip}{#1}}

\theoremstyle{definition}
\newtheorem{aim}[n]{Aim}
\newtheorem*{aim*}{Aim}
\newtheorem{aim**}{Aim}
\newtheorem{axiom}[n]{Axiom}
\newtheorem*{axiom*}{Axiom}
\newtheorem{axiom**}{Axiom}
\newtheorem{condition}[n]{Condition}
\newtheorem*{condition*}{Condition}
\newtheorem{condition**}{Condition}
\newtheorem{definition}[n]{Definition}
\newtheorem*{definition*}{Definition}
\newtheorem{definition**}{Definition}
\newtheorem{example}[n]{Example}
\newtheorem*{example*}{Example}
\newtheorem{example**}{Example}
\newtheorem{exercise}[n]{Exercise}
\newtheorem*{exercise*}{Exercise}
\newtheorem{exercise**}{Exercise}
\newtheorem{fact}[n]{Fact}
\newtheorem*{fact*}{Fact}
\newtheorem{fact**}{Fact}
\newtheorem{goal}[n]{Goal}
\newtheorem*{goal*}{Goal}
\newtheorem{goal**}{Goal}
\newtheorem{law}[n]{Law}
\newtheorem*{law*}{Law}
\newtheorem{law**}{Law}
\newtheorem{plan}[n]{Plan}
\newtheorem*{plan*}{Plan}
\newtheorem{plan**}{Plan}
\newtheorem{problem}[n]{Problem}
\newtheorem*{problem*}{Problem}
\newtheorem{problem**}{Problem}
\newtheorem{question}[n]{Question}
\newtheorem*{question*}{Question}
\newtheorem{question**}{Question}
\newtheorem{warning}[n]{Warning}
\newtheorem*{warning*}{Warning}
\newtheorem{warning**}{Warning}
\newtheorem{acknowledgements}[n]{Acknowledgements}
\newtheorem*{acknowledgements*}{Acknowledgements}
\newtheorem{acknowledgements**}{Acknowledgements}
\newtheorem{annotations}[n]{Annotations}
\newtheorem*{annotations*}{Annotations}
\newtheorem{annotations**}{Annotations}
\newtheorem{assumption}[n]{Assumption}
\newtheorem*{assumption*}{Assumption}
\newtheorem{assumption**}{Assumption}
\newtheorem{conclusion}[n]{Conclusion}
\newtheorem*{conclusion*}{Conclusion}
\newtheorem{conclusion**}{Conclusion}
\newtheorem{claim}[n]{Claim}
\newtheorem*{claim*}{Claim}
\newtheorem{claim**}{Claim}
\newtheorem{notation}[n]{Notation}
\newtheorem*{notation*}{Notation}
\newtheorem{notation**}{Notation}
\newtheorem{note}[n]{Note}
\newtheorem*{note*}{Note}
\newtheorem{note**}{Note}
\newtheorem{remark}[n]{Remark}
\newtheorem*{remark*}{Remark}
\newtheorem{remark**}{Remark}
\newtheorem{examples}[n]{Examples}
\newtheorem*{examples*}{Examples}
\newtheorem{examples**}{Examples}
\newtheorem{algorithm}[n]{Algorithm}
\newtheorem*{algorithm*}{Algorithm}
\newtheorem{algorithm**}{Algorithm}
\newtheorem{conjecture}[n]{Conjecture}
\newtheorem*{conjecture*}{Conjecture}
\newtheorem{conjecture**}{Conjecture}
\newtheorem{corollary}[n]{Corollary}
\newtheorem*{corollary*}{Corollary}
\newtheorem{corollary**}{Corollary}
\newtheorem{lemma}[n]{Lemma}
\newtheorem*{lemma*}{Lemma}
\newtheorem{lemma**}{Lemma}
\newtheorem{proposition}[n]{Proposition}
\newtheorem*{proposition*}{Proposition}
\newtheorem{proposition**}{Proposition}
\newtheorem{theorem}[n]{Theorem}
\newtheorem*{theorem*}{Theorem}
\newtheorem{theorem**}{Theorem}
\newtheorem{theorems}[n]{Theorems}

\newenvironment{widerequation}{%
    \begin{adjustwidth}{-2cm}{-2cm}\[}
    {\]\end{adjustwidth}}

% Lectures
\newcommand{\lecture}[3]{ % Lecture
  \marginpar{
    Lecture #1 \\
    #2 \\
    #3
  }
  \addtocontents{toc}{%
  \let\protect\mtnumberline\protect\numberline%
  \def\protect\numberline{%
  \hskip-\parindent%
  \global\let\protect\numberline\protect\mtnumberline%
  \protect\llap{\normalfont\normalsize Lecture #1 }%
  \hskip\parindent\protect\mtnumberline}}%
}

% Blackboard

\renewcommand{\AA}{\mathbb{A}} % Blackboard A
\newcommand{\BB}{\mathbb{B}}   % Blackboard B
\newcommand{\CC}{\mathbb{C}}   % Blackboard C
\newcommand{\DD}{\mathbb{D}}   % Blackboard D
\newcommand{\EE}{\mathbb{E}}   % Blackboard E
\newcommand{\FF}{\mathbb{F}}   % Blackboard F
\newcommand{\GG}{\mathbb{G}}   % Blackboard G
\newcommand{\HH}{\mathbb{H}}   % Blackboard H
\newcommand{\II}{\mathbb{I}}   % Blackboard I
\newcommand{\JJ}{\mathbb{J}}   % Blackboard J
\newcommand{\KK}{\mathbb{K}}   % Blackboard K
\newcommand{\LL}{\mathbb{L}}   % Blackboard L
\newcommand{\MM}{\mathbb{M}}   % Blackboard M
\newcommand{\NN}{\mathbb{N}}   % Blackboard N
\newcommand{\OO}{\mathbb{O}}   % Blackboard O
\newcommand{\PP}{\mathbb{P}}   % Blackboard P
\newcommand{\QQ}{\mathbb{Q}}   % Blackboard Q
\newcommand{\RR}{\mathbb{R}}   % Blackboard R
\renewcommand{\SS}{\mathbb{S}} % Blackboard S
\newcommand{\TT}{\mathbb{T}}   % Blackboard T
\newcommand{\UU}{\mathbb{U}}   % Blackboard U
\newcommand{\VV}{\mathbb{V}}   % Blackboard V
\newcommand{\WW}{\mathbb{W}}   % Blackboard W
\newcommand{\XX}{\mathbb{X}}   % Blackboard X
\newcommand{\YY}{\mathbb{Y}}   % Blackboard Y
\newcommand{\ZZ}{\mathbb{Z}}   % Blackboard Z

% Brackets

\renewcommand{\eval}[1]{\left. #1 \right|}                        % Evaluation
\newcommand{\br}{\del}                                            % Brackets
\newcommand{\abr}[1]{\left\langle #1 \right\rangle}               % Angle brackets
\newcommand{\fbr}[1]{\left\lfloor #1 \right\rfloor}               % Floor brackets
\newcommand{\intd}[4]{\int_{#1}^{#2} #3 \dif #4}            % Single integral
\newcommand{\iintd}[4]{\iint_{#1} #2 \dif #3 \dif #4}    % Double integral
\newcommand{\lintd}[4]{ \underline{\int_{#1}^{#2}} #3 \dif #4} % Lower integral
\newcommand{\uintd}[4]{ \overline{\int_{#1}^{#2}} #3 \dif #4}  % Upper integral


% Calligraphic

\newcommand{\AAA}{\mathcal{A}} % Calligraphic A
\newcommand{\BBB}{\mathcal{B}} % Calligraphic B
\newcommand{\CCC}{\mathcal{C}} % Calligraphic C
\newcommand{\DDD}{\mathcal{D}} % Calligraphic D
\newcommand{\EEE}{\mathcal{E}} % Calligraphic E
\newcommand{\FFF}{\mathcal{F}} % Calligraphic F
\newcommand{\GGG}{\mathcal{G}} % Calligraphic G
\newcommand{\HHH}{\mathcal{H}} % Calligraphic H
\newcommand{\III}{\mathcal{I}} % Calligraphic I
\newcommand{\JJJ}{\mathcal{J}} % Calligraphic J
\newcommand{\KKK}{\mathcal{K}} % Calligraphic K
\newcommand{\LLL}{\mathcal{L}} % Calligraphic L
\newcommand{\MMM}{\mathcal{M}} % Calligraphic M
\newcommand{\NNN}{\mathcal{N}} % Calligraphic N
\newcommand{\OOO}{\mathcal{O}} % Calligraphic O
\newcommand{\PPP}{\mathcal{P}} % Calligraphic P
\newcommand{\QQQ}{\mathcal{Q}} % Calligraphic Q
\newcommand{\RRR}{\mathcal{R}} % Calligraphic R
\newcommand{\SSS}{\mathcal{S}} % Calligraphic S
\newcommand{\TTT}{\mathcal{T}} % Calligraphic T
\newcommand{\UUU}{\mathcal{U}} % Calligraphic U
\newcommand{\VVV}{\mathcal{V}} % Calligraphic V
\newcommand{\WWW}{\mathcal{W}} % Calligraphic W
\newcommand{\XXX}{\mathcal{X}} % Calligraphic X
\newcommand{\YYY}{\mathcal{Y}} % Calligraphic Y
\newcommand{\ZZZ}{\mathcal{Z}} % Calligraphic Z

% Fraktur

\newcommand{\aaa}{\mathfrak{a}}   % Fraktur a
\newcommand{\bbb}{\mathfrak{b}}   % Fraktur b
\newcommand{\ccc}{\mathfrak{c}}   % Fraktur c
\newcommand{\ddd}{\mathfrak{d}}   % Fraktur d
\newcommand{\eee}{\mathfrak{e}}   % Fraktur e
\newcommand{\fff}{\mathfrak{f}}   % Fraktur f
\renewcommand{\ggg}{\mathfrak{g}} % Fraktur g
\newcommand{\hhh}{\mathfrak{h}}   % Fraktur h
\newcommand{\iii}{\mathfrak{i}}   % Fraktur i
\newcommand{\jjj}{\mathfrak{j}}   % Fraktur j
\newcommand{\kkk}{\mathfrak{k}}   % Fraktur k
\renewcommand{\lll}{\mathfrak{l}} % Fraktur l
\newcommand{\mmm}{\mathfrak{m}}   % Fraktur m
\newcommand{\nnn}{\mathfrak{n}}   % Fraktur n
\newcommand{\ooo}{\mathfrak{o}}   % Fraktur o
\newcommand{\ppp}{\mathfrak{p}}   % Fraktur p
\newcommand{\qqq}{\mathfrak{q}}   % Fraktur q
\newcommand{\rrr}{\mathfrak{r}}   % Fraktur r
\newcommand{\sss}{\mathfrak{s}}   % Fraktur s
\newcommand{\ttt}{\mathfrak{t}}   % Fraktur t
\newcommand{\uuu}{\mathfrak{u}}   % Fraktur u
\newcommand{\vvv}{\mathfrak{v}}   % Fraktur v
\newcommand{\www}{\mathfrak{w}}   % Fraktur w
\newcommand{\xxx}{\mathfrak{x}}   % Fraktur x
\newcommand{\yyy}{\mathfrak{y}}   % Fraktur y
\newcommand{\zzz}{\mathfrak{z}}   % Fraktur z

% Maps

\newcommand{\bijection}[7][]{    % Bijection
  \ifx &#1&
    \begin{array}{rcl}
      #2 & \longleftrightarrow & #3 \\
      #4 & \longmapsto         & #5 \\
      #6 & \longmapsfrom       & #7
    \end{array}
  \else
    \begin{array}{ccrcl}
      #1 & : & #2 & \longrightarrow & #3 \\
         &   & #4 & \longmapsto     & #5 \\
         &   & #6 & \longmapsfrom   & #7
    \end{array}
  \fi
}
\newcommand{\correspondence}[2]{ % Correspondence
  \cbr{
    \begin{array}{c}
      #1
    \end{array}
  }
  \qquad
  \leftrightsquigarrow
  \qquad
  \cbr{
    \begin{array}{c}
      #2
    \end{array}
  }
}
\newcommand{\function}[5][]{     % Function
  \ifx &#1&
    \begin{array}{rcl}
      #2 & \longrightarrow & #3 \\
      #4 & \longmapsto     & #5
    \end{array}
  \else
    \begin{array}{ccrcl}
      #1 & : & #2 & \longrightarrow & #3 \\
         &   & #4 & \longmapsto     & #5
    \end{array}
  \fi
}
\newcommand{\functions}[7][]{    % Functions
  \ifx &#1&
    \begin{array}{rcl}
      #2 & \longrightarrow & #3 \\
      #4 & \longmapsto     & #5 \\
      #6 & \longmapsto     & #7
    \end{array}
  \else
    \begin{array}{ccrcl}
      #1 & : & #2 & \longrightarrow & #3 \\
         &   & #4 & \longmapsto     & #5 \\
         &   & #6 & \longmapsto     & #7
    \end{array}
  \fi
}

% Matrices

\newcommand{\onebytwo}[2]{      % One by two matrix
  \begin{pmatrix}
    #1 & #2
  \end{pmatrix}
}
\newcommand{\onebythree}[3]{    % One by three matrix
  \begin{pmatrix}
    #1 & #2 & #3
  \end{pmatrix}
}
\newcommand{\twobyone}[2]{      % Two by one matrix
  \begin{pmatrix}
    #1 \\
    #2
  \end{pmatrix}
}
\newcommand{\twobytwo}[4]{      % Two by two matrix
  \begin{pmatrix}
    #1 & #2 \\
    #3 & #4
  \end{pmatrix}
}
\newcommand{\threebyone}[3]{    % Three by one matrix
  \begin{pmatrix}
    #1 \\
    #2 \\
    #3
  \end{pmatrix}
}
\newcommand{\threebythree}[9]{  % Three by three matrix
  \begin{pmatrix}
    #1 & #2 & #3 \\
    #4 & #5 & #6 \\
    #7 & #8 & #9
  \end{pmatrix}
}

\newenvironment{amatrix}[1]{%augmented matrix
  \left(\begin{array}{@{}*{#1}{c}|c@{}}
}{%
  \end{array}\right)
}

% Operators

\newoperator{ab}    % Abelian
\newoperator{AG}    % Affine geometry
\newoperator{alg}   % Algebraic
\newoperator{Ann}   % Annihilator
\newoperator{area}  % Area
\newoperator{Aut}   % Automorphism
\newoperator{BC}    % Bott-Chern
\newoperator{card}  % Cardinality
\newoperator{ch}    % Characteristic
\newoperator{Cl}    % Class
\newoperator{coker} % Cokernel
\newoperator{col}   % Column
\newoperator{Corr}  % Correspondence
\newoperator{diam}  % Diameter
\newoperator{Disc}  % Discriminant
\newoperator{dom}   % Domain
\newoperator{Eig}   % Eigenvalue
\newoperator{Em}    % Embedding
\newoperator{End}   % Endomorphism
\newoperator{Ext}   % Ext
\newoperator{fd}    % Flat dimension
\newoperator{fin}   % Finite
\newoperator{Fix}   % Fixed
\newoperator{Frac}  % Fraction
\newoperator{Frob}  % Frobenius
\newoperator{Fun}   % Function
\newoperator{Gal}   % Galois
\newoperator{gd}    % Global dimension
\newoperator{GL}    % General linear
\newoperator{Ham}   % Hamming
\newoperator{Hom}   % Homomorphism
\newoperator{Homeo} % Homeomorphism
\newoperator{id}    % Identity
\newoperator{im}    % Image
\newoperator{Ind}   % Index
\newoperator{ker}   % Kernel
\newoperator{lcm}   % Least common multiple
\newoperator{lgd}   % Left global dimension
\newoperator{Mat}   % Matrix
\newoperator{mult}  % Multiplicity
\newoperator{new}   % New
\newoperator{Nm}    % Norm
\newoperator{old}   % Old
\newoperator{op}    % Opposite
\newoperator{ord}   % Order
\newoperator{Pay}   % Payley
\newoperator{pd}    % Projective dimension
\newoperator{PG}    % Projective geometry
\newoperator{PGL}   % Projective general linear
\newoperator{prim}  % Primitive
\newoperator{PSL}   % Projective special linear
\newoperator{rad}   % Radical
\newoperator{ran}   % Range
\newoperator{Res}   % Residue
\newoperator{rgd}   % Right global dimension
\newoperator{rk}    % Rank
\newoperator{row}   % Row
\newoperator{sgn}   % Sign
\newoperator{Sing}  % Singular
\newoperator{SK}    % Skeleton
\newoperator{SL}    % Special linear
\newoperator{SO}    % Special orthogonal
\newoperator{sp}    % Span
\newoperator{Spec}  % Spectrum
\newoperator{srg}   % Strongly regular graph
\newoperator{Stab}  % Stabiliser
\newoperator{Star}  % Star
\newoperator{supp}  % Support
\newoperator{Sym}   % Symmetric
\newoperator{Tor}   % Tor
\newoperator{tors}  % Torsion
\newoperator{Tr}    % Trace
\newoperator{trdeg} % Transcendence degree
\newoperator{wgd}   % Weak global dimension
\newoperator{wt}    % Weight

% Roman

\newcommand{\A}{\mathrm{A}}   % Roman A
\newcommand{\B}{\mathrm{B}}   % Roman B
\newcommand{\C}{\mathrm{C}}   % Roman C
\newcommand{\D}{\mathrm{D}}   % Roman D
\newcommand{\E}{\mathrm{E}}   % Roman E
\newcommand{\F}{\mathrm{F}}   % Roman F
\newcommand{\G}{\mathrm{G}}   % Roman G
\renewcommand{\H}{\mathrm{H}} % Roman H
\newcommand{\I}{\mathrm{I}}   % Roman I
\newcommand{\J}{\mathrm{J}}   % Roman J
\newcommand{\K}{\mathrm{K}}   % Roman K
\renewcommand{\L}{\mathrm{L}} % Roman L
\newcommand{\M}{\mathrm{M}}   % Roman M
\newcommand{\N}{\mathrm{N}}   % Roman N
\renewcommand{\O}{\mathrm{O}} % Roman O
\renewcommand{\P}{\mathrm{P}} % Roman P
\newcommand{\Q}{\mathrm{Q}}   % Roman Q
\newcommand{\R}{\mathrm{R}}   % Roman R
\renewcommand{\S}{\mathrm{S}} % Roman S
\newcommand{\T}{\mathrm{T}}   % Roman T
\newcommand{\U}{\mathrm{U}}   % Roman U
\newcommand{\V}{\mathrm{V}}   % Roman V
\newcommand{\W}{\mathrm{W}}   % Roman W
\newcommand{\X}{\mathrm{X}}   % Roman X
\newcommand{\Y}{\mathrm{Y}}   % Roman Y
\newcommand{\Z}{\mathrm{Z}}   % Roman Z

\renewcommand{\a}{\mathrm{a}} % Roman a
\renewcommand{\b}{\mathrm{b}} % Roman b
\renewcommand{\c}{\mathrm{c}} % Roman c
\renewcommand{\d}{\mathrm{d}} % Roman d
\newcommand{\e}{\mathrm{e}}   % Roman e
\newcommand{\f}{\mathrm{f}}   % Roman f
\newcommand{\g}{\mathrm{g}}   % Roman g
\newcommand{\h}{\mathrm{h}}   % Roman h
\renewcommand{\i}{\mathrm{i}} % Roman i
\renewcommand{\j}{\mathrm{j}} % Roman j
\renewcommand{\k}{\mathrm{k}} % Roman k
\renewcommand{\l}{\mathrm{l}} % Roman l
\newcommand{\m}{\mathrm{m}}   % Roman m
\renewcommand{\n}{\mathrm{n}} % Roman n
\renewcommand{\o}{\mathrm{o}} % Roman o
\newcommand{\p}{\mathrm{p}}   % Roman p
\newcommand{\q}{\mathrm{q}}   % Roman q
\renewcommand{\r}{\mathrm{r}} % Roman r
\newcommand{\s}{\mathrm{s}}   % Roman s
\renewcommand{\t}{\mathrm{t}} % Roman t
\renewcommand{\u}{\mathrm{u}} % Roman u
\renewcommand{\v}{\mathrm{v}} % Roman v
\newcommand{\w}{\mathrm{w}}   % Roman w
\newcommand{\x}{\mathrm{x}}   % Roman x
\newcommand{\y}{\mathrm{y}}   % Roman y
\newcommand{\z}{\mathrm{z}}   % Roman z

% Tikz

\tikzset{
  arrow symbol/.style={"#1" description, allow upside down, auto=false, draw=none, sloped},
  subset/.style={arrow symbol={\subset}},
  cong/.style={arrow symbol={\cong}}
}

\pagestyle{fancy}
\lhead{\module}
\rhead{\nouppercase{\leftmark}}

% Make title

\title{\module}
\author{Lectured by \lecturer \\ Typed by Yu Coughlin}
%Headers and Covers taken from David Kurniadi Angdinata
\date{\term}

\begin{document}

% Title page
\maketitle
\cover
\vfill
\begin{abstract}
\noindent\syllabus
\end{abstract}

\pagebreak

% Contents page
\tableofcontents

\pagebreak

% Document page
\setcounter{section}{-1}

\section{Introduction}

\lecture{1}{Thursday}{10/01/19}

The following are references.
\begin{itemize}
\item E Artin, Galois theory, 1994
\item A Grothendieck and M Raynaud, Rev\^etements \'etales et groupe fondamental, 2002
\item I N Herstein, Topics in algebra, 1975
\item M Reid, Galois theory, 2014
\end{itemize}

\begin{notation*}
If $ K $ is a field, or a ring, I denote
$$ K\sbr{X} = \cbr{a_0 + \dots + a_nX^n \st a_i \in K}, $$
the \textbf{ring of polynomials} with coefficients in $ K $.
\end{notation*}

\section{Number systems}

\subsection{Naturals, integers and rationals}

\begin{definition}[Natural numbers]
    As in IUM, we define the \textbf{natural numbers}, $\NN$, from the Peano axioms: \begin{enumerate}
        \item[P1] $0$ is a natural number,
        \item[P6] if $n$ is a natural number then $S(n)$ is a natural number where $S(n)$ is the successor of $n$,
        \item[P9] the principle of mathematical induction.
    \end{enumerate}
    Clearly, there are many Peano axioms not included, these are however not particularly relevant to this course. Addition and multiplication is defined as expected and will descend to our other number systems
\end{definition}

\begin{definition}[Integers]
    The \textbf{integers} are defined as $\ZZ := \NN\times\NN/\sim$ where $\sim$ is the equivalence relation given by $(a,b)\sim(c,d)$ iff $a+d = b+c$. Subtraction is defined as expected and will also descend to our other number systems.
\end{definition}

\begin{definition}[Rationals]
    The \textbf{rationals} are defined as $\QQ := \ZZ\times\NN^{>0}/\sim$ where $\sim$ is the equivalence relation given by $(a,b)\sim(c,d)$ iff $ad=bc$. The equivalence class $(p,q)$ will be written as $\frac{p}{q}$. There is an element of each equivalence class $\frac{p'}{q'}$ with $\gcd(p',q')=1$, we say that $\frac{p'}{q'}$ is in \textbf{lowest terms}.
\end{definition}

\begin{theorem}[Axioms of the rationals]
    With the usual operations descended from $\NN$ and $\ZZ$, $\QQ$ satisfies the following axioms with $a,b,c\in\QQ$ throughout: \begin{enumerate}
        \item[Q1] $a+(b+c)=(a+b)+c$ ($+$ is associative),
        \item[Q2] $\exists 0\in\QQ$ such that $a+0=a$ ($0$ is the additive identity of $\QQ$),
        \item[Q3] $\forall a \in \QQ, \exists (-a)\in\QQ$ such that $a+ (-a) = 0$ ($\QQ$ is closed under additive inverses),
        \item[Q4] $a+b = b+a$ ($+$ is commutative), \\
        \item[Q5] $a\times(b\times c)=(a\times b)\times c$ ($\times$ is associative),
        \item[Q6] $\exists 1\in\QQ$ such that $a\times1=a$ ($1$ is the multiplicative identity of $\QQ$),
        \item[Q7] $a\times(b+c) = (a\times b) + (a\times c)$ ($\times$ is left distributive over $+$),
        \item[Q8] $(a+b)\times c = (a\times c) + (b\times c)$ ($\times$ is right distributive over $+$),
        \item[Q9] $a\times b = b\times a$ ($\times$ is commutative), \\
        \item[Q10] $\forall a \in \QQ, \exists a^{-1}\in\QQ$ such that $a \times a^{-1} = 1$ ($\QQ$ is closed under multiplicative inverses), \\
        \item[Q11] for all $a\in\QQ$ either $x<0$, $x=0$ or $x>=$ (Trichotomy),
        \item[Q12] for all $x,y\in\QQ$ we have $x>0,y>0\implies x+y>0$,
        \item[Q13] for all $x\in\QQ$ there exists a $n\in\NN$ such that $x<n$ (Archimedean axiom).
    \end{enumerate} $1$-$4$ says $(\QQ,+)$ is an abelian group, $1$-$9$ says $(\QQ,+,\times)$ is a commutative ring, $1$-$10$ says $(\QQ,+,\times)$ is a field.
\end{theorem}

\subsection{Decimal expansions}

\begingroup\belowdisplayskip=-0pt

\begin{definition}
    For $a_0\in\NN$ and $a_i\in[1,9]$ for $i>0\in\NN$, define the \textbf{periodic decimal} \[
        a_0.a_1a_2\ldots\overline{a_ia_{i+1}\ldots a_j}
    \textcolor{black}{,}
    \] to be equal to the rational number \[
        a_0 + \frac{a_1}{10} + \frac{a_2}{100} + \ldots + \frac{a_i}{10^i} + \br{\frac{a_{i+1}a_{i+2}\ldots a_j}{10^j}}\br{\frac{1}{1-10^{i-j}}}
    \textcolor{black}{.}
    \]
\end{definition}

\endgroup
\begingroup\belowdisplayskip=-10pt

\begin{theorem}
    If $x\in\QQ$ has $2$ decimal expansions, then they will be of the form \[
        x=a_0.a_1a_2\ldots a_n\overline{9} = a_0.a_1a_2\ldots(a_n+1), a_n\in[0,8]
    \textcolor{black}{.}    
    \]
\end{theorem}

\begin{definition}[Real numbers]
    The \textbf{real numbers}, $\RR$, can be defined as: \[
        \RR:= \{a_0.a_1a_2\ldots : a_0\in\ZZ, a_i\in[0,9],\not\exists N\in\NN \ \text{\textcolor{black}{such that }} a_i=9 \ \forall i\geq N\}
    \textcolor{black}{.}
    \]
\end{definition}

\endgroup

\subsection{Countability}

\begin{definition}[Countability]
    A set $S$ is \textbf{countably infinite} iff there exists a bijection $f:\NN\rightarrow S$, a set is \textbf{countable} if it is finite or countable infinite.
\end{definition}

\begin{theorem}
    All $S\subseteq\NN$ are countable, $\ZZ$ and $\QQ$ are both countable, $\RR$ is uncountable.
\end{theorem}

\section{Bounded sets}

\subsection{Supremums and infinums}
\begin{definition}[Maximum and minimum]
    $s\in\RR$ is the \textbf{maximum} of a set $S\subset\RR$ iff $\forall s'\in S$, $s\geq s'$. \textbf{Minimums} are defined similarly. Maximums and minimums are unique.
\end{definition}

\begin{definition}[Bounded]
    A non-empty set $S\subset\RR$ is \textbf{bounded above} if there exists some $M\in\RR$ such that $\forall s\in S$, $s\leq M$ with \textbf{bounded below} defined similarly. $S$ is \textbf{bounded} if it is both bounded above and bounded below.
\end{definition}

\begin{theorem}
    If $S$ is bounded then $\exists R>0$ such that $|s|<\RR$ for all $s\in S$.
\end{theorem}

\begin{definition}[Supremum and infinum]
    If $S\subset\RR$ is bounded above, we say $x\in\RR$ is the \textbf{least upper bound} or \textbf{supremum} iff $x$ is and upper bound for $S$ and for all $y\in\RR$ such that $y$ is an upper bound of $S$, $x\leq y$. The \textbf{infinum} is defined similarly.
\end{definition}

\subsection{Completeness}

\begin{theorem}[Completeness axiom]
    For all non-empty $S\subset\RR$, if $S$ is bounded above then $S$ has a supremum, and similarly for $S$ bounded below.
\end{theorem}

\subsection{Dedekind cuts}

\begin{definition}[Dedekind cut]
    A non-empty set $S\subset\QQ$ is a \textbf{Dedekind cut} if it satisfies: \begin{enumerate}
        \item $s\in S$ and $s>t\in\QQ \implies t\in S$ ($S$ is a semi-infinite interval to the left),
        \item $S$ is bounded above with no maximum.
    \end{enumerate} Dedekind cuts are in the form $S_r:=(-\infty,r)\cap\QQ$.
\end{definition}

\begin{theorem}[Real numbers]
    We can redefine the reals as the set of Dedekind cuts, $\RR:= \{S_r\subset\QQ\}$. All operations and orderings as well as the completeness axiom are held by this new Dedekind cut definition.
\end{theorem}

\begin{theorem}[Triangle innequality]
    For all $a,b\in\RR$ we have $|a+b|\leq |a|+|b|$.
\end{theorem}

\section{Sequences}

\begin{definition}[Real sequence]
    A \textbf{real sequence} is a function $a:\NN\rightarrow\RR$ written $(a_n)$. Sequences of other number systems are defined similarly.
\end{definition}

\subsection{Convergence}

\begin{definition}[Convergence]
    A real sequence $(a_n)$ \textbf{converges} to some $a\in\RR$  as $n\rightarrow\infty$ iff \[
    \forall \epsilon>0, \  \exists N_\epsilon \ \text{\textcolor{black}{such that }} \forall n\geq N_\epsilon, \ |a_n-a|<\epsilon
    \textcolor{black}{.}
    \] For complex series the definition is the same just with $|\cdot|$ referring to the modulus instead of the absolute value.
\end{definition}

\begin{customtheorem}[Yet Another Theorem][-10pt]
This is a custom theorem with a different optional displayskip.
\[
E=mc^2
\]
\end{customtheorem}

\begin{definition}
    asdfawe
\end{definition}

\subsection{Divergence}

\subsection{Limits}

\subsection{Monotone sequences}

\subsection{Cauchy sequences}

\subsection{Subsequences}

\section{Series}

\subsection{Convergence}

\subsection{Convergence tests}

\subsection{Rearrangement of series}

\subsection{Power series}

\subsection{Exponential series}

\section{Continuity}

\subsection{Limits}

\subsection{Continuous functions}

\subsection{Properties of continuity}

\section{Properties of subsets}

\subsection{Open sets}

\subsection{Closed sets}

\subsection{Compact sets}

\section{Uniform continuity and convergence}

\subsection{Uniform continuity}

\subsection{Convergence of sequences of functions}

\subsection{Convergence of series of functions}

\section{Differentiation}

\subsection{Differentiability}

\subsection{Operations on derivatives}

\subsection{Local extrema and mean values}

\subsection{L'Hôpital's rule}

\subsection{Taylor's theorem}

\subsection{Convexity}

\subsection{Exchange of limits and derivatives}

\subsection{Derivation of pi}

\section{Integration}

\subsection{Partitions}

\subsection{Darboux sums}

\subsection{Darboux integral}

\subsection{Integrability}

\subsection{Properties of integration}

\subsection{Fundamental theorem of calculus}

\subsection{Exchange of limits and integrals}

\subsection{Improper integrals}

\subsection{Lebesgue criterion}

\subsection{Irrationality of pi}

\end{document}