\def\module{MATH40002 Analysis 1}
\def\lecturer{Dr Ajay Chandra}
\def\term{Autumn 2023}
\def\cover{\vspace{1in}
$$
\begin{tikzcd}[ampersand replacement=\&, column sep=tiny]
\& \qquad \& \& \& L \arrow[dash, dashed]{dddd} \& \& \& \& \& \& \\
K\br{\alpha} \arrow[dash]{urrrr} \arrow[dash, dashed]{dddd} \& \& K\br{\alpha'} \arrow[dash]{urr} \arrow[dash, dashed]{dddd} \& \& \& K\br{\beta, \gamma} \arrow[dash]{ul} \arrow[dash, dashed]{dddd} \& \& \& K\br{\delta} \arrow[dash]{ullll} \arrow[dash, dashed]{dddd} \& \& K\br{\delta'} \arrow[dash]{ullllll} \arrow[dash, dashed]{dddd} \\
\& \& \& K\br{\beta} \arrow[crossing over, dash]{ulll} \arrow[dash]{ul} \arrow[crossing over, dash]{urr} \arrow[dash, dashed]{dddd} \& \& \& K\br{\beta\gamma} \arrow[crossing over, dash]{ul} \arrow[dash, dashed]{dddd} \& \& \& K\br{\gamma} \arrow[crossing over, dash]{ullll} \arrow[crossing over, dash]{ul} \arrow[crossing over, dash]{ur} \arrow[dash, dashed]{dddd} \& \\
\& \& \& \& \& \& \& K \arrow[crossing over, dash]{ullll} \arrow[crossing over, dash]{ul} \arrow[crossing over, dash]{urr} \arrow[dash, dashed]{dddd} \& \& \& \\
\& \qquad \& \& \& \abr{e} \& \& \& \& \& \& \\
\abr{\tau} \arrow[dash]{urrrr} \& \& \abr{\sigma^2\tau} \arrow[dash]{urr} \& \& \& \abr{\sigma^2} \arrow[dash]{ul} \& \& \& \abr{\sigma\tau} \arrow[dash]{ullll} \& \& \abr{\sigma^3\tau} \arrow[dash]{ullllll} \\
\& \& \& \abr{\sigma^2, \tau} \arrow[dash]{ulll} \arrow[dash]{ul} \arrow[dash]{urr} \& \& \& \abr{\sigma} \arrow[dash]{ul} \& \& \& \abr{\sigma^2, \sigma\tau} \arrow[dash]{ullll} \arrow[dash]{ul} \arrow[dash]{ur} \& \\
\& \& \& \& \& \& \& G \arrow[dash]{ullll} \arrow[dash]{ul} \arrow[dash]{urr} \& \& \&
\end{tikzcd}
$$
$$ G = \Gal\br{L / K} \cong \DDD_8 $$
}
\def\syllabus{Number systems, decimal expansions, sup and inf, sequences, series, convergence tests, power series, continuity, closure, compactness, uniform continuity, differentiation, integration}
\def\thm{subsection}

\input{../style/header}

\begin{document}

\input{../style/cover}

\section{Introduction}

\lecture{1}{Thursday}{10/01/19}

The following are references.
\begin{itemize}
\item E Artin, Galois theory, 1994
\item A Grothendieck and M Raynaud, Rev\^etements \'etales et groupe fondamental, 2002
\item I N Herstein, Topics in algebra, 1975
\item M Reid, Galois theory, 2014
\end{itemize}

\begin{notation*}
If $ K $ is a field, or a ring, I denote
$$ K\sbr{X} = \cbr{a_0 + \dots + a_nX^n \st a_i \in K}, $$
the \textbf{ring of polynomials} with coefficients in $ K $.
\end{notation*}

\section{Number systems}

\subsection{Naturals, integers and rationals}

\begin{definition}[Natural numbers]
    As in IUM, we define the \textbf{natural numbers}, $\NN$, from the Peano axioms: \begin{enumerate}
        \item[P1] $0$ is a natural number,
        \item[P6] if $n$ is a natural number then $S(n)$ is a natural number where $S(n)$ is the successor of $n$,
        \item[P9] the principle of mathematical induction.
    \end{enumerate}
    Clearly, there are many Peano axioms not included, these are however not particularly relevant to this course. Addition and multiplication is defined as expected and will descend to our other number systems
\end{definition}

\begin{definition}[Integers]
    The \textbf{integers} are defined as $\ZZ := \NN\times\NN/\sim$ where $\sim$ is the equivalence relation given by $(a,b)\sim(c,d)$ iff $a+d = b+c$. Subtraction is defined as expected and will also descend to our other number systems.
\end{definition}

\begin{definition}[Rationals]
    The \textbf{rationals} are defined as $\QQ := \ZZ\times\NN^{>0}/\sim$ where $\sim$ is the equivalence relation given by $(a,b)\sim(c,d)$ iff $ad=bc$. The equivalence class $(p,q)$ will be written as $\frac{p}{q}$. There is an element of each equivalence class $\frac{p'}{q'}$ with $\gcd(p',q')=1$, we say that $\frac{p'}{q'}$ is in \textbf{lowest terms}.
\end{definition}

\begin{theorem}[Axioms of the rationals]
    With the usual operations descended from $\NN$ and $\ZZ$, $\QQ$ satisfies the following axioms with $a,b,c\in\QQ$ throughout: \begin{enumerate}
        \item[Q1] $a+(b+c)=(a+b)+c$ ($+$ is associative),
        \item[Q2] $\exists 0\in\QQ$ such that $a+0=a$ ($0$ is the additive identity of $\QQ$),
        \item[Q3] $\forall a \in \QQ, \exists (-a)\in\QQ$ such that $a+ (-a) = 0$ ($\QQ$ is closed under additive inverses),
        \item[Q4] $a+b = b+a$ ($+$ is commutative),\vspace{5pt}
        \item[Q5] $a\times(b\times c)=(a\times b)\times c$ ($\times$ is associative),
        \item[Q6] $\exists 1\in\QQ$ such that $a\times1=a$ ($1$ is the multiplicative identity of $\QQ$),
        \item[Q7] $a\times(b+c) = (a\times b) + (a\times c)$ ($\times$ is left distributive over $+$),
        \item[Q8] $(a+b)\times c = (a\times c) + (b\times c)$ ($\times$ is right distributive over $+$),
        \item[Q9] $a\times b = b\times a$ ($\times$ is commutative),\vspace{5pt}
        \item[Q10] $\forall a \in \QQ, \exists a^{-1}\in\QQ$ such that $a \times a^{-1} = 1$ ($\QQ$ is closed under multiplicative inverses),\vspace{5pt}
        \item[Q11] for all $a\in\QQ$ either $x<0$, $x=0$ or $x>=$ (Trichotomy),
        \item[Q12] for all $x,y\in\QQ$ we have $x>0,y>0\implies x+y>0$,
        \item[Q13] for all $x\in\QQ$ there exists a $n\in\NN$ such that $x<n$ (Archimedean axiom).
    \end{enumerate} $1$-$4$ says $(\QQ,+)$ is an abelian group, $1$-$9$ says $(\QQ,+,\times)$ is a commutative ring, $1$-$10$ says $(\QQ,+,\times)$ is a field.
\end{theorem}

\subsection{Decimal expansions}

\begingroup\belowdisplayskip=-0pt

\begin{definition}
    For $a_0\in\NN$ and $a_i\in[1,9]$ for $i>0\in\NN$, define the \textbf{periodic decimal} \[
        a_0.a_1a_2\ldots\overline{a_ia_{i+1}\ldots a_j}
    \textcolor{black}{,}
    \] to be equal to the rational number \[
        a_0 + \frac{a_1}{10} + \frac{a_2}{100} + \ldots + \frac{a_i}{10^i} + \br{\frac{a_{i+1}a_{i+2}\ldots a_j}{10^j}}\br{\frac{1}{1-10^{i-j}}}
    \textcolor{black}{.}
    \]
\end{definition}

\endgroup
\begingroup\belowdisplayskip=-10pt

\begin{theorem}
    If $x\in\QQ$ has $2$ decimal expansions, then they will be of the form \[
        x=a_0.a_1a_2\ldots a_n\overline{9} = a_0.a_1a_2\ldots(a_n+1), a_n\in[0,8]
    \textcolor{black}{.}    
    \]
\end{theorem}

\begin{definition}[Real numbers]
    The \textbf{real numbers}, $\RR$, can be defined as: \[
        \RR:= \{a_0.a_1a_2\ldots : a_0\in\ZZ, a_i\in[0,9],\not\exists N\in\NN \ \text{\textcolor{black}{such that }} a_i=9 \ \forall i\geq N\}
    \textcolor{black}{.}
    \]
\end{definition}

\endgroup

\subsection{Countability}

\begin{definition}[Countability]
    A set $S$ is \textbf{countably infinite} iff there exists a bijection $f:\NN\rightarrow S$, a set is \textbf{countable} if it is finite or countable infinite.
\end{definition}

\begin{theorem}
    All $S\subseteq\NN$ are countable, $\ZZ$ and $\QQ$ are both countable, $\RR$ is uncountable.
\end{theorem}

\section{Bounded sets}

\subsection{Supremums and infinums}
\begin{definition}[Maximum and minimum]
    $s\in\RR$ is the \textbf{maximum} of a set $S\subset\RR$ iff $\forall s'\in S$, $s\geq s'$. \textbf{Minimums} are defined similarly. Maximums and minimums are unique.
\end{definition}

\begin{definition}[Bounded]
    A non-empty set $S\subset\RR$ is \textbf{bounded above} if there exists some $M\in\RR$ such that $\forall s\in S$, $s\leq M$ with \textbf{bounded below} defined similarly. $S$ is \textbf{bounded} if it is both bounded above and bounded below.
\end{definition}

\begin{theorem}
    If $S$ is bounded then $\exists R>0$ such that $|s|<\RR$ for all $s\in S$.
\end{theorem}

\begin{definition}[Supremum and infinum]
    If $S\subset\RR$ is bounded above, we say $x\in\RR$ is the \textbf{least upper bound} or \textbf{supremum} iff $x$ is and upper bound for $S$ and for all $y\in\RR$ such that $y$ is an upper bound of $S$, $x\leq y$. The \textbf{infinum} is defined similarly.
\end{definition}

\subsection{Completeness}

\begin{theorem}[Completeness axiom]
    For all non-empty $S\subset\RR$, if $S$ is bounded above then $S$ has a supremum, and similarly for $S$ bounded below.
\end{theorem}

\subsection{Dedekind cuts}

\begin{definition}[Dedekind cut]
    A non-empty set $S\subset\QQ$ is a \textbf{Dedekind cut} if it satisfies: \begin{enumerate}
        \item $s\in S$ and $s>t\in\QQ \implies t\in S$ ($S$ is a semi-infinite interval to the left),
        \item $S$ is bounded above with no maximum.
    \end{enumerate} Dedekind cuts are in the form $S_r:=(-\infty,r)\cap\QQ$.
\end{definition}

\begin{theorem}[Real numbers]
    We can redefine the reals as the set of Dedekind cuts, $\RR:= \{S_r\subset\QQ\}$. All operations and orderings as well as the completeness axiom are held by this new Dedekind cut definition.
\end{theorem}

\begin{theorem}[Triangle innequality]
    For all $a,b\in\RR$ we have $|a+b|\leq |a|+|b|$.
\end{theorem}

\section{Sequences}

\begin{definition}[Real sequence]
    A \textbf{real sequence} is a function $a:\NN\rightarrow\RR$ written $(a_n)$. Sequences of other number systems are defined similarly.
\end{definition}

\subsection{Convergence}

\begin{definition}[Convergence of sequences]
    A real sequence $(a_n)$ \textbf{converges} to some $a\in\RR$  as $n\rightarrow\infty$ iff \[
    \forall \epsilon>0, \  \exists N_\epsilon \ \text{\textcolor{black}{such that }} \forall n\geq N_\epsilon, \ |a_n-a|<\epsilon
    \textcolor{black}{.}
    \] For complex series the definition is the same just with $|\cdot|$ referring to the modulus instead of the absolute value. This is written $a_n\rightarrow a$ (as $n\rightarrow\infty$).
\end{definition}

\subsection{Divergence}

\begin{definition}[Divergence]
    A sequence $(a_n)$ \textbf{diverges} iff it doesn't converge.
\end{definition}

\begin{definition}[Divergence to infinity]
    A sequence $(a_n)$ \textbf{diverges to $\infty$} iff $\forall R>0, \ \exists N\in\NN$, such that $\forall n\geq N, a_n>R$. And similarly for a sequence diverging to $-\infty$.
\end{definition}

\subsection{Limits}

\begin{theorem}[Uniqueness of limits]
    Given a sequence $(a_n)$ if $a_n\rightarrow a$ and $a_n\rightarrow b$, $a=b$.
\end{theorem}

\begin{theorem}
    If a sequence $(a_n)$ converges then $(a_n)$ is bounded.
\end{theorem}

\begin{theorem}[Algebra of limits]
    Given two sequences $a_n\rightarrow a$ and $b_n\rightarrow b$ the following hold: \begin{itemize}
        \item $a_n + b_n \rightarrow a + b$,
        \item $a_nb_n \rightarrow ab$ (a special case of this is $ca_n\rightarrow ca$ for a constant $c$),
        \item $\displaystyle{\frac{a_n}{b_n} \rightarrow \frac{a}{b}}$ given $b\neq0$.
    \end{itemize}
\end{theorem}

\vspace{-15pt}

\begin{theorem}
    If $(a_n)$ is a positive sequence then $a_n\rightarrow0\iff\displaystyle{\frac{1}{a_n}\rightarrow+\infty}$, and similarly for negative sequences.
\end{theorem}

\begin{theorem}[Ratio test ]
    If a sequence $(a_n)$ satisfies $\displaystyle \abs{\frac{a_{n+1}}{a_n}}\rightarrow L <1$ then $a_n\rightarrow0$.
\end{theorem}

\subsection{Monotone sequences}

\begin{definition}[Monotonically increasing sequence]
    A sequence, $(a_n)$, is \textbf{monotonically increasing} iff $\forall m,n\in\NN$ with $n>m$ we have $a_n\geq a_m$, and similarly for monotonically decreasing and their strict equivalents.
\end{definition}

\begin{theorem}[Monotone convergence]
    If a sequence $(a_n)$ is monotone increasing and bounded above then $a_n\rightarrow a := \sup\{a_i:i\in\NN\}$ written $a_n\uparrow a$. This holds similarly for monotone decreasing sequences.
\end{theorem}

\subsection{Cauchy sequences}

\begin{definition}[Cauchy sequence]
   A sequence $(a_n)$ is a \textbf{Cauchy sequence} iff $\forall \epsilon>0\in\RR, \ \exists N\in\NN$ such that $\forall n,m<N, \ |a_n-a_m|<\epsilon$.
\end{definition}

\begin{theorem}[Cauchy convergence criterion]
    A sequence $(a_n)$ converges iff it is a Cauchy sequence.
\end{theorem}

\subsection{Subsequences}

\begin{definition}[Subsequence]
    Given a strictly monotonically increasing function $n:\NN\rightarrow\NN$ and a sequence $(a_n)$, the  sequence $(b_n)$ defined by $b_i:=a_{n(i)}$ is a \textbf{subsequence} of $(a_n)$.
\end{definition}

\begin{theorem}
    Given a subsequence of $(a_n)$, $(a_{n(i)})$, if $a_n\rightarrow a$ then $a_{n(i)}\rightarrow a$ as $i\rightarrow\infty$.
\end{theorem}

\begin{theorem}[Bolzano-Weierstrass]
    If a sequence $(a_n)$ is bounded then it has a convergent subsequence.
\end{theorem}

\begin{note}[Sketch of the Bolzano-Weierstrass theorem proof]
    The proof of the Bolzano-Weierstrass theorem is an equally valuable point as the statement of the theorem itself. The idea of the proof considers the ``peak points" of the sequence: if there are infinitely many peak points, then the peak points themselves form a monotonically decreasing subsequence; if there are finitely many, then the points after the final peak must have a monotonically decreasing subsequence bounded above by the final peak. By the monotone convergence theorem both of these subsequences must converge.
\end{note}

\section{Series}

\begingroup\belowdisplayskip=-10pt
\begin{definition}[Infinite series]
    An \textbf{(infinite) series} is an expression of the form $\displaystyle \sum_{i=1}^\infty a_i$ of $a_1 + a_2 + \ldots$ 
    
    \vspace{-8pt}
    for some sequence $(a_n)$. The sequence \textbf{partial sums} of the series $(S_n)$ is given by \[
        S_n := \sum_{i=1}^n a_i = a_1 + a_2 + \ldots + a_n
        \textcolor{black}{.}
    \]
\end{definition}
\endgroup

\subsection{Convergence}

\begin{definition}[Convergence of series]
    The series $\displaystyle \sum_{i=1}^\infty a_i$ of $(a_n)$ \textbf{converges} iff 
    $S_n\rightarrow A\in\RR$, 
        
    \vspace{-10pt}
    written $\displaystyle\sum_{n=1}^\infty a_n=A$.
\end{definition}

\vspace{-23pt}

\begin{theorem}
    For a sequence $(a_n)$, $\displaystyle\sum_{n=1}^\infty a_n$ converges if $a_n\rightarrow0$ (the converse is not true).
\end{theorem}

\vspace{-15pt}

\begin{theorem}
    Given a sequence non-negative sequence $(a_n)$, the convergence of the infinite series and the boundedness of $(S_n)$ are equivalent.
\end{theorem}

\begin{theorem}[Algebra of limits for series]
    A similar algebra of limits for series can be established from the algebra of limits for sequences acting on the partial sums of the series.
\end{theorem}

\begin{theorem}[Comparison I test]
    Given sequences $(a_n),(b_n)$ if $0\leq a_n \leq b_n$ then: \begin{itemize}
        \item If $\displaystyle\sum_{n=1}^\infty a_n$ and $\displaystyle\sum_{n=1}^\infty b_n$ converge, $0\leq \displaystyle\sum_{n=1}^\infty a_n \leq \displaystyle\sum_{n=1}^\infty b_n$,
        \item If $\displaystyle\sum_{n=1}^\infty a_n$ diverges, $\displaystyle\sum_{n=1}^\infty b_n$ also diverges.
    \end{itemize}
\end{theorem}

\begin{theorem}[Comparison II test (Sandwich theorem)]
    Given sequences $(a_n), (b_n), (c_n)$ with $a_n\leq b_n\leq c_n$, if $\displaystyle\sum_{n=1}^\infty a_n$ and $\displaystyle\sum_{n=1}^\infty c_n$ both converge, $\displaystyle\sum_{n=1}^\infty b_n$ converges.
\end{theorem}

\begin{theorem}
    If $\alpha>1\in\RR$, $\displaystyle\sum_{n=1}^\infty \frac{1}{n^\alpha}$ converges.
\end{theorem}

\begin{definition}[Alternating sequence]
    A sequence $(a_n)$ is \textbf{alternating} iff $a_{2n}\geq 0$ and $a_{2n-1}\leq0$ of vice versa for all $n\in\NN^{>0}$.
\end{definition}

\begin{theorem}
    If $(a_n)$ is alternating with $|a_n|\downarrow 0$, $a_n$ converges and $\displaystyle\sum_{n=1}^\infty a_n$ converges.
\end{theorem}

\subsection{Absolute convergence}

\begin{definition}[Absolute convergence]
    Given a sequence $(a_n)$ the series $\displaystyle\sum_{n=1}^\infty a_n$ is \textbf{absolutely convergent} iff $\displaystyle\sum_{n=1}^\infty |a_n|$ converges.
\end{definition}

\begin{theorem}
    Absolute convergence $\implies$ convergence.
\end{theorem}

\begin{theorem}[Comparison III test]
    Given sequences $(a_n),(b_n)$ with $\displaystyle \frac{a_n}{b_n}\rightarrow L\in\RR$ if $\displaystyle\sum_{n=1}^\infty b_n$ is absolutely convergent then $\displaystyle\sum_{n=1}^\infty a_n$ is also absolutely convergent.
\end{theorem}

\begin{theorem}[Ratio test]
    If the sequence $(a_n)$ is such that $\displaystyle\abs{\frac{a_{n+1}}{a_n}}\rightarrow r<1$ then $\displaystyle\sum_{n=1}^\infty a_n$ is absolutely convergent or divergent if $r>1$.
\end{theorem}

\begin{theorem}[Root test]
    If the sequence $(a_n)$ is such that $\displaystyle\abs{a_n}^{\frac{1}{n}}\rightarrow r<1$ then $\displaystyle\sum_{n=1}^\infty a_n$ is absolutely convergent or divergent is $r>1$.
\end{theorem}

\begin{remark}
    Both the ratio test and the root test are inconclusive if $r=1$.
\end{remark}

\subsection{Rearrangement of series}
Sometimes, series are easier to deal with and have cancellations when their terms are rearranged. However, the rearrangement of terms will only preserve limits under certain conditions.

\begin{definition}[Reordering]
    Given a bijection $n:\NN\rightarrow\NN$ and a sequence $(a_n)$, the sequence $(b_n)$ with $b_i:=a_{n(i)}$ is a \textbf{rearrangement} or \textbf{reordering} of $(a_n)$.
\end{definition}

\begin{theorem}
    If $(a_n)$ is a sequence satisying $a_n\rightarrow0$, $\displaystyle\sum_{n:a_n\geq0} a_n = \infty$ and $\displaystyle\sum_{n:a_n\leq0} a_n = -\infty$ then $\displaystyle\sum_{n=1}^\infty a_n$ can be rearranged to converge to any $r\in\RR$. 
\end{theorem}

\begin{theorem}
    If $(a_n)$ is a sequence with absolutely convergent series, $\displaystyle\sum_{n:a_n\geq0} a_n = A$ and $\displaystyle\sum_{n:a_n\leq0} a_n = B$ with all arrangements of $(a_n)$ converging to $A+B$.
\end{theorem}
 
\subsection{Power series}
Throughout this subsection $[0,\infty] := [0,\infty)\cup\{+\infty\}$.

\begin{definition}[Power series]
    For $z\in\CC$ and a complex sequence $(a_n)$, a \textbf{power series} is an expression in the form $\displaystyle\sum_{n=1}^\infty a_nz^n$.
\end{definition}

\begin{definition}[Radius of convergence]
    Given the power series $\displaystyle\sum_{n=1}^\infty a_nz^n$, there exists some $R\in[0,\infty]$ such that: \begin{itemize}
        \item $|z|<R \implies \displaystyle\sum_{n=1}^\infty a_nz^n$ converges,
        \item $|z|>R \implies \displaystyle\sum_{n=1}^\infty a_nz^n$ diverges.
    \end{itemize}
    We cannot tell what happens when $|z|=R$ so this has to be checked separately. $R$ is the \textbf{radius of convergence} of the power series.
\end{definition}

\begin{corollary}
    Given the same power series $\displaystyle\sum_{n=1}^\infty a_nz^n$, have $S:= \{|z|\in\RR^{\geq0}:a_nz^n\rightarrow0\}$ then \[
    R := \begin{dcases}
        \sup(S) & \text{\textcolor{black}{if }} S \ \text{\textcolor{black}{is bounded}} \\
        \hfil\infty & \hfil\text{\textcolor{black}{otherwise}}
    \end{dcases}
    \textcolor{black}{.}
    \] is the radius of convergence for the power series.
\end{corollary}

\begin{theorem}[Evaluating radius of convergence from tests]
    For the power series $\displaystyle\sum_{n=1}^\infty a_nz^n$: \begin{itemize}
        \item if $\ \displaystyle\abs{\frac{a_{n+1}}{a_n}}\rightarrow a\in[0,\infty]$ then $\displaystyle R=\frac{1}{a}$ is the radius of convergence for the power series,
        \item if $\ \displaystyle\abs{a_n}^\frac{1}{n}\rightarrow a\in[0,\infty]$ then $\displaystyle R=\frac{1}{a}$ is the radius of convergence for the power series,
    \end{itemize}
\end{theorem}

\begin{definition}[Cauchy product]
    Given two series $\displaystyle\sum_{n=1}^\infty a_n$, $\displaystyle\sum_{n=1}^\infty b_n$; their \textbf{Cauchy product} is the series $\displaystyle\sum_{n=0}^\infty\sum_{i=0}^n a_ib_{n-i}$.
\end{definition}

\begin{remark}
     If $(a_n)$, $(b_n)$ are the coefficients for a power series, then the Cauchy product of their series will be the coefficients of the product of the power series.
\end{remark}

\begin{theorem}
    If $\displaystyle\sum_{n=1}^\infty a_n$, $\displaystyle\sum_{n=1}^\infty b_n$ are absolutely convergent their Cauchy product converges absolutely to $\displaystyle\br{\sum_{n=1}^\infty a_n}\br{\sum_{n=1}^\infty b_n}$.
\end{theorem}

\begin{theorem}
    If the power series $\displaystyle\sum_{n=1}^\infty a_nz^n$, $\displaystyle\sum_{n=1}^\infty b_nz^n$ have radii of convergence $R_a,R_b$ respectively then their Cauchy product has radius of convergence $R_c\geq\min(R_a,R_b)$.
\end{theorem}

\subsection{Exponential series}

\begin{definition}
    For $z\in\CC$, its \textbf{exponential series} is \[
    E(z) := \sum_{n=}^\infty \frac{z^n}{n!} = 1 + z + \frac{z^2}{2!} + \frac{z^3}{3!} + \ldots
    \textcolor{black}{,}
    \] with $E(z)$ converging absolutely for all $z\in\CC$.
\end{definition}

\begin{theorem}[Properties of exponential series]
    For all $z,w \in \CC$: \\ \begin{enumerate*}
        \item $E(z)E(w)=E(z+w)$,
        \item $\displaystyle\frac{1}{E(z)}=E(-z)$,
        \item $E(z)\neq0$.
    \end{enumerate*}
\end{theorem}

\begin{theorem}
    For all $x\in\QQ$, $E(x)=e^x$, with $e:=E(1)$.
\end{theorem}

\section{Continuity}

\subsection{Continuous functions}
\begin{definition}[Limit of real functions]
    For a function $f:\RR\rightarrow\RR$ and some $a,b\in\RR$ we have \textbf{$f(x)\rightarrow b$ as $x\rightarrow a$} of \textbf{$\lim\limits_{x\rightarrow a}f(x)=b$} iff: \[
        \forall\epsilon>0, \ \exists\delta>0 \ \text{\textcolor{black}{such that }} |x-a|<\delta \impliedby |f(x)-b|<\epsilon
        \textcolor{black}{.}
    \]
\end{definition}

\begin{definition}[Continuity of real functions]
    Given the function $f:\RR\rightarrow\RR$\begin{enumerate}
        \item $f$ is \textbf{continuous at a point $a\in\RR$} iff $\lim\limits_{x\rightarrow a}f(x) = f(a)$,
        \item $f$ is \textbf{continuous (on $\RR$)} iff $f$ is continuous at all $a\in\RR$.
    \end{enumerate}
\end{definition}

\begin{definition}[Discontinuity of real functions]
    The function $f:\RR\rightarrow\RR$ is \textbf{discontinuous} at a point if it is not continuous at that point.
\end{definition}

\begin{definition}[Sequential continuity]
    The function $f:\RR\rightarrow\RR$ is continuous at $a\in\RR \iff f(a_n)\rightarrow f(a)$ as $n\rightarrow\infty$ for all sequences $(a_n)$ converging to $a$.
\end{definition}

\begin{remark}
    The definition for limits and continuity of complex functions is similar with $|\cdot|$ being the modulus instead of the absolute values. The same definitition also applies for functions that are continuous on certain subsets of $\RR$ or $\CC$.
\end{remark}

\begin{theorem}
    $E:\CC\rightarrow\CC$ given by $\displaystyle E(z):=\sum_{n=}^\infty \frac{z^n}{n!}$ is continuous on $\CC$.
\end{theorem}

\begin{theorem}[Properties of the real exponential function]
    Given the exponential function $E:\RR\rightarrow(0,\infty)$: \begin{enumerate}
        \item for all $x\in\RR$, $E(x)>0$,
        \item $x>0\implies E(x)>1$,
        \item $E(x)$ is a strictly increasing function,
        \item For $|x|<1$, $\displaystyle |E(x)-1|\leq \frac{|x|}{1-|x|}$,
        \item $E$ is a continuous bijection.
    \end{enumerate}
\end{theorem}

\begin{theorem}
    The inverse of $E(x)=e^x$ is the \textbf{natural logarithm} function $\ln:(0,\infty)\rightarrow\RR$ satisfying $y=\ln x \iff x = e^y$ for all $x,y\in\RR$.
\end{theorem}

\begin{definition}[Exponentiation of positive bases]
    For $a\in(0,\infty)$, for all $x\in\RR$ define $a^x:=E(x\ln a)$.
\end{definition}

\begin{definition}[Trigonomentric functions]
    The \textbf{sine} and \textbf{cosine} functions are defined as: \[
    \sin(\theta):=\Im[E(i\theta)] \textcolor{black}{,} \qquad \cos(\theta):=\Re[E(i\theta)]
    \textcolor{black}{.}
    \] and are both continuous functions from $\RR\rightarrow[-1,1]$.
\end{definition}

\begin{theorem}[Continuity of piecewise functions]
    For $a,c\in\RR$ with functions $f_1:(-\infty,a)\rightarrow\RR$ and $f_2:(a,\infty)\rightarrow\RR$, the \textbf{piecewise function} $f:\RR\rightarrow\RR$, defined as, \[
        f(x) := \begin{dcases}
            f_1(x) & \text{\textcolor{black}{if }} x<a \\
            \ \hfil c\hfil & \text{\textcolor{black}{if }} x=a \\
            f_2(x) & \text{\textcolor{black}{if }} x>a \\
        \end{dcases}
    \] is continuous on $\RR$ iff both $f_1$ and $f_2$ are continuous on their respective domains and 
    \begingroup\belowdisplayskip=-15pt
    \[
    \lim_{x\uparrow a}f_1(x) = \lim_{x\downarrow a}f_2(x) = c
    \textcolor{black}{.}
    \]
    \endgroup
\end{theorem}

\subsection{Properties of continuity}

\begin{theorem}
    For $f,g:\RR\rightarrow\RR$ continuous at $a\in\RR$ the following functions are also continuous at $a$: \\ \begin{enumerate*}
        \item $\alpha f$ for all $\alpha\in\RR$; \hspace{100pt}
        \item $f+g, f \cdot g$; \hspace{100pt}
        \item $\displaystyle\frac{f}{g}$, given $g(a)\neq0$.
    \end{enumerate*}
\end{theorem}

\begin{samepage}
\begin{theorem}
    The following functions (all by their well known definitions) are continuous: \begin{enumerate}
        \item $f(x)=x^n$, for $n\in\NN_0$ (\textbf{monomials});
        \item $\displaystyle p(x)=\sum_{i=1}^na_ix^i$, given $(a_n)$ is a real sequence (\textbf{polynomials});
        \vspace{-10pt}
        \item $\displaystyle \frac{p(x)}{q(x)}$  at $a\in\RR$ given $p,q$ are polynomials with $q(a)\neq 0$ (\textbf{rational functions});
        \item $\sin(x)$, $\cos(x)$ on $\RR$ and $\tan(x)$ whenever $\cos(x)\neq0$, plus their reciprocals under similar conditions;
        \item $f\circ g$ at $a\in\RR$ when $g$ is continuous at $a$ and $f$ is continuous at $g(a)$.
        \end{enumerate}
\end{theorem}
\end{samepage}

\begin{theorem}[Intermediate value theorem]
    Given $a,b\in\RR$ with $a\leq b$, if $f:[a,b]\rightarrow\RR$ is continuous, then for all $c$ between $f(a)$ and $f(b)$ there exists some $x\in[a,b]$ such that $f(x)=c$.
\end{theorem}

\begin{definition}[Boundedness of real functions]
    Given some $S\subseteq\RR$ a function $f:S\rightarrow\RR$ is \textbf{bounded above} iff $\exists M\in\RR$ such that $f(x)\leq M$ for all $x\in\RR$. The definitions for \textbf{bounded below} and \textbf{bounded} extend naturally from this.
\end{definition}

\begin{theorem}[Extreme value theorem]
    Given $a,b\in\RR$ with $a\leq b$, if $f:[a,b]\rightarrow\RR$ is continuous then $f$ is bounded.
\end{theorem}

\section{Properties of subsets}

\subsection{Open sets}

\begin{definition}[Open sets]
    A set $S\subset\RR$ is \textbf{open} iff $\forall x\in S, \ \exists\delta$ such that $(x-\delta,x+\delta)\subset S$.
\end{definition}

\begin{theorem}[Union of open sets]
    For a collection of open sets in $\RR$, $\{S_i\}$, given the indexing set $\III$ (could be countable or uncountable), $\bigcup\limits_{i\in\III}S_i$ is open in $\RR$.
\end{theorem}

\begin{theorem}[Finite intersections of open sets]
    The intersection of finitely many open sets in $\RR$ is open in $\RR$.
\end{theorem}

\subsection{Closed and compact sets}

\begin{definition}[Closed sets]
    A set $S\subset\RR$ is \textbf{closed} in $\RR$ if all convergent subsequences of $S$ have a limit in $S$.
\end{definition}

\begin{definition}[Compact sets]
    A set $S\subset\RR$ is \textbf{compact} in $\RR$ if it is closed and bounded in $\RR$.
\end{definition}

\begin{theorem}
    The complement of an open set is closed.
\end{theorem}

\begin{remark}
    Not every set in $\RR$ is either open or closed. Half-open intervals are neither open nor closed while $\RR$ and $\emptyset$ are both open and closed.
\end{remark}

\begin{theorem}
    The finite union or any intersection of closed sets in $\RR$ is closed.
\end{theorem}

\begin{theorem}
    A set $S\subset\RR$ is compact iff every subsequence of $S$ has as convergent subsequence $x_{n(i)}\rightarrow x\in S$.
\end{theorem}

\begin{theorem}[Extreme value theorem for compact sets]
    If $S\subset\RR$ is compact with $f:S\rightarrow\RR$ continuous, there exists some $c,d\in S$ with $f(c)=\inf\limits_{x\in S} f(x)$ and $f(d)=\sup\limits_{x\in S} f(x)$.
\end{theorem}

\section{Uniform continuity and convergence}

\subsection{Uniform continuity}

\begin{definition}[Uniform continuity]
    A fuction $f:S\rightarrow\RR$ is \textbf{uniformly continuous} iff \[
        \forall\epsilon>0, \ \exists\delta>0 \ \text{\textcolor{black}{such that }} \forall x,y\in S, |x-y|<\delta\implies|f(x)-f(y)|<\epsilon
        \textcolor{black}{.}
    \] Uniform continuity is a more powerful notion that continuity with $f$ is uniformly continuous $\implies$ $f$ is continuous.
\end{definition}

\begin{theorem}
    If $S\subset\RR$ is compact and $f:S\rightarrow\RR$ continuous then $f$ is uniformly continuous.
\end{theorem}

\subsection{Convergence of sequences of functions}

\begin{definition}[Pointwise convergence]
    For some $S\subset\RR$ with the sequence $f_1,f_2,\ldots:S\rightarrow \RR$, $f_n$ \textbf{converges pointwise} to some $f:S\rightarrow\RR$ if \[
        \forall x\in S,\ \forall\epsilon>0, \  \exists N\in\NN \ \text{\textcolor{black}{such that }}  \forall n\geq N, \ |f(x)-f_n(x)|<\epsilon
    \textcolor{black}{.}
    \] Written $\forall x\in S, \lim\limits_{n\rightarrow\infty}f_n(x)=f(x)$.
\end{definition}

\begin{definition}[Uniform convergence]
    For some $S\subset\RR$, the sequence $f_1,f_2,\ldots:S\rightarrow \RR$ \textbf{uniformly converges} to some $f:S\rightarrow\RR$ if\[
        \forall\epsilon>0, \ \exists N\in\NN  \ \text{\textcolor{black}{such that }} \forall x\in S, \ \text{\textcolor{black}{and }} \forall n>N, |f(x)-f_n(x)|<\epsilon
    \textcolor{black}{.}
    \]
\end{definition}

\begin{theorem}
    If a sequence of (uniformly) continuous functions converges uniformly to a function $f$ then $f$ is (uniformly) continuous.
\end{theorem}

\begin{theorem}
    If, given $S\subset\RR$, $(f_n):S\rightarrow\RR$ is a uniformly convergent sequence of continuous functions with $a\in S$ open in $S$, $\lim\limits_{n\rightarrow\infty}\lim\limits_{x\rightarrow a}f_n(x)=\lim\limits_{x\rightarrow a}\lim\limits_{n\rightarrow\infty}f_n(x)$.
\end{theorem}

\subsection{Convergence of series of functions}

\begin{definition}[Convergence of series of functions]
    Given $(f_n):S\rightarrow\RR$ defined on $S\subset\RR$, the series $\displaystyle\sum_{n=1}^\infty f_n(x)$ \textbf{converges (uniformly)} iff the sequence of partial sums $S_n(x)=\displaystyle\sum_{n=1}^n f_n(x)$ converges (uniformly).
\end{definition}

\begingroup\belowdisplayskip=-8pt
\begin{theorem}[Weierstrass M-test]
    Given continuous $(f_n):S\rightarrow\RR$ defined on $S\subset\RR$, \[
    \forall x\in S \ \text{\textcolor{black}{and }} \forall i\in\NN, \ \exists M_1,M_2,\ldots \in\RR \
\text{\textcolor{black}{ such that }} |f_i(x)|\leq M_i \ \text{\textcolor{black}{and }} \displaystyle\sum_{i=1}^\infty M_i \ \text{\textcolor{black}{ converges }}\]\[ \implies \displaystyle\sum_{n=1}^\infty f_i(x) \ \text{\textcolor{black}{converges uniformly to some continuous }} g:S\rightarrow\RR
\textcolor{black}{.}
\]
\end{theorem}
\endgroup

\section{Differentiation}

\subsection{Differentiability}

\subsection{Operations on derivatives}

\subsection{Local extrema and mean values}

\subsection{L'Hôpital's rule}

\subsection{Taylor's theorem}

\subsection{Convexity}

\subsection{Exchange of limits and derivatives}

\subsection{Derivation of pi}

\section{Integration}

\subsection{Partitions}

\subsection{Darboux sums}

\subsection{Darboux integral}

\subsection{Integrability}

\subsection{Properties of integration}

\subsection{Fundamental theorem of calculus}

\subsection{Exchange of limits and integrals}

\subsection{Improper integrals}

\subsection{Lebesgue criterion}

\subsection{Irrationality of pi}

\end{document}